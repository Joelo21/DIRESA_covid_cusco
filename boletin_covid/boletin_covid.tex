\documentclass[12pt,a4paper,openany]{book}
%------------------------------------------------------------------------------------------------------------------------------------------------------------------------------------------------------------------------------
% PACKAGES
%--------------------------------------------------------------------------------------------------------------------------------------------------------------------------------------------------------------------------------

% Selección de idioma
\usepackage[spanish]{babel}

% algo
\usepackage[utf8]{inputenc}

% Par acomodar la foto del editorial
\usepackage{wrapfig}

% Paquetes útiles
\usepackage[table]{xcolor}
\usepackage{amssymb}
\usepackage{amsmath}
%\usepackage{mathbbol}
\usepackage{bbm}
\usepackage{amsthm}
\usepackage{pdfpages}
\usepackage{graphicx,color,psfrag}
\usepackage{epstopdf}
\usepackage{pdflscape}
\usepackage{tabularx}
\usepackage{longtable}
\usepackage{breakurl}
\usepackage{enumitem}
\usepackage[normalem]{ulem}
\usepackage{blindtext}
\usepackage{mathtools,breqn}


\usepackage{fancyhdr}
\usepackage{graphicx}

\usepackage{fnpct}

\usepackage{subcaption}

\usepackage{newpxtext}
\usepackage{lscape}

% La mejor fuente de la letra: https://www3.gobiernodecanarias.org/medusa/ecoblog/lortrodm/files/2015/03/tarea-formatos-word.pdf
% caption fonts
%\usepackage[font={large,bf}]{caption} 
\usepackage[T1]{fontenc}
\usepackage{verdana}


\usepackage{setspace}
\usepackage{longtable}
\usepackage{threeparttable}  
\usepackage{tabulary}
\usepackage{booktabs}
\usepackage{float}
\usepackage{caption}
\usepackage{subcaption}
\usepackage{rotating}
\usepackage[titletoc,title]{appendix}

\usepackage{array,multirow}

\usepackage[round]{natbib}
\bibpunct{(}{)}{;}{a}{,}{;}
\setcounter{MaxMatrixCols}{10}

\topmargin=-1.8cm \textheight=23.8cm \oddsidemargin=-0.3cm
\evensidemargin=-0.5cm \textwidth=17.1cm

\newtheorem{theorem}{Theorem}
\newtheorem{corollary}[theorem]{Corollary}
\newtheorem{proposition}{Proposition}
\newtheorem{assumption}{Assumption}
\newtheorem{assumption2}{Assumption A}

\newtheorem{lemma}{Lemma}

\usepackage{tikz}
\usetikzlibrary{positioning}
\tikzset{>=stealth}
\usepackage{amsmath}
\usepackage{verbatim}
\usetikzlibrary{arrows,shapes}

% Definir colores
\definecolor{mycolor1}{RGB}{221, 165, 230}
\definecolor{mycolor2}{RGB}{54, 56, 120}	
\definecolor{mycolor3}{RGB}{205, 24, 24}
\definecolor{mycolor4}{RGB}{164, 93, 93}
\definecolor{mycolor5}{RGB}{243, 149, 13}
\definecolor{mycolor6}{RGB}{3, 83, 151}
\definecolor{mycolor7}{RGB}{52, 103, 81}
\usepackage[colorlinks=true,linkcolor=myblue, allcolors=mycolor2]{hyperref}
\usepackage{soul}

% Tablas
\usepackage{tabularx}
\usepackage{multirow}
\usepackage{multicol} 
\usepackage{booktabs}%\usepackage{booktabs, calc} %This is the package to use to have nice-looking tables. More documentation on the tables in LateX: https://www.tug.org/pracjourn/2007-1/mori/mori.pdf
\usepackage{threeparttable} 

\usepackage{lmodern}
\usepackage{booktabs}
\usepackage{pgfplots}

\graphicspath{{../figuras/}}

\begin{document}
	
	%---------------------------------------------------------------------------
	% TITLE PAGE
	%---------------------------------------------------------------------------
	\doublespacing
	
	\title{Boletín COVID-19}
	\author{Autores}
	
	\date{}
	
	%\maketitle
	
	
	%\thispagestyle{empty}\baselineskip1.385\baselineskip \newpage{}
	
	\pagestyle{plain}\pagenumbering{arabic}
	
	%insertar el cover
	\includepdf[pages={1}]{../editorial/portada.pdf}
	\clearpage
	
	\pagestyle{plain}\pagenumbering{arabic}
	
	\clearpage
	
	
	\begin{center}
		
		{\large Gerencia Regional de Salud}
		
		\textbf{MSP. Javier Ramírez Escóbar}
		
		Gerente Regional \vspace{1.0cm}
		
		Dirección Ejecutiva de Inteligencia Sanitaria
		
		\textbf{MSP. Darío Francisco Navarro Mendoza}
		
		Director
		
		\vspace{1.5cm}
		\noindent
		\begin{minipage}[t]{.45\textwidth}
			\centering
			Dirección de Epidemiología e Investigación  \\
			\textbf{MSC. Fátima R. Concha Velasco}\\
			Directora \vspace{1.0cm}\\
			% Por orden alfabético del apellido
			\textit{Equipo de Epidemiología e Investigación }\vspace{.5cm}\\
			Bach. Eddie Briam Cassa Chavez \\
			Lic. Nadia Isabel Cáceres Pillco \\
			TAP. Edgar Waldo Capcha Salcedo \\
			M.S.P. Pablo Fidel Grajeda Ancca \\
			M.C. Alex Jaramillo Corrales \\ 
			M.C. Katia Luque Quispe \\
			M.C. Ana Gabriela Eulalia Moncada Arias \\
			M.C. Jesus Kevin Perez Castilla \\
			Lic. Enf. Ruth Nelly Oscco Abarca \\
			Ing. Joel Wilfredo Sumerente Ayerbe \\
			Lic. Enf. Guinetta Margarita Yabar Herrera \vspace{1.5cm}\\	
		\end{minipage}
		\hfill
		\noindent
		\begin{minipage}[t]{.45\textwidth}
			\centering
			Dirección de Estadística, Informática y Telecomunicaciones\\
			\textbf{Ing. Abel Rimasca Chacón} \\
			Director \vspace{1.0cm} \\
			% Por orden alfabético del apellido
			\textit{Equipo de Estadística, Informática y Telecomunicaciones} \vspace{.5cm} \\
			Ing. Iván Atayupanqui Rondón \\
			Ing. Miguel Ángel Campana Alarcón \\
			Ing. Uriel Lacuta Farfán \\
			Ing. Jorge Fernando Lovatón Ramos \\
			Ing. Danny Robert Moscoso Sánchez \\
			Lic. Ray Milton Valderrama Álverez \vspace{1.5cm}\\
		\end{minipage}
		Secretaria: Sra. Ruth Baca Mendoza
	\end{center}
	\let\cleardoublepage\clearpage
	\tableofcontents
	\begin{center}
		Visite nuestro Dashboard interactivo sobre COVID-19 haciendo clic \href{https://geresacusco.shinyapps.io/DASHBOARD\_COVID-19\_CUSCO/}{AQUÍ}
	\end{center}
	
	%\mainmatter
	%---------------------------------------------------------------------------
	% CAPÍTULO: EDITORIAL
	%---------------------------------------------------------------------------
	
	\pagebreak
	
	\section*{Editorial}	\addcontentsline{toc}{chapter}{Editorial}
	\begin{wrapfigure}{l}{5cm}
		\label{wrap-fig:1}
		\includegraphics[width=5 cm]{../editorial/editorial_noemi_cdc}
		\caption*{
			\centering
			Lic. Noemi Flores  Jaime		
			\textit{Equipo técnico de la Unidad Técnica de EDA/IRA/SGB, Influenza  y COVID-19 - CDC Perú}
			
		}
	\end{wrapfigure}
	\noindent \textbf{Vigilancia Ambiental del SARS-COV-2 para Complementar la Vigilancia de la Salud Pública}
	
	
	La vigilancia de las aguas residuales se ha convertido en una herramienta útil en la respuesta de salud pública a la pandemia de COVID-19 (1). Las graves consecuencias para la salud y la propagación global de la pandemia de COVID-19 han requerido el rápido desarrollo de programas de vigilancia para informar las respuestas de salud pública. Los esfuerzos para respaldar la capacidad de vigilancia han incluido una respuesta de investigación mundial sin precedentes sobre el uso de señales genéticas del SARS-CoV-2 en aguas residuales luego de la demostración inicial de la detectabilidad del virus en aguas residuales a principios de 2020 (2).
	"Este tipo de vigilancia puede proporcionar pruebas complementarias de que el virus del SARS-CoV-2 está circulando", "El muestreo y las pruebas sistemáticas de las aguas residuales no tratadas pueden ser una forma complementaria y no invasiva de vigilar la presencia del virus (3). 
	La vigilancia de las aguas residuales no es nueva y el método se ha aplicado a patógenos como el poliovirus o los enterovirus, así como para comprobar la presencia de bacterias resistentes a los antibióticos. Utiliza la prueba PCR para detectar el material genético del virus en las aguas residuales de la comunidad procedentes de los sistemas municipales.
	Los resultados sugieren que la epidemiología basada en aguas residuales (WBE siglas en inglés) es una alerta de alerta temprana valiosa y una herramienta de vigilancia complementaria útil para la respuesta de salud pública, para adaptar las medidas de contención y mitigación y para determinar las poblaciones objetivo para la prueba. En entornos de saneamiento deficiente, los ríos contaminados podrían utilizarse alternativamente como fuente para la vigilancia ambiental (4). 
	La vigilancia continua de la difusión de COVID-19 sigue siendo crucial para controlar su difusión y anticipar las olas de infección. La detección de la carga de ARN viral en muestras de aguas residuales se ha sugerido como un enfoque eficaz para el seguimiento de epidemias y el desarrollo de un sistema de alerta eficaz. Sin embargo, su vínculo cuantitativo con el estado epidémico y las etapas del brote aún es esquivo (5).
	La Organización Mundial de la Salud (OMS) ofrece asesoramiento a los países sobre el muestreo y las pruebas para detectar el virus del SARS-CoV-2 en aguas residuales no tratadas como parte de la vigilancia ambiental para complementar las estrategias de control de la COVID-19. Laboratorios de Argentina, Brasil, Canadá, Chile, Costa Rica, Colombia, Ecuador, México, Perú y Estados Unidos se encuentran entre los que vigilan la presencia del virus con este método en la región de las Américas (3).
	En el marco de la emergencia sanitaria por COVID-19, a iniciativa del Ministerio de Vivienda, Construcción y Saneamiento (MVCS) se desarrolla la herramienta de alerta temprana a partir del monitoreo epidemiológico del SARS-COV-2 en aguas residuales. Este estudio cuenta con el apoyo del Programa SECOSAN de la Cooperación Suiza – SECO1, y se aplica en dos ámbitos de estudio (i) Lima Metropolitana y (ii) Arequipa Metropolitana (6) .Son veinte (20) puntos de monitoreo distribuidos en los dos ámbitos de estudio, en cuanto a Lima Metropolitana comprende cinco (05) PTAR y nueve (09) colectores de la red de alcantarillado; en cuanto a Arequipa Metropolitana comprende dos (02) PTAR y cuatro (04) colectores de la red de alcantarillado. El detalle de estos puntos de monitoreo, su ubicación y los distritos de donde se recolectan las aguas residuales.
	A partir de los resultados obtenidos, se evidencia que la epidemiología basada en aguas residuales para la detección del SARS-CoV-2 aplicada en Lima Metropolitana y Arequipa Metropolitana, permite al sector saneamiento contribuir con información de utilidad a la estrategia de salud pública, con la identificación anticipada de brotes, concentración y la evolución del virus en zonas localizadas de la ciudad.  
	
\textbf{	Referencias Bibliográficas }
	
	1. 	Xiao A, Wu F, Bushman M, Zhang J, Imakaev M, Chai PR, et al. Metrics to relate COVID-19 wastewater data to clinical testing dynamics. MedRxiv Prepr Serv Health Sci. 16 de junio de 2021;2021.06.10.21258580. 
	
	
	2. 	Hrudey SE, Conant B. The devil is in the details: emerging insights on the relevance of wastewater surveillance for SARS-CoV-2 to public health. J Water Health. enero de 2022;20(1):246-70. 
	
	
	3. 	Organización Panamericana de la Salud. Vigilancia ambiental: herramienta complementaria para el seguimiento de la COVID-19 - OPS/OMS | Organización Panamericana de la Salud [Internet]. [citado 29 de abril de 2022]. Disponible en: https://www.paho.org/es/noticias/29-4-2022-vigilancia-ambiental-herramienta-complementaria-para-seguimiento-covid-19
	
	
	4. 	Aguiar-Oliveira M de L, Campos A, R Matos A, Rigotto C, Sotero-Martins A, Teixeira PFP, et al. Wastewater-Based Epidemiology (WBE) and Viral Detection in Polluted Surface Water: A Valuable Tool for COVID-19 Surveillance-A Brief Review. Int J Environ Res Public Health. 10 de diciembre de 2020;17(24):E9251. 
	
	
	5. 	Proverbio D, Kemp F, Magni S, Ogorzaly L, Cauchie HM, Gonçalves J, et al. Model-based assessment of COVID-19 epidemic dynamics by wastewater analysis. Sci Total Environ. 1 de marzo de 2022;827:154235. 
	
	
	6. 	Ministerio de Vivienda, Construcción y Saneamiento. Boletín Semanal No 3. Monitoreo epidemiológico del SARS-CoV-2 en aguas residuales [Internet]. [citado 3 de mayo de 2022]. Disponible en: https://www.gob.pe/institucion/vivienda/informes-publicaciones/2210676-boletin-semanal-n-3-monitoreo-epidemiologico-del-sars-cov-2-en-aguas-residuales
	
	
	
	%---------------------------------------------------------------------------
	% CAPÍTULO: CARACTERÍSTICAS GENERALES
	%-------------------------------- -------------------------------------------
	%insertar el cover del capitulo
	\includepdf[pages={1}]{../editorial/2.pdf}
	\clearpage	
	\section*{Características Generales}
	\addcontentsline{toc}{chapter}{Características Generales}
	
	
	
	\noindent En la Figura \ref{fig:casos_edad_sexo} se muestra la cantidad de casos confirmados de COVID-19, por prueba antigénica y molecular por grupo etario (en intervalos de 10 años) y sexo. El grupo etario con mayor número de casos acumulados es el de 30 a 39 años, con sólo 1 caso más reportado en relación al anterior reporte (12 114 casos acumulados). 	Es preciso recalcar que en el intervalo de edad de 10 a 89 años la mayor cantidad de casos diagnosticados corresponden al sexo femenino.  
	
	
	\begin{figure}[h]
		\caption{Casos Confirmados de COVID-19 según Grupo de Edad y Sexo en la Región Cusco hasta la SE 16-2022(*).}\label{fig:casos_edad_sexo}
		\begin{center}
			\includegraphics[width=0.75\linewidth]{../figuras/casos_etapavida_2022}
		\end{center}
		{\footnotesize {Fuente de datos: SISCOVID, NOTICOVID.(*)Sólo se incluye información del 2022.}}
	\end{figure}
	\pagebreak
	
	
	La Figura \ref{fig:fallecidos_edad_sexo}  muestra el número de muertes reportadas por COVID-19 conforme al grupo etario y sexo hasta la SE 16. Se observa que el mayor número de muertes se registró en el grupo etario de 80 a 89 años (72 muertes acumuladas) seguido del grupo etario de 70 a 79 años (43 muertes acumuladas).
	
	\begin{figure}[h]
		\caption{Casos fallecidos por COVID-19 según Grupo Etario y Sexo en la Región Cusco hasta la SE 16-2022(*).}\label{fig:fallecidos_edad_sexo}
		\begin{center}
			\includegraphics[width=0.75\linewidth]{../figuras/defunciones_etapavida_2022}
		\end{center}
		{\footnotesize {Fuente de datos: SISCOVID, NOTICOVID.(*) Sólo se incluye información del 2022.}}
	\end{figure}
	
	
	
	\cleardoublepage
	%---------------------------------------------------------------------------
	% CAPÍTULO: CARACTERÍSTICAS CLÍNICAS
	%---------------------------------------------------------------------------
	%insertar el cover del capitulo
	\includepdf[pages={1}]{../editorial/3.pdf}
	
	\clearpage
	
	\section*{Características Clínicas}
	\addcontentsline{toc}{chapter}{Características Clínicas}	
	\noindent En la Figura \ref{fig:sintomas} se presentan los síntomas más frecuentes autorreportados por los pacientes con diagnóstico de COVID-19, el dolor de garganta (18,9 $\%$) es el síntomas más reportado, seguido de tos (18,5 $\%$) y malestar (13,4$\%$). Dentro de los signos más frecuentes (Figura \ref{fig:signos}), el exudado faríngeo constituye el signo más prevalente (84,6$\%$). 
	
	\begin{figure}[h]
		\caption{Síntomas más frecuentes de los pacientes diagnosticados por COVID-19 en la Región Cusco hasta la SE 16-2022.  }\label{fig:sintomas}
		\begin{center}
			\includegraphics[width=0.85\linewidth]{../figuras/figura_sintoma.pdf}
		\end{center}
		{\footnotesize {Fuente de datos: SISCOVID, NOTICOVID.}}
	\end{figure}
	
	\begin{figure}[h]
		\caption{Signos más frecuentes de los pacientes diagnosticados por COVID-19 en la Región Cusco hasta la SE 16-2022.}\label{fig:signos}
		\begin{center}
			\includegraphics[width=0.65\linewidth]{../figuras/figura_signo.pdf}
		\end{center}
		{\footnotesize {Fuente de datos: NOTICOVID.}}
	\end{figure}
	
	La Figura \ref{fig:comorbilidades} muestra la frecuencia de comorbilidades autoreportadas por los pacientes con COVID-19, siendo las más prevalentes; la obesidad (28,5$\%$), diabetes (24,6$\%$) y las comorbilidades cardiovasculares (16,3$\%$).  
	\begin{figure}[h]
		\caption{Comorbilidades más frecuentes de los pacientes diagnosticados por COVID-19 en la Región Cusco hasta la SE 16-2022. }\label{fig:comorbilidades}
		\begin{center}
			\includegraphics[width=0.65\linewidth]{../figuras/figura_comorbilidad.pdf}
		\end{center}
		{\footnotesize {Fuente de datos: NOTICOVID.}}
	\end{figure}
	\clearpage
	En la Figura \ref{fig:sintomaticos_asintomati} se evidencia la curva epidémica de casos sintomáticos y asintomáticos, comparada con los casos sintomáticos y asintomáticos desde el año 2020. Desde la SE 04 se evidencia una tendencia marcada al descenso de casos tanto asintomáticos como sintomáticos llegando a alcanzar cifras similares al inicio de la pandemia.
	
	\begin{figure}[h]
		\caption{Casos Sintomáticos y Asintomáticos de COVID-19 por Semana Epidemiológica en la Región Cusco, hasta la SE 16-2022.  }\label{fig:sintomaticos_asintomati}
		
		\begin{center}
			\includegraphics[width=0.75\linewidth]{../figuras/sintomaticos_20_21_22.png}
		\end{center}
		{\footnotesize {Fuente de datos: SISCOVID, NOTICOVID.}}
	\end{figure}
	\clearpage
	
	
	%---------------------------------------------------------------------------
	% CAPÍTULO: ANÁLISIS DE INDICADORES
	%---------------------------------------------------------------------------
	%insertar el cover del capitulo
	\includepdf[pages={1}]{../editorial/4.pdf}
	\clearpage
	
	\section*{Análisis de Indicadores}
	\addcontentsline{toc}{chapter}{Análisis de Indicadores}
	\subsection*{Tasa de Incidencia y Tasa de Positividad}
	\noindent La evolución de la tasa de incidencia a lo largo del tiempo se encuentra graficada en la Figura \ref{fig:incidencia}, se evidencia que tras alcanzar su pico en las primeras semanas del 2022, la tasa de incidencia presenta una marcada tendencia al descenso llegando a alcanzar la cifra más baja reportada desde el 2021 en la semana 17 (4 casos/ 1 000 000 de personas).
	\begin{figure}[h]
		\caption{Tasa de Incidencia de COVID-19 en la región Cusco hasta la SE 17-2022(*).  }\label{fig:incidencia}
		\begin{center}
			\includegraphics[width=0.80\linewidth]{../figuras/tasa_incidencia_2021_2022.png}
		\end{center}
		{\footnotesize {Fuente de datos: SISCOVID, NOTICOVID. (*) Se considera como caso positivo sólo a los pacientes con prueba molecular o antigénica positiva.}}
	\end{figure}
	
	La Figura \ref{fig:total_muestras_procesada} muestra un comparativo de las tasas de positividad ($\%$) de pruebas moleculares (PCR) y antigénicas (AG). Se observa que desde la SE 07 ambas tasas de positividad se encuentran en descenso sostenido, la tasa de incidencia de pruebas antigénicas llegó a alcanzar su menor valor desde la SE 12 (1$\%$), del mismo modo las pruebas moleculares presentan tasas de positividad menor del 5 $\%$ desde la SE 12. 
	
	\begin{figure}[h]
		\caption{Tasa de positividad para muestras antigénicas y moleculares por COVID-19 en la región Cusco hasta la SE 16-2022. }\label{fig:total_muestras_procesada}
		\begin{center}
			\includegraphics[width=0.75\linewidth]{../figuras/positividad_diaria_2021_2022.png}
		\end{center}
		{\footnotesize {Fuente de datos: SISCOVID, NOTICOVID.}}
	\end{figure}
	
	
	Las Figuras \ref{fig:positividad_pcr} y \ref{fig:positividad_ag} muestran el número de positivos detectados por pruebas moleculares y antigénicas y sus tasas de positividad, en ambas pruebas se observa una marcada tendencia al descenso desde la SE 04. 
	
	
	\begin{landscape}
		\begin{figure}[h]
			\caption{Positividad y Tasa de Positividad de pruebas moleculares tomadas por COVID-19 en la región Cusco hasta la SE 16-2022.}\label{fig:positividad_pcr}
			\begin{center}
				\includegraphics[width=0.90\linewidth]{../figuras/positividad_pcr.pdf}
			\end{center}
			{\footnotesize {Fuente de datos: SISCOVID, NOTICOVID.}}
		\end{figure}
	\end{landscape}
	\clearpage
	\begin{landscape}
		
		\begin{figure}[h]
			\caption{ Positividad y Tasa de Positividad de pruebas antigénicas tomadas por COVID-19 en la región Cusco hasta la SE 16-2022.}\label{fig:positividad_ag}
			\begin{center}
				\includegraphics[width=0.90\linewidth]{../figuras/positividad_ag.pdf}
			\end{center}
			{\footnotesize {Fuente de datos: SISCOVID, NOTICOVID.}}
		\end{figure}
	\end{landscape}
	%\clearpage
	%\subsection*{Análisis de Indicadores en población pediátrica}
	%\noindent Las figuras inferiores (Figura \ref{fig:niños_2021} y Figura \ref{fig:niño_2022} ) muestran el número de casos positivos(barras rosadas) y número de muertes (línea negra) en la población pediátrica para los quinquenios respectivos.
	%
	%Para el año 2020 el mayor número de casos positivos en niños se reportó en la SE 35 tras lo cuál el número de casos se ha mantenido variable en el resto del año. Con respecto a las defunciones,  el número máximo de muertes reportada fue de 1 por semana en cada grupo etario. 
	%
	%En el año 2021 el número de casos positivos se ha mantenido mas o menos constante a lo largo del año hasta la SE 52, tras lo cual se incrementa exponencialmente en todos los quinquenios. Del mismo modo el número de muertes no excedió de 1 muerte por semana en los tres grupos etarios.  
	%
	%Finalmente para el año 2022 se evidencia un incremento considerable en 
	%el número de casos positivos en población pediátrica hasta la SE 06, tras lo cuál se observa el descenso en el número de casos, a partir de las SE 07 no se reportaron muertes en el grupo etario de 0 a 15 años.  
	%
	%
	%En la Cuadro \ref{table:1} se evidencia el número de casos positivos, las defunciones y la tasa de letalidad de menores de 15 años, agrupados por quinquenios. Se evidencia que hasta la SE 12 del año 2022 se reportaron 2503 casos positivos y 6 defunciones en este grupo etario, teniendo una letalidad de 0,24 $\%$, porcentaje que es menor a lo reportado en las primeras olas de afección por COVID-19.   
	%
	%\begin{figure}[h]
	%	\caption{Casos y defunciones por quinquenio en población pediátrica 2020-2021.}
	%	\label{fig:niños_2021}
	%	\centering
	%	\begin{subfigure}[b]{0.45\textwidth}
		%		\centering
		%		\includegraphics[width=\textwidth]{../figuras/niños_2020_1.pdf}
		%		\caption{De 0 a 5 años - \textbf{2020}}
		%		%\label{fig:}
		%	\end{subfigure}
	%	\hfill
	%	\begin{subfigure}[b]{0.45\textwidth}
		%		\centering
		%		\includegraphics[width=\textwidth]{../figuras/niños_2021_1.pdf}
		%		\caption{De 0 a 5 años - \textbf{2021}}
		%		%\label{fig:70 a 79 años}
		%	\end{subfigure}
	%	
	%	\vspace{10mm}
	%	\begin{subfigure}[b]{0.45\textwidth}
		%		\centering
		%		\includegraphics[width=\textwidth]{../figuras/niños_2020_2.pdf}
		%		\caption{De 6 a 11 años - \textbf{2020}}
		%		%\label{fig:60 a 69 años}
		%	\end{subfigure}
	%	\hfill
	%	\begin{subfigure}[b]{0.45\textwidth}
		%		\centering
		%		\includegraphics[width=\textwidth]{../figuras/niños_2021_2.pdf}
		%		\caption{De 6 a 10 años - \textbf{2021}}
		%		%\label{fig:50 a 59 años}
		%	\end{subfigure}
	%	
	%	\vspace{10mm}
	%	\begin{subfigure}[b]{0.45\textwidth}
		%		\centering
		%		\includegraphics[width=\textwidth]{../figuras/niños_2020_3.pdf}
		%		\caption{De 11 a 15 años - \textbf{2020}}
		%		%\label{fig:40 a 49 años}
		%	\end{subfigure}
	%	\hfill
	%	\begin{subfigure}[b]{0.45\textwidth}
		%		\centering
		%		\includegraphics[width=\textwidth]{../figuras/niños_2021_3.pdf}
		%		\caption{De 11 a 15 años - \textbf{2021}}
		%		%\label{fig:40 a 49 años}
		%	\end{subfigure}
	%\end{figure}
	%
	%\begin{figure}[h]
	%	\caption{Casos y defunciones por quinquenio en población pediátrica - 2022.}
	%	\label{fig:niño_2022}
	%	\centering
	%	\begin{subfigure}[b]{0.45\textwidth}
		%		\centering
		%		\includegraphics[width=\textwidth]{../figuras/niños_2022_1.pdf}
		%		\caption{De 0 a 5 años - \textbf{2022}}
		%		%\label{fig:40 a 49 años}
		%	\end{subfigure}
	%	
	%	\centering
	%	\begin{subfigure}[b]{0.45\textwidth}
		%		\centering
		%		\includegraphics[width=\textwidth]{../figuras/niños_2022_2.pdf}
		%		\caption{De 6 a 10 años - \textbf{2022}}
		%		%\label{fig:40 a 49 años}
		%	\end{subfigure}
	%	
	%	\vspace{10mm}
	%	\begin{subfigure}[b]{0.45\textwidth}
		%		\centering
		%		\includegraphics[width=\textwidth]{../figuras/niños_2022_3.pdf}
		%		\caption{De 11 a 15 años - \textbf{2022}}
		%		%\label{fig:40 a 49 años}
		%	\end{subfigure}
	%\end{figure}
	%\clearpage
	%	\begin{table}[h]
		%			\caption{Tasa de letalidad de COVID-19 en población pediátrica 2020-2022.}
		%				\label{table:1}
		%		\resizebox{\textwidth}{!}{%
			%				\begin{tabular}{lcccc}
		\cline{2-5}
		\multicolumn{1}{l|}{} &
		\multicolumn{1}{c|}{\cellcolor[HTML]{ECF4FF}Etapa de Vida} &
		\multicolumn{1}{c|}{\cellcolor[HTML]{ECF4FF}Positivos} &
		\multicolumn{1}{c|}{\cellcolor[HTML]{ECF4FF}Defunciones} &
		\multicolumn{1}{c|}{\cellcolor[HTML]{ECF4FF}Letalidad(\%)} \\ \hline
		\multicolumn{1}{|l|}{} &
		\multicolumn{1}{c|}{0 a 5 años} &
		\multicolumn{1}{c|}{851} &
		\multicolumn{1}{c|}{4} &
		\multicolumn{1}{c|}{0.47} \\ \cline{2-5} 
		\multicolumn{1}{|l|}{} &
		\multicolumn{1}{c|}{6 a 10 años} &
		\multicolumn{1}{c|}{705} &
		\multicolumn{1}{c|}{2} &
		\multicolumn{1}{c|}{0.28} \\ \cline{2-5} 
		\multicolumn{1}{|l|}{} &
		\multicolumn{1}{c|}{11 a 15 años} &
		\multicolumn{1}{c|}{1146} &
		\multicolumn{1}{c|}{2} &
		\multicolumn{1}{c|}{0.17} \\ \cline{2-5} 
		\multicolumn{1}{|l|}{\multirow{-4}{*}{2020}} &
		\multicolumn{1}{c|}{\cellcolor[HTML]{ECF4FF}Total} &
		\multicolumn{1}{c|}{\cellcolor[HTML]{ECF4FF}2702} &
		\multicolumn{1}{c|}{\cellcolor[HTML]{ECF4FF}8} &
		\multicolumn{1}{c|}{\cellcolor[HTML]{ECF4FF}0.30} \\ \hline
		&
		\multicolumn{1}{l}{} &
		\multicolumn{1}{l}{} &
		\multicolumn{1}{l}{} &
		\multicolumn{1}{l}{} \\ \cline{2-5} 
		\multicolumn{1}{l|}{} &
		\multicolumn{1}{c|}{\cellcolor[HTML]{ECF4FF}Etapa de Vida} &
		\multicolumn{1}{c|}{\cellcolor[HTML]{ECF4FF}Positivos} &
		\multicolumn{1}{c|}{\cellcolor[HTML]{ECF4FF}Defunciones} &
		\multicolumn{1}{c|}{\cellcolor[HTML]{ECF4FF}Letalidad(\%)} \\ \hline
		\multicolumn{1}{|l|}{} &
		\multicolumn{1}{c|}{0 a 5 años} &
		\multicolumn{1}{c|}{517} &
		\multicolumn{1}{c|}{7} &
		\multicolumn{1}{c|}{1.4} \\ \cline{2-5} 
		\multicolumn{1}{|l|}{} &
		\multicolumn{1}{c|}{6 a 10 años} &
		\multicolumn{1}{c|}{484} &
		\multicolumn{1}{c|}{4} &
		\multicolumn{1}{c|}{0.83} \\ \cline{2-5} 
		\multicolumn{1}{|l|}{} &
		\multicolumn{1}{c|}{11 a 15 años} &
		\multicolumn{1}{c|}{1284} &
		\multicolumn{1}{c|}{3} &
		\multicolumn{1}{c|}{0.23} \\ \cline{2-5} 
		\multicolumn{1}{|l|}{\multirow{-4}{*}{2021}} &
		\multicolumn{1}{c|}{\cellcolor[HTML]{ECF4FF}Total} &
		\multicolumn{1}{c|}{\cellcolor[HTML]{ECF4FF}2285} &
		\multicolumn{1}{c|}{\cellcolor[HTML]{ECF4FF}14} &
		\multicolumn{1}{c|}{\cellcolor[HTML]{ECF4FF}0.61} \\ \hline
		&
		\multicolumn{1}{l}{} &
		\multicolumn{1}{l}{} &
		\multicolumn{1}{l}{} &
		\multicolumn{1}{l}{} \\ \cline{2-5} 
		\multicolumn{1}{l|}{} &
		\multicolumn{1}{c|}{\cellcolor[HTML]{ECF4FF}Etapa de Vida} &
		\multicolumn{1}{c|}{\cellcolor[HTML]{ECF4FF}Positivos} &
		\multicolumn{1}{c|}{\cellcolor[HTML]{ECF4FF}Defunciones} &
		\multicolumn{1}{c|}{\cellcolor[HTML]{ECF4FF}Letalidad(\%)} \\ \hline
		\multicolumn{1}{|l|}{} &
		\multicolumn{1}{c|}{0 a 5 años} &
		\multicolumn{1}{c|}{651} &
		\multicolumn{1}{c|}{5} &
		\multicolumn{1}{c|}{0.77} \\ \cline{2-5} 
		\multicolumn{1}{|l|}{} &
		\multicolumn{1}{c|}{6 a 10 años} &
		\multicolumn{1}{c|}{738} &
		\multicolumn{1}{c|}{1} &
		\multicolumn{1}{c|}{0.14} \\ \cline{2-5} 
		\multicolumn{1}{|l|}{} &
		\multicolumn{1}{c|}{11 a 15 años} &
		\multicolumn{1}{c|}{1114} &
		\multicolumn{1}{c|}{0} &
		\multicolumn{1}{c|}{0} \\ \cline{2-5} 
		\multicolumn{1}{|l|}{\multirow{-4}{*}{2022}} &
		\multicolumn{1}{c|}{\cellcolor[HTML]{ECF4FF}Total} &
		\multicolumn{1}{c|}{\cellcolor[HTML]{ECF4FF}2503} &
		\multicolumn{1}{c|}{\cellcolor[HTML]{ECF4FF}6} &
		\multicolumn{1}{c|}{\cellcolor[HTML]{ECF4FF}0.24} \\ \hline
	\end{tabular}
			%		}
		%	
		%	
		%		{\footnotesize Fuente de datos: NOTICOVID, SISCOVID, SINADEF. Actualizado a la SE 12-2022.}
		%	\end{table}	
	\clearpage
	
	\subsection*{Análisis de la Mortalidad}
	
	\noindent En la Figura \ref{fig:mortalidad_edad} se muestra la mortalidad semanal para las edades agrupadas en decenios, se evidencia el ascenso de la curva de mortalidad hasta la SE 05 a expensas de las muertes reportadas en el grupo etario de mayores de 80 años, tras ello a partir de la SE 09 la tasa de mortalidad en todos los grupos etarios se muestra francamente en descenso alcanzando cifras menores a las reportadas en el año 2021. 
	
	\begin{figure}[h]
		\caption{Tasa de Mortalidad por COVID-19 por Grupo Etario hasta la SE 16-2022.}\label{fig:mortalidad_edad}
		\begin{center}
			\includegraphics[width=0.65\linewidth]{../figuras/mortalidad_edad_2021_2022.pdf}
		\end{center}
		{\footnotesize Fuente de datos: SINADEF} 
	\end{figure}
	
	
	La Figura \ref{fig:mortalidad_grupo_edad} muestra la relación entre la tasa de mortalidad y la vacunación en todos los grupos etarios. Las líneas de referencia rojas representan las fechas del inicio de la vacunación (linea roja) para el correspondiente grupo etario y la linea verde el inicio de la tercera ola pandémica. Se observa que desde la SE 09, la curva de mortalidad presenta una tendencia al descenso en todos los grupos etarioa, llegando a alcanzar cifras similares o menores a las reportadas antes de la tercera ola. 
	
	\begin{figure}[h]
		\caption{Tasa de Mortalidad por COVID-19 por Grupo Etario hasta la SE 16-2022.}
		\label{fig:mortalidad_grupo_edad}
		\centering
		\begin{subfigure}[b]{0.45\textwidth}
			\centering
			\includegraphics[width=\textwidth]{../figuras/mortalidad_edad_80.pdf}
			\caption{Más de 80 años}
			%\label{fig:}
		\end{subfigure}
		\hfill
		\begin{subfigure}[b]{0.45\textwidth}
			\centering
			\includegraphics[width=\textwidth]{../figuras/mortalidad_edad_70.pdf}
			\caption{70 a 79 años}
			%\label{fig:70 a 79 años}
		\end{subfigure}
		
		\vspace{10mm}
		\begin{subfigure}[b]{0.45\textwidth}
			\centering
			\includegraphics[width=\textwidth]{../figuras/mortalidad_edad_60.pdf}
			\caption{60 a 69 años}
			%\label{fig:60 a 69 años}
		\end{subfigure}
		\hfill
		\begin{subfigure}[b]{0.45\textwidth}
			\centering
			\includegraphics[width=\textwidth]{../figuras/mortalidad_edad_50.pdf}
			\caption{50 a 59 años}
			%\label{fig:50 a 59 años}
		\end{subfigure}
		
		\vspace{10mm}
		\begin{subfigure}[b]{0.45\textwidth}
			\centering
			\includegraphics[width=\textwidth]{../figuras/mortalidad_edad_40.pdf}
			\caption{40 a 49 años}
			%\label{fig:40 a 49 años}
		\end{subfigure}
		\hfill
		\begin{subfigure}[b]{0.45\textwidth}
			\centering
			\includegraphics[width=\textwidth]{../figuras/mortalidad_edad_30.pdf}
			\caption{30 a 39 años}
			%\label{fig:40 a 49 años}
		\end{subfigure}
	\end{figure}
	
	\begin{figure}[h]
		\caption{Tasa de Mortalidad por COVID-19 por Grupo Etario hasta la SE 16-2022.}
		\label{fig:mortalidad_grupo_edad_2}
		\centering
		\begin{subfigure}[b]{0.45\textwidth}
			\centering
			\includegraphics[width=\textwidth]{../figuras/mortalidad_edad_20.pdf}
			\caption{20 a 29 años}
			%\label{fig:40 a 49 años}
		\end{subfigure}
		
		\centering
		\begin{subfigure}[b]{0.45\textwidth}
			\centering
			\includegraphics[width=\textwidth]{../figuras/mortalidad_edad_10.pdf}
			\caption{10 a 19 años}
			%\label{fig:40 a 49 años}
		\end{subfigure}
		
		\vspace{10mm}
		\begin{subfigure}[b]{0.45\textwidth}
			\centering
			\includegraphics[width=\textwidth]{../figuras/mortalidad_edad_0.pdf}
			\caption{0 a 09 años}
			%\label{fig:40 a 49 años}
		\end{subfigure}
	\end{figure}
	\clearpage	
	\subsection*{Exceso de Muertes por Todas las Causas}
	\noindent La Figura \ref{fig:exceso_regional} muestra la tendencia del exceso de muertes por todas las causas con respecto al año 2019. Para la SE 17 se muestra un exceso de muertes de menos 05 muertes (exceso negativo), es decir que para esta semana se reporta una menor cantidad de muertes en comparación de la semana homónima en el año 2019. 
	
	\begin{figure}[h]
		\caption{Exceso de Fallecidos por Todas las Causas en la Región Cusco hasta la SE 16-2022.}\label{fig:exceso_regional}
		\begin{center}
			\includegraphics[width=0.85\linewidth]{../figuras/exceso_region_2022.pdf}
		\end{center}
		{\footnotesize {Fuente de datos: SISCOVID, NOTICOVID.}}
	\end{figure}
	\clearpage
	
	\subsection*{Cobertura de Vacunación por COVID-19 en la Región Cusco, hasta la SE 07-2022.}
	\noindent La Figura \ref{fig:vacuna_edad} muestra la cobertura de vacunación por grupo etario en la Región Cusco. El grupo con mejor cobertura de vacunación de 2 dosis fue el grupo etario de 60 a 79 años con una cobertura de 90,3 $\%$. Mientras que el grupo etario con menor cobertura de vacunación es el de 5 a 11 años con sólo 32,2 $\%$ de segundas dosis aplicadas, siendo el grupo que inició la vacunación más tardía para COVID-19.
	
	La Figura \ref{fig:Cobertura_Vacunacion_Provincias} muestra el avance de vacunación (2da y 3era dosis aplicadas) por provincia de la región Cusco. La provincia con mejor cobertura es la provincia de Cusco, seguida de la provincia de Anta, La Convención y Urubamba, mientras que las provincias con la menor cobertura son las provincias Paucartambo, Espinar y Chumbivilcas.
	\begin{figure}[h]
		\caption{Cobertura de Vacunación por Grupo Etario en la Región Cusco hasta la SE 16-2022. }\label{fig:vacuna_edad}
		\begin{center}
			\includegraphics[width=0.90\linewidth]{../figuras/vacunacion_grupo_edad.png}
		\end{center}
		{\footnotesize {Fuente de datos: SICOVAC, HIS-MINSA.}}
	\end{figure}
	\clearpage
	
	\begin{landscape}
		\begin{figure}[ht]
			\caption{Cobertura de segunda y tercera dosis aplicadas por provincia en la región Cusco-2022}\label{fig:Cobertura_Vacunacion_Provincias}
			\begin{center}
				\includegraphics[width=0.90\linewidth]{../sala_covid/../sala_nacional/Cobertura_Vacunacion_Provincias.jpg}
			\end{center}
			{\footnotesize {Fuente de datos: SICOVAC, HIS-MINSA.}}
		\end{figure}
	\end{landscape}
	
	
	%La Figura \ref{fig:cobertura_vacunaci_provincia}  muestra la cobertura de vacunación en cada una de las provincias de Cusco por grupo etario. Es preciso señalar que la provincia de Espinar tiene la cobertura más baja de la región, en los grupos etarios desde los 50 años en adelante.
	%
	%\begin{figure}[h]
	%	\caption{Cobertura de Vacunación por Provincia y por Grupo Etario en la Región Cusco, hasta la SE 51.}
	%	\label{fig:cobertura_vacunaci_provincia}
	%	\centering
	%	\begin{subfigure}[b]{0.45\textwidth}
		%		\centering
		%		\includegraphics[width=\textwidth]{../figuras/vacunacion_provincial_edad_1}
		%		\caption{ De 12 a 19 años}
		%		%\label{fig:}
		%	\end{subfigure}
	%	\hfill
	%	\begin{subfigure}[b]{0.45\textwidth}
		%		\centering
		%		\includegraphics[width=\textwidth]{../figuras/vacunacion_provincial_edad_2}
		%		\caption{De 20 a 29 años}
		%		%\label{fig:70 a 79 años}
		%	\end{subfigure}
	%	\begin{subfigure}[b]{0.45\textwidth}
		%		\centering
		%		\includegraphics[width=\textwidth]{../figuras/vacunacion_provincial_edad_3}
		%		\caption{De 30 a 39 años}
		%		%\label{fig:60 a 69 años}
		%	\end{subfigure}
	%	\hfill
	%	\begin{subfigure}[b]{0.45\textwidth}
		%		\centering
		%		\includegraphics[width=\textwidth]{../figuras/vacunacion_provincial_edad_4}
		%		\caption{De 40 a 49 años}
		%		%\label{fig:50 a 59 años}
		%	\end{subfigure}
	%	\begin{subfigure}[b]{0.45\textwidth}
		%		\centering
		%		\includegraphics[width=\textwidth]{../figuras/vacunacion_provincial_edad_5}
		%		\caption{De 50 a 59 años}
		%		%\label{fig:40 a 49 años}
		%	\end{subfigure}
	%	\hfill
	%	\begin{subfigure}[b]{0.45\textwidth}
		%		\centering
		%		\includegraphics[width=\textwidth]{../figuras/vacunacion_provincial_edad_6}
		%		\caption{De 60 a 69 años}
		%		%\label{fig:40 a 49 años}
		%	\end{subfigure}
	%\begin{subfigure}[b]{0.45\textwidth}
	%	\centering
	%	\includegraphics[width=\textwidth]{../figuras/vacunacion_provincial_edad_6}
	%	\caption{De 70 a 79 años}
	%	%\label{fig:40 a 49 años}
	%\end{subfigure}
	%\hfill
	%\begin{subfigure}[b]{0.45\textwidth}
	%	\centering
	%	\includegraphics[width=\textwidth]{../figuras/vacunacion_provincial_edad_7}
	%	\caption{Más de 80 años}
	%	%\label{fig:40 a 49 años}
	%\end{subfigure}
	%\end{figure}
	
	%\clearpage
	%\subsection*{Análisis de Supervivencia y Vacunas en Hospitalizados con COVID-19 de la Región Cusco}
	%\noindent Las curvas de sobrevida (Figura \ref{fig:supervivencia_2}) mostraron que los hospitalizados por COVID-19 con tres o dos dosis completas presentan menor probabilidad de muerte, a partir de su ingreso hasta el alta o defunción. Habiendo fallecido con tres dosis el 0.31$\%$, con dos dosis el 5.35$\%$ versus el 89.14$\%$ que no tuvo vacunación. 
	%Las curvas de sobrevida en hospitalizados por COVID-19 comenzaron a divergir en el 70$\%$ de eventos de muerte en día 18 después del ingreso (long rank test <0,0001).
	%
	%\begin{figure}[h]
	%	\caption{Defunciones en vacunados durante la hospitalización en la Región Cusco hasta la SE 07-2022.}\label{fig:supervivencia_2}
	%	\begin{center}
		%		\includegraphics[width=0.90\linewidth]{../figuras/supervivencia_1.png}
		%	\end{center}
	%	{\footnotesize {Fuente de datos: SICOVAC, Referencias y contrareferencias.}}
	%\end{figure}
	%
	%Las curvas de sobrevida (Figura \ref{fig:supervivencia_2}) mostraron que, conforme al grupo etario de un hospitalizado por COVID-19, se tuvo mayor sobrevida en los grupos de 0 a 17 años, siendo el grupo de mayores de 60 años el que tuvo menor sobrevida durante la hospitalización (long rank test <0,0001).
	%
	%\begin{figure}[h]
	%	\caption{Defunciones en vacunados durante la hospitalización conforme al grupo etario, Región Cusco hasta la SE 07-2022. }\label{fig:supervivencia_2}
	%	\begin{center}
		%		\includegraphics[width=0.90\linewidth]{../figuras/supervivencia_2.png}
		%	\end{center}
	%	{\footnotesize {Fuente de datos: SICOVAC, Referencias y contrareferencias}}
	%\end{figure}
	%
	%Para la evaluación de muerte relacionada con COVID-19 a partir de la última dosis de vacunación (Figura \ref{fig:supervivencia_4}), se verificó diferencias entre la aplicación de dosis incompleta (1 dosis) versus dosis completas (2 dosis), siendo muy escaso el número de hospitalizados con 3 dosis. Se observó divergencia en la presentación de muerte en el 88.5$\%$ alrededor del día 14 después de la última dosis de vacunación para las personas que presentaron dosis incompletas y completas. (long rank test <0.001)
	
	%\begin{figure}[h]
	%	\caption{Defunciones en vacunados durante la hospitalización a partir de la última dosis de vacunación, Región Cusco hasta la SE 07-2022. }\label{fig:supervivencia_4}
	%	\begin{center}
		%		\includegraphics[width=0.90\linewidth]{../figuras/supervivencia_4.png}
		%	\end{center}
	%	{\footnotesize {Fuente de datos: SICOVAC, Referencias y contrareferencias}}
	%\end{figure}
	
	
	\clearpage
	\subsection*{Ocupación de Camas}
	\noindent La disponibilidad y ocupación de camas UCI se ve resumida en la Figura \ref{fig:ocupacion_uci}, tras alcanzar el porcentaje máximo de ocupación del año en la SE 11 (93 $\%$), el porcentaje de ocupación se mantuvo en descenso, para la SE 16 el  porcentaje de ocupación fue del 63$\%$ siendo el menor reportado en el año.
	
	\begin{figure}[h]
		\caption{Ocupación de Camas UCI COVID-19 en la Región Cusco hasta la SE 16- 2022.}\label{fig:ocupacion_uci}
		\begin{center}
			\includegraphics[width=0.95\linewidth]{../figuras/uci.pdf}
		\end{center}
		{\footnotesize {Fuente de datos: REFERENCIAS Y CONTRAREFERENCIAS.}}
	\end{figure}
	\cleardoublepage
	
	En la Figura \ref{fig:ocupacion_3_nivel}, se plasma el porcentaje de ocupación y número de camas no-UCI COVID-19 en el III nivel Hospitalario. Desde la SE 09 se evidencia que el porcentaje de ocupación de camas presentó una tendencia al descenso manteniendose por debajo del 10 $\%$ de camas ocupadas en las últimas 4 semanas.  
	
	\begin{figure}[htpb]
		\caption{Ocupación de Camas no UCI COVID-19 en el nivel III en la Región Cusco hasta la SE 16-2022.}\label{fig:ocupacion_3_nivel}
		\begin{center}
			\includegraphics[width=0.95\linewidth]{../figuras/nivel_3.pdf}
		\end{center}
		{\footnotesize {Fuente de datos: REFERENCIAS Y CONTRAREFERENCIAS.}}
	\end{figure}
	
	\clearpage
	
	En la Figura \ref{fig:ocupacion_2nivel}, se observa el número de camas disponibles y su porcentaje de ocupación en el II Nivel. A partir de la SE 09 el porcentaje de ocupación de camas presenta una tendencia al descenso, manteniendo valores por debajo de 10 $\%$.
	
	\begin{figure}[h]
		\caption{Disponibilidad y Ocupación de Camas-COVID a Nivel de Hospitales del Nivel II en la Región Cusco hasta la SE 16-2022.}\label{fig:ocupacion_2nivel}
		\begin{center}
			\includegraphics[width=0.95\linewidth]{../figuras/nivel_2.pdf}
		\end{center}
		{\footnotesize {Fuente de datos: REFERENCIAS Y CONTRAREFERENCIAS.}}
	\end{figure}
	\clearpage
	\begin{landscape}
		
		\subsection*{Evaluación Provincial de Defunciones por COVID-19 para el año 2022.} 
		
		\begin{tabular}{lrrrrr}
	\rowcolor[HTML]{ECF4FF} 
	\textbf{Provincias}                   & \multicolumn{1}{l}{\cellcolor[HTML]{ECF4FF}\textbf{Población}} & \multicolumn{1}{l}{\cellcolor[HTML]{ECF4FF}\textbf{Pruebas Totales}} & \multicolumn{1}{l}{\cellcolor[HTML]{ECF4FF}\textbf{Defunciones}} & \multicolumn{1}{l}{\cellcolor[HTML]{ECF4FF}\textbf{Tasa de letalidad}} & \multicolumn{1}{l}{\cellcolor[HTML]{ECF4FF}\textbf{\begin{tabular}[c]{@{}l@{}}Tasa de mortalidad x \\   100.000 hab\end{tabular}}} \\
	\cellcolor[HTML]{FD6864}CANCHIS       & 105,049                                                        & 2,599                                                                & 19                                                             & 0.7\%                                                                  & 18.1                                                                                                                               \\
	\cellcolor[HTML]{FD6864}QUISPICANCHI  & 92,566                                                         & 1,192                                                                & 14                                                             & 1.2\%                                                                  & 15.1                                                                                                                               \\
	\cellcolor[HTML]{FFCE93}LA CONVENCIÓN & 185,793                                                        & 3,565                                                                & 22                                                             & 0.6\%                                                                  & 11.8                                                                                                                               \\
	\cellcolor[HTML]{FFCE93}CUSCO         & 463,656                                                        & 21,550                                                               & 50                                                             & 0.2\%                                                                  & 10.8                                                                                                                               \\
	\cellcolor[HTML]{FFFC9E}URUBAMBA      & 66,439                                                         & 1,230                                                                & 6                                                              & 0.5\%                                                                  & 9.0                                                                                                                                \\
	\cellcolor[HTML]{FFFC9E}PAUCARTAMBO   & 52,989                                                         & 465                                                                  & 4                                                              & 0.9\%                                                                  & 7.5                                                                                                                                \\
	\cellcolor[HTML]{FFFC9E}CHUMBIVILCAS  & 84,925                                                         & 905                                                                  & 6                                                              & 0.7\%                                                                  & 7.1                                                                                                                                \\
	\cellcolor[HTML]{FFFC9E}ANTA          & 57,731                                                         & 732                                                                  & 4                                                              & 0.5\%                                                                  & 6.9                                                                                                                                \\
	\cellcolor[HTML]{FFFC9E}ESPINAR       & 71,304                                                         & 937                                                                  & 4                                                              & 0.4\%                                                                  & 5.6                                                                                                                                \\
	\cellcolor[HTML]{FFFC9E}CALCA         & 76,462                                                         & 713                                                                  & 4                                                              & 0.6\%                                                                  & 5.2                                                                                                                                \\
	\cellcolor[HTML]{FFFC9E}CANAS         & 40,420                                                         & 488                                                                  & 2                                                              & 0.4\%                                                                  & 4.9                                                                                                                                \\
	\cellcolor[HTML]{9AFF99}ACOMAYO       & 28,477                                                         & 273                                                                  & 1                                                              & 0.4\%                                                                  & 3.5                                                                                                                                \\
	\cellcolor[HTML]{9AFF99}PARURO        & 31,264                                                         & 224                                                                  & 1                                                              & 0.4\%                                                                  & 3.2                                                                                                                                \\
	&                                                                &                                                                      &                                                                &                                                                        &                                                                                                                                    \\
	\rowcolor[HTML]{ECF4FF} 
	\textbf{Totales generales}            & \textbf{1,357,075}                                             & \textbf{34,873}                                                      & \textbf{137}                                                   & \textbf{0,39\%}                                                        & \textbf{10.1}                                                                                                                     
\end{tabular}
		
		
		{\footnotesize Fuente de datos: NOTICOVID, SISCOVID, SINADEF. Actualizado a la SE 16-2022.}
		
		\noindent 
		
	\end{landscape}
	%---------------------------------------------------------------------------
	% CAPÍTULO: EVALUACIÓN DE PROVINCIAS
	%---------------------------------------------------------------------------
	
	%insertar el cover del capitulo
	\includepdf[pages={1}]{../editorial/5.pdf}
	\clearpage
	
	\section*{Evaluación de Priorización y riesgo para COVID-19 por provincias}
	\addcontentsline{toc}{chapter}{Evaluación para Provincias Priorizadas}
	\noindent La Figura \ref{fig:incidencia_provincias} muestra las tasas de incidencia acumuladas por provincia desde el 1 de enero hasta el 03 de mayo del 2022, ordenadas de mayor a menor. Se evidencia que la mayor tasa de incidencia acumulada es para la provincia de Cusco (632,3 casos / 10 000 personas), seguida de la provincia de Canchis (288,8 casos/ 10 000 personas) y Urubamba (245,9 casos/ 10 000 personas).
	
	\begin{figure}[!htpb]
		\caption{Tasa de Incidencia Acumulada por Provincia en la Región Cusco, hasta el 03 de mayo del 2022*. }\label{fig:incidencia_provincias}
		\begin{center}
			\includegraphics[width=0.60\linewidth]{../figuras/incidencia_provincial_2022.png}
		\end{center}
		{\footnotesize {
				Fuente de datos: SISCOVID, NOTICOVID.(*)Se considera como caso positivo sólo a pacientes con prueba molecular o antigénica positiva}}
	\end{figure}
	
	La Figura \ref{fig:mortalidad_ordenada} muestra a las provincias de la región ordenadas de mayor a menor según la tasa de mortalidad acumulada, desde el 1 de enero hasta el 25 de abril del 2022. Se evidencia a partir de la SE 09 las tasas de mortalidad presentan un comportamiento en meceta, encontrandose las mayores cifras  en las provincias de Canchis y Paucartambo. 
	
	\begin{figure}[h]
		\caption{Tasa de Mortalidad Acumulada por Provincia en la Región Cusco, hasta la SE 16-2022. }\label{fig:mortalidad_ordenada}
		\begin{center}
			\includegraphics[width=0.60\linewidth]{../figuras/mortalidad_provincial_2022.png}
		\end{center}
		{\footnotesize {Fuente de datos: SISCOVID, NOTICOVID.}}
	\end{figure}
	
	La Figura \ref{fig:incidencia_provincial} muestra la tendencia de la incidencia acumulada a través del año 2022. Desde la SE 05 la incidencia acumulada muestra un comportamiento en meceta. 
	%
	\begin{figure}[h]
		\caption{Tendencia Provincial de Incidencia acumulada de COVID-19 hasta la SE 12-2022. }\label{fig:incidencia_provincial}
		\begin{center}
			\includegraphics[width=0.60\linewidth]{../figuras/incidencia_provincial_acumulada_2022.pdf}
		\end{center}
		{\footnotesize {Fuente de datos: SINADEF.}}
	\end{figure}
	
	\clearpage
	
	\section*{Evaluación Provincial de 5 Indicadores}
	\noindent El objetivo de estas figuras es comparar a cada provincia consigo misma de acuerdo a su historia  en la primera ola (en el año 2020). Se evaluaron los siguientes indicadores: incidencia (tomando en cuenta pruebas moleculares y antigénicas), tasa de mortalidad, tasa de positividad por prueba molecular, tasa de positividad por prueba antigénica, y exceso de defunciones para cada provincia.
	
	\subsection*{Provincia de Acomayo}
	\noindent La Figura \ref{fig:inc_mort_acomayo} se evidencia el descenso sostenido de la tasa de incidencia a partir de la SE 05. Con respecto a la tasa de mortalidad no se reportaron muertes desde la SE 09.
	\noindent La Figura \ref{fig:positividad_acomayo} muestra la tendencia al descenso de la tasa de positividad de ambas pruebas desde la SE 05.
	
	En la Figura \ref{fig:exceso_acomayo} se muestra que hay exceso de menos 4 defunciones respecto al año 2019.
	
	\begin{figure}[h]
		\caption{Tasa de Incidencia y Mortalidad Comparativa en la Provincia de Acomayo hasta la SE 16-2022.}\label{fig:inc_mort_acomayo}
		\begin{center}
			\includegraphics[width=0.70\linewidth]{../figuras/incidencia_mortalidad_20_21_1.png}
		\end{center}
		{\footnotesize {Fuente de datos: NOTICOVID, SISCOVID, SINADEF.}}
	\end{figure}
	
	\begin{figure}[h]
		\caption{Tasa de Positividad de Prueba Molecular y Antigénica Comparativa en la Provincia de Acomayo hasta la SE 16-2022. }\label{fig:positividad_acomayo}
		\begin{center}
			\includegraphics[width=0.7\linewidth]{../figuras/positividad_20_21_1.png}
		\end{center}
		{\footnotesize {Fuente de datos: NOTICOVID, SISCOVID.}}
	\end{figure}
	
	\begin{figure}[h]
		\caption{Exceso de Defunciones Comparativo en la Provincia de Acomayo hasta la SE 16-2022.}\label{fig:exceso_acomayo}
		\begin{center}
			\includegraphics[width=0.7\linewidth]{../figuras/exceso_1.pdf}
		\end{center}
		{\footnotesize {Fuente de datos: SINADEF.}}
	\end{figure}
	
	% Anta
	\clearpage
	
	\subsection*{Provincia de Anta}
	\noindent La Figura \ref{fig:inc_mort_anta} se observa la tendencia del descenso de la tasa de incidencia desde la SE 05, con respecto a las muertes no se reportaron muertes desde  SE 10. 
	\noindent La Figura
	\ref{fig:positividad_anta} muestra la marcada disminución desde la SE 05 de la tasa de positividad de pruebas moleculares y antigénicas. 
	
	En la Figura \ref{fig:exceso_anta} se muestra que hay exceso de menos 5 defunciones respecto al año 2019.
	
	\begin{figure}[h]
		\caption{Tasa de Incidencia y Mortalidad Comparativa en la Provincia de Anta hasta la SE 16-2022.}\label{fig:inc_mort_anta}
		\begin{center}
			\includegraphics[width=0.85\linewidth]{../figuras/incidencia_mortalidad_20_21_2.png}
		\end{center}
		{\footnotesize {Fuente de datos: NOTICOVID, SISCOVID, SINADEF.}}
	\end{figure}
	
	\begin{figure}[h]
		\caption{Tasa de Positividad de Prueba Molecular y Antigénica Comparativa en la Provincia de Anta hasta la SE 16-2022.}\label{fig:positividad_anta}
		\begin{center}
			\includegraphics[width=0.7\linewidth]{../figuras/positividad_20_21_2.png}
		\end{center}
		{\footnotesize {Fuente de datos: NOTICOVID, SISCOVID.}}
	\end{figure}
	
	\begin{figure}[h]
		\caption{Exceso de Defunciones Comparativo en la Provincia de Anta hasta la SE 16-2022.}\label{fig:exceso_anta}
		\begin{center}
			\includegraphics[width=0.7\linewidth]{../figuras/exceso_2.pdf}
		\end{center}
		{\footnotesize {Fuente de datos: SINADEF.}}
	\end{figure}
	
	% Canas
	\clearpage
	
	\subsection*{Provincia de Canas}
	\noindent La Figura \ref{fig:inc_mort_canas} se evidencia un descenso sostenido de la tasa de incidencia desde la SE 05, al igual que la tasa de positividad de pruebas antigénicas y moleculares (Figura \ref{fig:positividad_canas}). Con respecto a la tasa de mortalidad, no se reportaron muertes desde la SE 11. 
	
	La Figura \ref{fig:exceso_canas} muestra que hay exceso de 1 defunción respecto al año 2019.
	
	\begin{figure}[h]
		\caption{Tasa de Incidencia y Mortalidad Comparativa en la Provincia de Canas hasta la SE 16-2022.}\label{fig:inc_mort_canas}
		\begin{center}
			\includegraphics[width=0.85\linewidth]{../figuras/incidencia_mortalidad_20_21_3.png}
		\end{center}
		{\footnotesize {Fuente de datos: NOTICOVID, SISCOVID, SINADEF.}}
	\end{figure}
	
	\begin{figure}[h]
		\caption{Tasa de Positividad de Prueba Molecular y Antigénica Comparativa en la Provincia de Canas hasta la SE 16-2022.}\label{fig:positividad_canas}
		\begin{center}
			\includegraphics[width=0.7\linewidth]{../figuras/positividad_20_21_3.png}
		\end{center}
		{\footnotesize {Fuente de datos: NOTICOVID, SISCOVID.}}
	\end{figure}
	
	\begin{figure}[h]
		\caption{Exceso de Defunciones Comparativo en la Provincia de Canas hasta la SE 16-2022.}\label{fig:exceso_canas}
		\begin{center}
			\includegraphics[width=0.7\linewidth]{../figuras/exceso_3.pdf}
		\end{center}
		{\footnotesize {Fuente de datos: SINADEF.}}
	\end{figure}
	
	% Calca
	\clearpage
	
	\subsection*{Provincia de Calca}
	\noindent Las figuras de abajo (Figura \ref{fig:inc_mort_calca}, \ref{fig:positividad_calca}) muestran el comportamiento de la tasa de incidencia, mortalidad y  positividad. Con respecto a la tasa de incidencia se evidencia su descenso sostenido desde la SE 05 al igual que las tasas de positividad de ambas pruebas llegando alcanzar valores menores a los reportados antes de la tercera ola. Con respecto a la tasa de mortalidad, no se reportan muertes desde la SE 08. 
	
	En la Figura \ref{fig:exceso_calca} se muestra que hay exceso de menos 2 defunciones respecto al año 2019.
	
	\begin{figure}[h]
		\caption{Tasa de Incidencia y Mortalidad Comparativa en la Provincia de Calca hasta la SE 16-2022.}\label{fig:inc_mort_calca}
		\begin{center}
			\includegraphics[width=0.85\linewidth]{../figuras/incidencia_mortalidad_20_21_4.pdf}
		\end{center}
		{\footnotesize {Fuente de datos: NOTICOVID, SISCOVID, SINADEF.}}
	\end{figure}
	
	\begin{figure}[h]
		\caption{Tasa de Positividad de Prueba Molecular y Antigénica Comparativa en la Provincia de Calca hasta la SE 16-2022.}\label{fig:positividad_calca}
		\begin{center}
			\includegraphics[width=0.7\linewidth]{../figuras/positividad_20_21_4.pdf}
		\end{center}
		{\footnotesize {Fuente de datos: NOTICOVID, SISCOVID.}}
	\end{figure}
	
	\begin{figure}[h]
		\caption{Exceso de Defunciones Comparativo en la Provincia de Calca hasta la SE 16-2022.}\label{fig:exceso_calca}
		\begin{center}
			\includegraphics[width=0.7\linewidth]{../figuras/exceso_4.pdf}
		\end{center}
		{\footnotesize {Fuente de datos: SINADEF.}}
	\end{figure}
	
	% Canchis
	\clearpage
	
	\subsection*{Provincia de Canchis}
	\noindent La Figura \ref{fig:inc_mort_canchis} muestra la tendencia al descenso de la tasa de incidencia a partir de la SE 05. El reporte de muertes se ha mantenido variable en las últimas semanas aunque no se reportaron muertes desde la SE 12.    
	\noindent La Figura \ref{fig:positividad_canchis} muestra el descenso de las tasas de positividad de ambas pruebas desde la SE 05.
	
	En la Figura \ref{fig:exceso_canchis} se evidencia exceso de 2 defunciones con respecto al año 2019.
	
	\begin{figure}[h]
		\caption{Tasa de Incidencia y Mortalidad Comparativa en la Provincia de Canchis hasta la SE 16-2022.}\label{fig:inc_mort_canchis}
		\begin{center}
			\includegraphics[width=0.85\linewidth]{../figuras/incidencia_mortalidad_20_21_5.png}
		\end{center}
		{\footnotesize {Fuente de datos: NOTICOVID, SISCOVID, SINADEF.}}
	\end{figure}
	
	\begin{figure}[h]
		\caption{Tasa de Positividad de Prueba Molecular y Antigénica Comparativa en la Provincia de Canchis hasta la SE 16-2022.}\label{fig:positividad_canchis}
		\begin{center}
			\includegraphics[width=0.7\linewidth]{../figuras/positividad_20_21_5.png}
		\end{center}
		{\footnotesize {Fuente de datos: NOTICOVID, SISCOVID.}}
	\end{figure}
	
	\begin{figure}[h]
		\caption{Exceso de Defunciones Comparativo en la Provincia de Canchis hasta la SE 16-2022.}\label{fig:exceso_canchis}
		\begin{center}
			\includegraphics[width=0.7\linewidth]{../figuras/exceso_5.pdf}
		\end{center}
		{\footnotesize {Fuente de datos: SINADEF.}}
	\end{figure}
	
	\clearpage
	
	% Chumbivilcas
	\subsection*{Provincia de Chumbivilcas}
	\noindent En la Figura \ref{fig:inc_mort_chumbivilcas} se evidencia el descenso sostenido de la tasa de incidencia desde la SE 05. El reporte de muertes se ha mantenido variable en las últimas semanas.
	\noindent La Figura \ref{fig:positividad_chumbivilcas} muestra una tendencia al descenso de la tasa de positividad de pruebas antigénicas desde la SE 05, mientras que la positividad de pruebas moleculares ha presentado un discreto incremento desde la SE 13.
	
	En la Figura \ref{fig:exceso_chumbivilcas} se muestra que hay exceso de 1 defunción respecto al año 2019.
	
	\begin{figure}[h]
		\caption{Tasa de Incidencia y Mortalidad Comparativa en la Provincia de Chumbivilcas hasta la SE 16-2022.}\label{fig:inc_mort_chumbivilcas}
		\begin{center}
			\includegraphics[width=0.85\linewidth]{../figuras/incidencia_mortalidad_20_21_6.png}
		\end{center}
		{\footnotesize {Fuente de datos: NOTICOVID, SISCOVID, SINADEF.}}
	\end{figure}
	
	\begin{figure}[h]
		\caption{Tasa de Positividad de Prueba Molecular y Antigénica Comparativa en la Provincia de Chumbivilcas 2020 hasta la SE 16-2022.}\label{fig:positividad_chumbivilcas}
		\begin{center}
			\includegraphics[width=0.7\linewidth]{../figuras/positividad_20_21_6.png}
		\end{center}
		{\footnotesize {Fuente de datos: NOTICOVID, SISCOVID.}}
	\end{figure}
	
	\begin{figure}[h]
		\caption{Exceso de Defunciones Comparativo en la Provincia de Chumbivilcas hasta la SE 16-2022.}\label{fig:exceso_chumbivilcas}
		\begin{center}
			\includegraphics[width=0.7\linewidth]{../figuras/exceso_6.pdf}
		\end{center}
		{\footnotesize {Fuente de datos: SINADEF.}}
	\end{figure}
	
	% Cusco
	\clearpage
	
	\subsection*{Provincia de Cusco}
	\noindent En la Figura \ref{fig:inc_mort_cusco} se evidencia un descenso marcado de la tasa de incidencia desde la SE 05, mientras que el reporte de muertes fue disminuyendo paulatinamente.   
	\noindent La  Figura \ref{fig:positividad_cusco} muestra el mismo comportamiento para la tasa de positividad de ambas pruebas a partir de la SE 05 del 2022.
	
	En la Figura \ref{fig:exceso_cusco} se muestra que hay exceso de menos 10 defunciones respecto al año 2019.
	
	\begin{figure}[h]
		\caption{Tasa de Incidencia y Mortalidad Comparativa en la Provincia de Cusco hasta la SE 16-2022.}\label{fig:inc_mort_cusco}
		\begin{center}
			\includegraphics[width=0.85\linewidth]{../figuras/incidencia_mortalidad_20_21_7.png}
		\end{center}
		{\footnotesize {Fuente de datos: NOTICOVID, SISCOVID, SINADEF.}}
	\end{figure}
	
	\begin{figure}[h]
		\caption{Tasa de Positividad de Prueba Molecular y Antigénica Comparativa en la Provincia de Cusco hasta la SE 16-2022.}\label{fig:positividad_cusco}
		\begin{center}
			\includegraphics[width=0.7\linewidth]{../figuras/positividad_20_21_7.png}
		\end{center}
		{\footnotesize {Fuente de datos: NOTICOVID, SISCOVID.}}
	\end{figure}
	
	\begin{figure}[h]
		\caption{Exceso de Defunciones Comparativo en la Provincia de Cusco hasta la SE 16-2022.}\label{fig:exceso_cusco}
		\begin{center}
			\includegraphics[width=0.7\linewidth]{../figuras/exceso_7.pdf}
		\end{center}
		{\footnotesize {Fuente de datos: SINADEF.}}
	\end{figure}
	
	% Espinar
	\clearpage
	
	\subsection*{Provincia de Espinar}
	\noindent Las figuras de abajo (Figura \ref{fig:inc_mort_espinar}, \ref{fig:positividad_espinar}) muestran el comportamiento de la tasa de incidencia, mortalidad y positividad. Se evidencia la tendencia al descenso de la tasa de incidencia desde la SE 05, mientras que con respecto a la tasa de mortalidad, no se reportaron muertes desde la SE 09. 
	
	En la Figura \ref{fig:exceso_espinar} se muestra que hay exceso de menos 2 defunciones respecto al año 2019.
	
	\begin{figure}[h]
		\caption{Tasa de Incidencia y Mortalidad Comparativa en la Provincia de Espinar hasta la SE 16-2022.}\label{fig:inc_mort_espinar}
		\begin{center}
			\includegraphics[width=0.85\linewidth]{../figuras/incidencia_mortalidad_20_21_8.png}
		\end{center}
		{\footnotesize {Fuente de datos: NOTICOVID, SISCOVID, SINADEF.}}
	\end{figure}
	
	\begin{figure}[h]
		\caption{Tasa de Positividad de Prueba Molecular y Antigénica Comparativa en la Provincia de Espinar hasta la SE 16-2022.}\label{fig:positividad_espinar}
		\begin{center}
			\includegraphics[width=0.7\linewidth]{../figuras/positividad_20_21_8.png}
		\end{center}
		{\footnotesize {Fuente de datos: NOTICOVID, SISCOVID.}}
	\end{figure}
	
	\begin{figure}[h]
		\caption{Exceso de Defunciones Comparativo en la Provincia de Espinar hasta la SE 16-2022.}\label{fig:exceso_espinar}
		\begin{center}
			\includegraphics[width=0.7\linewidth]{../figuras/exceso_8.pdf}
		\end{center}
		{\footnotesize {Fuente de datos: SINADEF.}}
	\end{figure}
	
	% La Convención
	\clearpage
	
	\subsection*{Provincia de La Convención}
	\noindent Las figuras inferiores (Figura \ref{fig:inc_mort_laconv}, \ref{fig:positividad_laconv}) muestran el comportamiento de la tasa de incidencia y mortalidad, con respecto a la tasa de incidencia se muestra una tendencia al descenso desde la SE 05. Tras reportarse muertes en los primeros dos meses del año, la tasa de mortalidad presenta una pendiente en descenso desde la SE 11. 
	
	En la Figura \ref{fig:exceso_laconv}  muestra que hay exceso de 5 defunciones respecto al año 2019.     
	
	\begin{figure}[h]
		\caption{Tasa de Incidencia y Mortalidad Comparativa en la Provincia de La Convención hasta la SE 16-2022.}\label{fig:inc_mort_laconv}
		\begin{center}
			\includegraphics[width=0.85\linewidth]{../figuras/incidencia_mortalidad_20_21_9.png}
		\end{center}
		{\footnotesize {Fuente de datos: NOTICOVID, SISCOVID, SINADEF.}}
	\end{figure}
	
	\begin{figure}[h]
		\caption{Tasa de Positividad de Prueba Molecular y Antigénica Comparativa en la Provincia de La Convención hasta la SE 16-2022.}\label{fig:positividad_laconv}
		\begin{center}
			\includegraphics[width=0.7\linewidth]{../figuras/positividad_20_21_9.png}
		\end{center}
		{\footnotesize {Fuente de datos: NOTICOVID, SISCOVID.}}
	\end{figure}
	
	\begin{figure}[h]
		\caption{Exceso de Defunciones Comparativo en la Provincia de La Convención hasta la SE 16-2022.}\label{fig:exceso_laconv}
		\begin{center}
			\includegraphics[width=0.7\linewidth]{../figuras/exceso_9.pdf}
		\end{center}
		{\footnotesize {Fuente de datos: SINADEF.}}
	\end{figure}
	
	% Paruro
	\clearpage
	
	\subsection*{Provincia de Paruro}
	\noindent Las figuras de abajo (Figura \ref{fig:inc_mort_paruro}, \ref{fig:positividad_paruro}) muestran el comportamiento de la tasa de incidencia, mortalidad y positividad. Con respecto a la tasa de incidencia se observa su descenso a partir de la SE 07, mientras que con respecto a la tasa de mortalidad no se reportaron muertes desde la SE 10.
	
	En la Figura \ref{fig:exceso_paruro} muestra que hubo un exceso de 2 muertes con respecto al año 2019.
	
	\begin{figure}[h]
		\caption{Tasa de Incidencia y Mortalidad Comparativa en la Provincia de Paruro, hasta la SE 16-2022.}\label{fig:inc_mort_paruro}
		\begin{center}
			\includegraphics[width=0.85\linewidth]{../figuras/incidencia_mortalidad_20_21_10.png}
		\end{center}
		{\footnotesize {Fuente de datos: NOTICOVID, SISCOVID, SINADEF.}} 
	\end{figure}
	
	\begin{figure}[h]
		\caption{Tasa de Positividad de Prueba Molecular y Antigénica Comparativa en la Provincia de Paruro hasta la SE 16-2022.}\label{fig:positividad_paruro}
		\begin{center}
			\includegraphics[width=0.7\linewidth]{../figuras/positividad_20_21_10.png}
		\end{center}
		{\footnotesize {Fuente de datos: NOTICOVID, SISCOVID.}}
	\end{figure}
	
	\begin{figure}[h]
		\caption{Exceso de Defunciones Comparativo en la Provincia de Paruro hasta la SE 16-2022.}\label{fig:exceso_paruro}
		\begin{center}
			\includegraphics[width=0.7\linewidth]{../figuras/exceso_10.pdf}
		\end{center}
		{\footnotesize {Fuente de datos: SINADEF.}}
	\end{figure}
	
	
	% Paucartambo
	\clearpage
	
	\subsection*{Provincia de Paucartambo}
	\noindent Las figuras de abajo (Figura \ref{fig:inc_mort_paucartam}, \ref{fig:positividad_paucartam}) muestran el comportamiento de la tasa de incidencia, mortalidad y positividad. Se evidencia la tendencia al descenso de la tasa de incidencia desde la SE 05, mientras quela tasa de mortalidad se ha mantenido variable.  
	En la Figura \ref{fig:exceso_paucartam} se muestra que hay exceso de menos 3 defunciones respecto al año 2021.  
	
	\begin{figure}[h]
		\caption{Tasa de Incidencia y Mortalidad Comparativa en la Provincia de Paucartambo hasta la SE 16-2022.}\label{fig:inc_mort_paucartam}
		\begin{center}
			\includegraphics[width=0.85\linewidth]{../figuras/incidencia_mortalidad_20_21_11.png}
		\end{center}
		{\footnotesize {Fuente de datos: NOTICOVID, SISCOVID, SINADEF.}}
	\end{figure}
	
	\begin{figure}[h]
		\caption{Tasa de Positividad de Prueba Molecular y Antigénica Comparativa en la Provincia de Paucartambo hasta la SE 16-2022.}\label{fig:positividad_paucartam}
		\begin{center}
			\includegraphics[width=0.7\linewidth]{../figuras/positividad_20_21_11.png}
		\end{center}
		{\footnotesize {Fuente de datos: NOTICOVID, SISCOVID.}}
	\end{figure}
	
	\begin{figure}[h]
		\caption{Exceso de Defunciones Comparativo en la Provincia de Paucartambo hasta la SE 16-2022.}\label{fig:exceso_paucartam}
		\begin{center}
			\includegraphics[width=0.7\linewidth]{../figuras/exceso_11.pdf}
		\end{center}
		{\footnotesize {Fuente de datos: SINADEF.}}
	\end{figure}
	
	% Quispicanchi
	\clearpage
	
	\subsection*{Provincia de Quispicanchi}
	\noindent Las figuras de abajo (Figura \ref{fig:inc_mort_quisp}, \ref{fig:positividad_quisp}) muestran el comportamiento de la tasa de incidencia, mortalidad y positividad. La tasa de incidencia muestra una tendencia al descenso sostenida desde la SE 05, mientras que la tasa de mortalidad se ha mantenido variable en las últimas semanas.      
	
	En la Figura \ref{fig:exceso_quisp} se muestra que no hay exceso de defunciones respecto al año 2019.
	
	\begin{figure}[h]
		\caption{Tasa de Incidencia y Mortalidad Comparativa en la Provincia de Quispicanchi hasta la SE 16-2022.}\label{fig:inc_mort_quisp}
		\begin{center}
			\includegraphics[width=0.85\linewidth]{../figuras/incidencia_mortalidad_20_21_12.png}
		\end{center}
		{\footnotesize {Fuente de datos: NOTICOVID, SISCOVID, SINADEF.}}
	\end{figure}
	
	\begin{figure}[h]
		\caption{Tasa de Positividad de Prueba Molecular y Antigénica Comparativa en la Provincia de Quispicanchi hasta la SE 16-2022.}\label{fig:positividad_quisp}
		\begin{center}
			\includegraphics[width=0.7\linewidth]{../figuras/positividad_20_21_12.png}
		\end{center}
		{\footnotesize {Fuente de datos: NOTICOVID, SISCOVID.}}
	\end{figure}
	
	\begin{figure}[h]
		\caption{Exceso de Defunciones Comparativo en la Provincia de Quispicanchis hasta la SE 16-2022.}\label{fig:exceso_quisp}
		\begin{center}
			\includegraphics[width=0.7\linewidth]{../figuras/exceso_12.pdf}
		\end{center}
		{\footnotesize {Fuente de datos: SINADEF.}}
	\end{figure}
	
	% Urubamba
	\clearpage
	
	\subsection*{Provincia de Urubamba}
	\noindent Las figuras de abajo (Figura \ref{fig:inc_urub}, \ref{fig:positividad_urub}) muestran el comportamiento de la tasa de incidencia, mortalidad y positividad. Con respecto a la tasa de incidencia se evidencia un descenso de la misma desde la SE 05. Y en cuanto a la tasa de mortaliad no se reportan muertes desde la SE 08.
	
	En la Figura \ref{fig:exceso_urub} se muestra que no hay exceso de defunciones respecto al año 2019.
	
	\begin{figure}[h]
		\caption{Tasa de Incidencia y Mortalidad Comparativa en la Provincia de Urubamba hasta la SE 16-2022.}\label{fig:inc_urub}
		\begin{center}
			\includegraphics[width=0.85\linewidth]{../figuras/incidencia_mortalidad_20_21_13.png}
		\end{center}
		{\footnotesize {Fuente de datos: NOTICOVID, SISCOVID, SINADEF.}}
	\end{figure}
	
	\begin{figure}[h]
		\caption{Tasa de Positividad de Prueba Molecular y Antigénica Comparativa en la Provincia de Urubamba hasta la SE 16-2022.}\label{fig:positividad_urub}
		\begin{center}
			\includegraphics[width=0.7\linewidth]{../figuras/positividad_20_21_13.png}
		\end{center}
		{\footnotesize {Fuente de datos: NOTICOVID, SISCOVID.}}
	\end{figure}
	
	\begin{figure}[h]
		\caption{Exceso de Defunciones Comparativo en la Provincia de Urubamba hasta la SE 16-2022.}\label{fig:exceso_urub}
		\begin{center}
			\includegraphics[width=0.7\linewidth]{../figuras/exceso_13.pdf}
		\end{center}
		{\footnotesize {Fuente de datos: SINADEF.}}
	\end{figure}
	
	\clearpage
	%---------------------------------------------------------------------------
	% CAPÍTULO: VARIANTES DE COVID-19
	%---------------------------------------------------------------------------
	%insertar el cover del capitulo
	\includepdf[pages={1}]{../editorial/6.pdf}
	\clearpage
	
	\section* {Variantes de COVID-19 en la Región Cusco}
	\addcontentsline{toc}{chapter}{Variantes de COVID-19}
	\noindent La aparición de la variante ómicron ha generado la tercera ola de COVID-19 en el Perú debido a su gran transmisibilidad. Asimismo para la SE 12 se ha identificado la primera subvariante BA.2 de ómicron. En la Figura \ref{fig:variantes} se observa que en la región de Cusco, la variante ómicron (100$\%$) persiste como la más prevalente en la región desplazando a las demás variantes.
	Hasta el 25 de abril del 2022 se secuenciaron 854 muestras a nivel de la región de Cusco	encontrándose las variantes beta (B.1.1.348), gamma (P.1, P.1.7), lambda (C.37), delta (B.1617.2), mu y ómicron (BA.1.1 y BA.2). 
	La vigilancia genómica es realizada en colaboración con 4 instituciones externas a GERESA-Cusco.
	
	\begin{figure}[h]
		\caption{Prevalencia de las variantes de SARS Cov-2 aisladas en la región de Cusco, hasta Abril-2022. }\label{fig:variantes}
		\begin{center}
			\includegraphics[width=0.85\linewidth]{../figuras/variantes.pdf}
		\end{center}
		{\footnotesize {Fuente de datos: INS-NETLAB, UPCH, UNSAAC}}
	\end{figure}
	
	Asimismo, la Figura \ref{fig:mapa_variantes} muestra las variantes de COVID-19 aisladas por provincias. Se evidencia la amplia distribución de la variante Ómicron en la región, reportándose casos por esta variante en 9 de las 13 provincias. 
	
	\begin{figure}[h]
		\caption{Distribución provincial de las variantes de SARS-CoV-2 aisladas en la Región Cusco hasta la SE 16-2022.}
		\label{fig:mapa_variantes}
		\centering
		\begin{subfigure}[b]{0.40\textwidth}
			\centering
			\includegraphics[width=\textwidth]{../figuras/variantes_provincial_lambda.pdf}
			\caption{Variante Lambda}
			%\label{fig:}
		\end{subfigure}
		\hfill
		\begin{subfigure}[b]{0.40\textwidth}
			\centering
			\includegraphics[width=\textwidth]{../figuras/variantes_provincial_gamma.pdf}
			\caption{Variante Gamma}
			%\label{fig:70 a 79 años}
		\end{subfigure}
		
		\begin{subfigure}[b]{0.40\textwidth}
			\centering
			\includegraphics[width=\textwidth]{../figuras/variantes_provincial_delta.pdf}
			\caption{Variante Delta}
			%\label{fig:60 a 69 años}
		\end{subfigure}
		\vspace{0.5mm}
		\hspace{25mm}
		\begin{subfigure}[b]{0.40\textwidth}
			\centering
			\includegraphics[width=\textwidth]{../figuras/variantes_provincial_omicron.png}
			\caption{Variante Ómicron}
			%\label{fig:60 a 69 años}
		\end{subfigure}
	\end{figure}
	
	\clearpage
	%---------------------------------------------------------------------------
	% CAPÍTULO: DEFUNCIONES CERO
	%-------------------------------------------
	
	%insertar el cover del capitulo
	
	\includepdf[pages={1}]{../editorial/7.pdf}
	\clearpage
	\section*{Semanas con Cero Defunciones por COVID-19 por Semana a Nivel Provincial}\addcontentsline{toc}{chapter}{Defunciones Cero}
	
	\noindent En la tabla inferior se muestra las provincias con cero defunciones reportadas (casillas en amarillo) por cada semana epidemiológica.Tras en incremento de muertes en los dos primeros meses del 2022 se evidencia una tendencia al descenso de las muertes reportadas que inicia desde la SE 12. Para la SE 16 sólo la provincia de Paruro reportó una muerte por COVID-19 en su territorio. Mientras que las provincias de Acomayo, Canas, Espinar y Urubamba no reportaron muertes desde la SE 09.   
	
	\begin{table}[h]		\caption{Defunciones Cero por COVID-19 a nivel Provincial hasta la SE 16-2022.}
		\resizebox{\textwidth}{!}{%
			\begin{tabular}{lccccccccc}
	\textbf{}              	  & \multicolumn{1}{l}{}                        & \multicolumn{1}{l}{}      & \multicolumn{1}{l}{}                         & \multicolumn{1}{l}{}                         & \multicolumn{1}{l}{}                         & \multicolumn{1}{l}{}                        & \multicolumn{1}{l}{}                         & \multicolumn{1}{l}{}                         & \multicolumn{1}{l}{}     \\
	\textbf{}                                                                               
	&\textbf{SE-03}
	&\textbf{SE-04}								&\textbf{SE-05}	
	&\textbf{SE-06}								&\textbf{SE-07}				&\textbf{SE-08}
	&\textbf{SE-09}								&\textbf{SE-10}
	&\textbf{SE-11}\\
	\textbf{}              	  	
	&\textbf{16ene-22ene}						&\textbf{23ene-29ene}						&\textbf{30ene-05feb}
	&\textbf{05feb-12feb}						&\textbf{13feb-19feb}
	&\textbf{20feb-26feb}						&\textbf{27feb-05mar}
	&\textbf{06mar-12mar}						&\textbf{13mar-19mar}\\
	\textbf{Acomayo}                        	
	&\cellcolor[HTML]{FCC46C}				    &\cellcolor[HTML]{FCC46C}
	&\cellcolor[HTML]{FCC46C}					&\cellcolor[HTML]{FCC46C}
	&\cellcolor[HTML]{FCC46C}					&1
	&\cellcolor[HTML]{FCC46C}					&\cellcolor[HTML]{FCC46C} 
	&\cellcolor[HTML]{FCC46C}\\
	\textbf{Anta}                                                          				
	&\cellcolor[HTML]{FCC46C}					&2 				
	&1											&\cellcolor[HTML]{FCC46C}					&2
	&\cellcolor[HTML]{FCC46C}					&\cellcolor[HTML]{FCC46C}					&1
	&\cellcolor[HTML]{FCC46C}\\
	\textbf{Calca}      				       								            &\cellcolor[HTML]{FCC46C}					&\cellcolor[HTML]{FCC46C}
	&1 											&1	
	&\cellcolor[HTML]{FCC46C}					&1											&\cellcolor[HTML]{FCC46C} 				&1											&1\\             			
	\textbf{Canas}                              		
	&\cellcolor[HTML]{FCC46C}					&1
	&1											&\cellcolor[HTML]{FCC46C}
	&1											&\cellcolor[HTML]{FCC46C}
	&\cellcolor[HTML]{FCC46C}					&\cellcolor[HTML]{FCC46C}
	&\cellcolor[HTML]{FCC46C} \\
	\textbf{Canchis}                             		
	&4											&3
	&2											&4
	&1											&1
	&\cellcolor[HTML]{FCC46C}					&3
	&1\\
	\textbf{Chumbivilcas}                      			
	&\cellcolor[HTML]{FCC46C} 					&1
	&\cellcolor[HTML]{FCC46C}					&3
	&4											&\cellcolor[HTML]{FCC46C}
	&\cellcolor[HTML]{FCC46C}					&\cellcolor[HTML]{FCC46C}
	&1\\
	\textbf{Cusco}                            										
	&11											&9 	
	&14 										&4
	&8											&3
	&1											&1
	&1\\
	\textbf{Espinar}       					             								
	 &\cellcolor[HTML]{FCC46C}
	&1											&1
	&\cellcolor[HTML]{FCC46C}					&1
	&2											&\cellcolor[HTML]{FCC46C}	
	&\cellcolor[HTML]{FCC46C} 					&\cellcolor[HTML]{FCC46C}\\
	\textbf{La Convención}                      					
	&3
	&4											&5
	&2											&3
	&1 											&1 
	&1											&\cellcolor[HTML]{FCC46C}\\
	\textbf{Paruro}                            					
	&1
	&\cellcolor[HTML]{FCC46C}					&\cellcolor[HTML]{FCC46C}
	&\cellcolor[HTML]{FCC46C} 					&1
	&1											&1
	&\cellcolor[HTML]{FCC46C}					&\cellcolor[HTML]{FCC46C}\\
	\textbf{Paucartambo}               		                       					
	&1											&1		
	&1											&\cellcolor[HTML]{FCC46C}
	&\cellcolor[HTML]{FCC46C}
	&\cellcolor[HTML]{FCC46C}					&\cellcolor[HTML]{FCC46C}
	&1											&\cellcolor[HTML]{FCC46C}\\
	\textbf{Quispicanchi}                                         	                 	
	&3											&5
	&4											&\cellcolor[HTML]{FCC46C}
	&1											&\cellcolor[HTML]{FCC46C}
	&1											&\cellcolor[HTML]{FCC46C}
	&\cellcolor[HTML]{FCC46C}\\
	\textbf{Urubamba}                                                          			
	&1
	&\cellcolor[HTML]{FCC46C}					&3
	&\cellcolor[HTML]{FCC46C}					&1
	&2											&\cellcolor[HTML]{FCC46C}
	&\cellcolor[HTML]{FCC46C}					&\cellcolor[HTML]{FCC46C}\\	
	&\multicolumn{1}{l}{}                       &\multicolumn{1}{l}{}            &\multicolumn{1}{l}{}                         
	&\multicolumn{1}{l}{}                       &\multicolumn{1}{l}{}            &\multicolumn{1}{l}{}                       &\multicolumn{1}{l}{}                       &\multicolumn{1}{l}{}            &\multicolumn{1}{l}{}    
\end{tabular}
		}
		{\footnotesize {Fuente de datos: SINADEF.}}
	\end{table}
	\pagebreak
	\section*{Resumen de Indicadores COVID-19}\addcontentsline{toc}{chapter}{Resumen de Indicadores Covid19}
	\begin{table}[h]		\caption{Tabla de Letalidad, Mortalidad e Incidenciaa nivel Regional por Covid19, 2020 - 2022 (SE 17)}
		\resizebox{\textwidth}{!}{%
				\begin{tabular}{lccc|cccccc|}
		\cline{5-10}
		&
		\multicolumn{1}{l}{} &
		&
		&
		\multicolumn{6}{c|}{\cellcolor[HTML]{F2F2F2}} \\
		&
		\multicolumn{1}{l}{} &
		\multicolumn{1}{l}{} &
		\multicolumn{1}{l|}{} &
		\multicolumn{6}{c|}{\multirow{-2}{*}{\cellcolor[HTML]{F2F2F2}\textbf{Etapa de Vida}}} \\ \cline{5-10} 
		&
		\multicolumn{1}{l}{} &
		\multicolumn{1}{l}{} &
		\multicolumn{1}{l|}{} &
		\multicolumn{1}{c|}{\cellcolor[HTML]{F2F2F2}\textbf{Niño}} &
		\multicolumn{1}{l|}{\cellcolor[HTML]{F2F2F2}\textbf{Adolescente}} &
		\multicolumn{1}{l|}{\cellcolor[HTML]{F2F2F2}\textbf{Joven}} &
		\multicolumn{1}{l|}{\cellcolor[HTML]{F2F2F2}\textbf{Adulto}} &
		\multicolumn{1}{l|}{\cellcolor[HTML]{F2F2F2}\textbf{Adulto Mayor}} &
		\cellcolor[HTML]{F2F2F2}\textbf{Total} \\ \cline{2-10} 
		\multicolumn{1}{l|}{} &
		\multicolumn{1}{c|}{\cellcolor[HTML]{ECF4FF}} &
		\multicolumn{1}{c|}{\cellcolor[HTML]{ECF4FF}} &
		\cellcolor[HTML]{ECF4FF}\textbf{Tasa (\%)} &
		\multicolumn{1}{c|}{\cellcolor[HTML]{ECF4FF}0.4} &
		\multicolumn{1}{c|}{\cellcolor[HTML]{ECF4FF}0.049} &
		\multicolumn{1}{c|}{\cellcolor[HTML]{ECF4FF}0.12} &
		\multicolumn{1}{c|}{\cellcolor[HTML]{ECF4FF}0.57} &
		\multicolumn{1}{c|}{\cellcolor[HTML]{ECF4FF}7.9} &
		\cellcolor[HTML]{ECF4FF}1.3 \\ \cline{4-10} 
		\multicolumn{1}{l|}{} &
		\multicolumn{1}{c|}{\cellcolor[HTML]{ECF4FF}} &
		\multicolumn{1}{c|}{\multirow{-2}{*}{\cellcolor[HTML]{ECF4FF}\textbf{Letalidad}}} &
		\cellcolor[HTML]{ECF4FF} &
		\multicolumn{1}{c|}{\cellcolor[HTML]{ECF4FF}} &
		\multicolumn{1}{c|}{\cellcolor[HTML]{ECF4FF}} &
		\multicolumn{1}{c|}{\cellcolor[HTML]{ECF4FF}} &
		\multicolumn{1}{c|}{\cellcolor[HTML]{ECF4FF}} &
		\multicolumn{1}{c|}{\cellcolor[HTML]{ECF4FF}} &
		\cellcolor[HTML]{ECF4FF} \\ \cline{3-3}
		\multicolumn{1}{l|}{} &
		\multicolumn{1}{c|}{\cellcolor[HTML]{ECF4FF}} &
		\multicolumn{1}{c|}{\cellcolor[HTML]{ECF4FF}} &
		\multirow{-2}{*}{\cellcolor[HTML]{ECF4FF}\textbf{Defunciones}} &
		\multicolumn{1}{c|}{\multirow{-2}{*}{\cellcolor[HTML]{ECF4FF}07}} &
		\multicolumn{1}{c|}{\multirow{-2}{*}{\cellcolor[HTML]{ECF4FF}01}} &
		\multicolumn{1}{c|}{\multirow{-2}{*}{\cellcolor[HTML]{ECF4FF}29}} &
		\multicolumn{1}{c|}{\multirow{-2}{*}{\cellcolor[HTML]{ECF4FF}375}} &
		\multicolumn{1}{c|}{\multirow{-2}{*}{\cellcolor[HTML]{ECF4FF}973}} &
		\multirow{-2}{*}{\cellcolor[HTML]{ECF4FF}1385} \\ \cline{4-10} 
		\multicolumn{1}{l|}{} &
		\multicolumn{1}{c|}{\cellcolor[HTML]{ECF4FF}} &
		\multicolumn{1}{c|}{\multirow{-2}{*}{\cellcolor[HTML]{ECF4FF}\textbf{Mortalidad}}} &
		\cellcolor[HTML]{ECF4FF}\textbf{Tasa*} &
		\multicolumn{1}{c|}{\cellcolor[HTML]{ECF4FF}5.2} &
		\multicolumn{1}{c|}{\cellcolor[HTML]{ECF4FF}0.74} &
		\multicolumn{1}{c|}{\cellcolor[HTML]{ECF4FF}21} &
		\multicolumn{1}{c|}{\cellcolor[HTML]{ECF4FF}276} &
		\multicolumn{1}{c|}{\cellcolor[HTML]{ECF4FF}717} &
		\cellcolor[HTML]{ECF4FF}1020 \\ \cline{3-10} 
		\multicolumn{1}{l|}{} &
		\multicolumn{1}{c|}{\cellcolor[HTML]{ECF4FF}} &
		\multicolumn{1}{c|}{\cellcolor[HTML]{ECF4FF}} &
		\cellcolor[HTML]{ECF4FF}\textbf{Casos +} &
		\multicolumn{1}{c|}{\cellcolor[HTML]{ECF4FF}1749} &
		\multicolumn{1}{c|}{\cellcolor[HTML]{ECF4FF}2029} &
		\multicolumn{1}{c|}{\cellcolor[HTML]{ECF4FF}25091} &
		\multicolumn{1}{c|}{\cellcolor[HTML]{ECF4FF}66024} &
		\multicolumn{1}{c|}{\cellcolor[HTML]{ECF4FF}12255} &
		\cellcolor[HTML]{ECF4FF}107148 \\ \cline{4-10} 
		\multicolumn{1}{l|}{} &
		\multicolumn{1}{c|}{\multirow{-6}{*}{\cellcolor[HTML]{ECF4FF}\textbf{2020}}} &
		\multicolumn{1}{c|}{\multirow{-2}{*}{\cellcolor[HTML]{ECF4FF}\textbf{Incidencia}}} &
		\cellcolor[HTML]{ECF4FF}\textbf{Tasa*} &
		\multicolumn{1}{c|}{\cellcolor[HTML]{ECF4FF}1288} &
		\multicolumn{1}{c|}{\cellcolor[HTML]{ECF4FF}1495} &
		\multicolumn{1}{c|}{\cellcolor[HTML]{ECF4FF}18483} &
		\multicolumn{1}{c|}{\cellcolor[HTML]{ECF4FF}48637} &
		\multicolumn{1}{c|}{\cellcolor[HTML]{ECF4FF}9028} &
		\cellcolor[HTML]{ECF4FF}7.9 \\ \cline{2-10} 
		\multicolumn{1}{l|}{} &
		\multicolumn{1}{c|}{\cellcolor[HTML]{FFFFC7}} &
		\multicolumn{1}{c|}{\cellcolor[HTML]{FFFFC7}} &
		\cellcolor[HTML]{FFFFC7}\textbf{Tasa (\%)} &
		\multicolumn{1}{c|}{\cellcolor[HTML]{FFFFC7}0.94} &
		\multicolumn{1}{c|}{\cellcolor[HTML]{FFFFC7}0.087} &
		\multicolumn{1}{c|}{\cellcolor[HTML]{FFFFC7}0.13} &
		\multicolumn{1}{c|}{\cellcolor[HTML]{FFFFC7}1.9} &
		\multicolumn{1}{c|}{\cellcolor[HTML]{FFFFC7}19} &
		\cellcolor[HTML]{FFFFC7}3.8 \\ \cline{4-10} 
		\multicolumn{1}{l|}{} &
		\multicolumn{1}{c|}{\cellcolor[HTML]{FFFFC7}} &
		\multicolumn{1}{c|}{\multirow{-2}{*}{\cellcolor[HTML]{FFFFC7}\textbf{Letalidad}}} &
		\cellcolor[HTML]{FFFFC7} &
		\multicolumn{1}{c|}{\cellcolor[HTML]{FFFFC7}} &
		\multicolumn{1}{c|}{\cellcolor[HTML]{FFFFC7}} &
		\multicolumn{1}{c|}{\cellcolor[HTML]{FFFFC7}} &
		\multicolumn{1}{c|}{\cellcolor[HTML]{FFFFC7}} &
		\multicolumn{1}{c|}{\cellcolor[HTML]{FFFFC7}} &
		\cellcolor[HTML]{FFFFC7} \\ \cline{3-3}
		\multicolumn{1}{l|}{} &
		\multicolumn{1}{c|}{\cellcolor[HTML]{FFFFC7}} &
		\multicolumn{1}{c|}{\cellcolor[HTML]{FFFFC7}} &
		\multirow{-2}{*}{\cellcolor[HTML]{FFFFC7}\textbf{Defunciones}} &
		\multicolumn{1}{c|}{\multirow{-2}{*}{\cellcolor[HTML]{FFFFC7}11}} &
		\multicolumn{1}{c|}{\multirow{-2}{*}{\cellcolor[HTML]{FFFFC7}04}} &
		\multicolumn{1}{c|}{\multirow{-2}{*}{\cellcolor[HTML]{FFFFC7}25}} &
		\multicolumn{1}{c|}{\multirow{-2}{*}{\cellcolor[HTML]{FFFFC7}826}} &
		\multicolumn{1}{c|}{\multirow{-2}{*}{\cellcolor[HTML]{FFFFC7}2127}} &
		\multirow{-2}{*}{\cellcolor[HTML]{FFFFC7}2993} \\ \cline{4-10} 
		\multicolumn{1}{l|}{} &
		\multicolumn{1}{c|}{\cellcolor[HTML]{FFFFC7}} &
		\multicolumn{1}{c|}{\multirow{-2}{*}{\cellcolor[HTML]{FFFFC7}\textbf{Mortalidad}}} &
		\cellcolor[HTML]{FFFFC7}\textbf{Tasa*} &
		\multicolumn{1}{c|}{\cellcolor[HTML]{FFFFC7}8.1} &
		\multicolumn{1}{c|}{\cellcolor[HTML]{FFFFC7}2.9} &
		\multicolumn{1}{c|}{\cellcolor[HTML]{FFFFC7}18} &
		\multicolumn{1}{c|}{\cellcolor[HTML]{FFFFC7}608} &
		\multicolumn{1}{c|}{\cellcolor[HTML]{FFFFC7}1567} &
		\cellcolor[HTML]{FFFFC7}2205 \\ \cline{3-10} 
		\multicolumn{1}{l|}{} &
		\multicolumn{1}{c|}{\cellcolor[HTML]{FFFFC7}} &
		\multicolumn{1}{c|}{\cellcolor[HTML]{FFFFC7}} &
		\cellcolor[HTML]{FFFFC7}\textbf{Casos +} &
		\multicolumn{1}{c|}{\cellcolor[HTML]{FFFFC7}1173} &
		\multicolumn{1}{c|}{\cellcolor[HTML]{FFFFC7}4573} &
		\multicolumn{1}{c|}{\cellcolor[HTML]{FFFFC7}19526} &
		\multicolumn{1}{c|}{\cellcolor[HTML]{FFFFC7}43215} &
		\multicolumn{1}{c|}{\cellcolor[HTML]{FFFFC7}11129} &
		\cellcolor[HTML]{FFFFC7}79616 \\ \cline{4-10} 
		\multicolumn{1}{l|}{} &
		\multicolumn{1}{c|}{\multirow{-6}{*}{\cellcolor[HTML]{FFFFC7}\textbf{2021}}} &
		\multicolumn{1}{c|}{\multirow{-2}{*}{\cellcolor[HTML]{FFFFC7}\textbf{Incidencia}}} &
		\cellcolor[HTML]{FFFFC7}\textbf{Tasa*} &
		\multicolumn{1}{c|}{\cellcolor[HTML]{FFFFC7}864} &
		\multicolumn{1}{c|}{\cellcolor[HTML]{FFFFC7}3369} &
		\multicolumn{1}{c|}{\cellcolor[HTML]{FFFFC7}14304} &
		\multicolumn{1}{c|}{\cellcolor[HTML]{FFFFC7}31834} &
		\multicolumn{1}{c|}{\cellcolor[HTML]{FFFFC7}8198} &
		\cellcolor[HTML]{FFFFC7}58649 \\ \cline{2-10} 
		\multicolumn{1}{l|}{} &
		\multicolumn{1}{c|}{\cellcolor[HTML]{E2EFDA}} &
		\multicolumn{1}{c|}{\cellcolor[HTML]{E2EFDA}} &
		\cellcolor[HTML]{E2EFDA}\textbf{Tasa(\%)} &
		\multicolumn{1}{c|}{\cellcolor[HTML]{E2EFDA}0.35} &
		\multicolumn{1}{c|}{\cellcolor[HTML]{E2EFDA}0.099} &
		\multicolumn{1}{c|}{\cellcolor[HTML]{E2EFDA}0.028} &
		\multicolumn{1}{c|}{\cellcolor[HTML]{E2EFDA}0.13} &
		\multicolumn{1}{c|}{\cellcolor[HTML]{E2EFDA}3.7} &
		\cellcolor[HTML]{E2EFDA}0.48 \\ \cline{4-10} 
		\multicolumn{1}{l|}{} &
		\multicolumn{1}{c|}{\cellcolor[HTML]{E2EFDA}} &
		\multicolumn{1}{c|}{\multirow{-2}{*}{\cellcolor[HTML]{E2EFDA}\textbf{Letalidad}}} &
		\cellcolor[HTML]{E2EFDA} &
		\multicolumn{1}{c|}{\cellcolor[HTML]{E2EFDA}} &
		\multicolumn{1}{c|}{\cellcolor[HTML]{E2EFDA}} &
		\multicolumn{1}{c|}{\cellcolor[HTML]{E2EFDA}} &
		\multicolumn{1}{c|}{\cellcolor[HTML]{E2EFDA}} &
		\multicolumn{1}{c|}{\cellcolor[HTML]{E2EFDA}} &
		\cellcolor[HTML]{E2EFDA} \\ \cline{3-3}
		\multicolumn{1}{l|}{} &
		\multicolumn{1}{c|}{\cellcolor[HTML]{E2EFDA}} &
		\multicolumn{1}{c|}{\cellcolor[HTML]{E2EFDA}} &
		\multirow{-2}{*}{\cellcolor[HTML]{E2EFDA}\textbf{Defunciones}} &
		\multicolumn{1}{c|}{\multirow{-2}{*}{\cellcolor[HTML]{E2EFDA}07}} &
		\multicolumn{1}{c|}{\multirow{-2}{*}{\cellcolor[HTML]{E2EFDA}02}} &
		\multicolumn{1}{c|}{\multirow{-2}{*}{\cellcolor[HTML]{E2EFDA}04}} &
		\multicolumn{1}{c|}{\multirow{-2}{*}{\cellcolor[HTML]{E2EFDA}37}} &
		\multicolumn{1}{c|}{\multirow{-2}{*}{\cellcolor[HTML]{E2EFDA}193}} &
		\multirow{-2}{*}{\cellcolor[HTML]{E2EFDA}234} \\ \cline{4-10} 
		\multicolumn{1}{l|}{} &
		\multicolumn{1}{c|}{\cellcolor[HTML]{E2EFDA}} &
		\multicolumn{1}{c|}{\multirow{-2}{*}{\cellcolor[HTML]{E2EFDA}\textbf{Mortalidad}}} &
		\cellcolor[HTML]{E2EFDA}\textbf{Tasa *} &
		\multicolumn{1}{c|}{\cellcolor[HTML]{E2EFDA}5.2} &
		\multicolumn{1}{c|}{\cellcolor[HTML]{E2EFDA}1.5} &
		\multicolumn{1}{c|}{\cellcolor[HTML]{E2EFDA}2.9} &
		\multicolumn{1}{c|}{\cellcolor[HTML]{E2EFDA}27} &
		\multicolumn{1}{c|}{\cellcolor[HTML]{E2EFDA}142} &
		\cellcolor[HTML]{E2EFDA}172 \\ \cline{3-10} 
		\multicolumn{1}{l|}{} &
		\multicolumn{1}{c|}{\cellcolor[HTML]{E2EFDA}} &     
		\multicolumn{1}{c|}{\cellcolor[HTML]{E2EFDA}} &
		\cellcolor[HTML]{E2EFDA}\textbf{Casos +} &
		\multicolumn{1}{c|}{\cellcolor[HTML]{E2EFDA}2001} &
		\multicolumn{1}{c|}{\cellcolor[HTML]{E2EFDA}2025} &
		\multicolumn{1}{c|}{\cellcolor[HTML]{E2EFDA}14174} &
		\multicolumn{1}{c|}{\cellcolor[HTML]{E2EFDA}27487} &
		\multicolumn{1}{c|}{\cellcolor[HTML]{E2EFDA}5178} &
		\cellcolor[HTML]{E2EFDA}50865 \\ \cline{4-10} 
		\multicolumn{1}{l|}{} &
		\multicolumn{1}{c|}{\multirow{-6}{*}{\cellcolor[HTML]{E2EFDA}\textbf{2022}}} &
		\multicolumn{1}{c|}{\multirow{-2}{*}{\cellcolor[HTML]{E2EFDA}\textbf{Incidencia}}} &
		\cellcolor[HTML]{E2EFDA}\textbf{Tasa} &
		\multicolumn{1}{c|}{\cellcolor[HTML]{E2EFDA}1474} &
		\multicolumn{1}{c|}{\cellcolor[HTML]{E2EFDA}1492} &
		\multicolumn{1}{c|}{\cellcolor[HTML]{E2EFDA}10441} &
		\multicolumn{1}{c|}{\cellcolor[HTML]{E2EFDA}20248} &
		\multicolumn{1}{c|}{\cellcolor[HTML]{E2EFDA}3814} &
		\cellcolor[HTML]{E2EFDA}37470 \\ \cline{2-10} 
	\end{tabular}
		}
		{ \footnotesize {Fuente de datos: NOTICOVID, SISCOVID, SINADEF. 
				
				Tasa de mortalidad ajustada 1 000 000 de habitantes* - Tasa de incidencia ajustada 1 000 000 de habitantes*}}
	\end{table}
	
	\clearpage
	
	
	
	
	
	%---------------------------------------------------------------------------
	% CAPÍTULO: AGRADECIMIENTOS
	%--------------------------------------------------------------------------	\section*{Agradecimientos}
	\addcontentsline{toc}{chapter}{Agradecimientos}
	
	\centering
	{\large El presente Boletín Epidemiológico COVID-19 se ha elaborado gracias a la información y esfuerzo conjunto de los Equipos de Inteligencia Sanitaria de los Hospitales y Redes de la GERESA Cusco:
		
		\vspace{0.5cm}
		\noindent
		\begin{minipage}[t]{.45\textwidth}
			\centering
			Hospital Regional del Cusco \\
			M.S.P. Marina Ochoa Linares \vspace{0.5cm}\\
			Hospital Antonio Lorena \\
			Dr. Homero Dueñas \vspace{.5cm}\\
			Hospital Nacional Adolfo Guevara Velasco\\
			M.S.P. Lucio Velasquez Cuentas \vspace{.5cm}\\
			Red de Salud Norte \\
			M.C. Guido Giraldo Alencastre\vspace{0.5cm}\\
			Red de Salud Sur\\
			Lic. Luz Marina Bernable Villasante \vspace{0.5cm}\\	
		\end{minipage}
		\hfill
		\noindent
		\begin{minipage}[t]{.45\textwidth}
			\centering
			Red de Salud La Convención\\
			Dra. Leila  Castellón \vspace{0.5cm}\\
			Red de Salud Chumbivilcas\\
			Lic. Eduarda Benito Calderón \vspace{.5cm}\\
			Red de Salud Canas Canchis Espinar\\
			MC. Heber Jaime Quispe Jihuallanca \vspace{.5cm}\\
			Red de Salud Kimbiri Pichari \\
			Lic. Fiorella Castillo Tinoco\vspace{0.5cm}\\	
		\end{minipage}
		%---------------------------------------------------------------------------
		% CAPÍTULO: AGRADECIMIENTOS
		%---------------------------------------------------------------------------
		\chapter*{Diseño y Edición}
		\addcontentsline{toc}{chapter}{Diseño y Edición}
		\begin{center}
			
			% Como siempre, por orden alfabético del apellid0
			
			MSC. Fátima R. Concha Velasco
			
			M.C. Ana Gabriela Eulalia Moncada Arias 
			
			Ing. Joel Wilfredo Sumerente Ayerbe
		\end{center}
		
		%insertar la última página
		\includepdf[pages={1}]{../editorial/pagina_final.pdf}
		\clearpage
		
	\end{document}
