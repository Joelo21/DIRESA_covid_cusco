\documentclass[12pt,a4paper,openany]{book}
%------------------------------------------------------------------------------------------------------------------------------------------------------------------------------------------------------------------------------
% PACKAGES
%--------------------------------------------------------------------------------------------------------------------------------------------------------------------------------------------------------------------------------

% Selección de idioma
\usepackage[spanish]{babel}

% algo
\usepackage[utf8]{inputenc}

% Par acomodar la foto del editorial
\usepackage{wrapfig}

% Paquetes útiles
\usepackage[table]{xcolor}
\usepackage{amssymb}
\usepackage{amsmath}
%\usepackage{mathbbol}
\usepackage{bbm}
\usepackage{amsthm}
\usepackage{pdfpages}
\usepackage{graphicx,color,psfrag}
\usepackage{epstopdf}
\usepackage{pdflscape}
\usepackage{tabularx}
\usepackage{longtable}
\usepackage{breakurl}
\usepackage{enumitem}
\usepackage[normalem]{ulem}
\usepackage{blindtext}
\usepackage{mathtools,breqn}


\usepackage{fancyhdr}
\usepackage{graphicx}

\usepackage{fnpct}

\usepackage{subcaption}

\usepackage{newpxtext}
\usepackage{lscape}

% La mejor fuente de la letra: https://www3.gobiernodecanarias.org/medusa/ecoblog/lortrodm/files/2015/03/tarea-formatos-word.pdf
% caption fonts
%\usepackage[font={large,bf}]{caption} 
\usepackage[T1]{fontenc}
\usepackage{verdana}


\usepackage{setspace}
\usepackage{longtable}
\usepackage{threeparttable}  
\usepackage{tabulary}
\usepackage{booktabs}
\usepackage{float}
\usepackage{caption}
\usepackage{subcaption}
\usepackage{rotating}
\usepackage[titletoc,title]{appendix}

\usepackage{array,multirow}

\usepackage[round]{natbib}
\bibpunct{(}{)}{;}{a}{,}{;}
\setcounter{MaxMatrixCols}{10}

\topmargin=-1.8cm \textheight=23.8cm \oddsidemargin=-0.3cm
\evensidemargin=-0.5cm \textwidth=17.1cm

\newtheorem{theorem}{Theorem}
\newtheorem{corollary}[theorem]{Corollary}
\newtheorem{proposition}{Proposition}
\newtheorem{assumption}{Assumption}
\newtheorem{assumption2}{Assumption A}

\newtheorem{lemma}{Lemma}

\usepackage{tikz}
\usetikzlibrary{positioning}
\tikzset{>=stealth}
\usepackage{amsmath}
\usepackage{verbatim}
\usetikzlibrary{arrows,shapes}

% Definir colores
\definecolor{mycolor1}{RGB}{221, 165, 230}
\definecolor{mycolor2}{RGB}{54, 56, 120}	
\definecolor{mycolor3}{RGB}{205, 24, 24}
\definecolor{mycolor4}{RGB}{164, 93, 93}
\definecolor{mycolor5}{RGB}{243, 149, 13}
\definecolor{mycolor6}{RGB}{3, 83, 151}
\definecolor{mycolor7}{RGB}{52, 103, 81}
\usepackage[colorlinks=true,linkcolor=myblue, allcolors=mycolor2]{hyperref}
\usepackage{soul}

% Tablas
\usepackage{tabularx}
\usepackage{multirow}
\usepackage{multicol} 
\usepackage{booktabs}%\usepackage{booktabs, calc} %This is the package to use to have nice-looking tables. More documentation on the tables in LateX: https://www.tug.org/pracjourn/2007-1/mori/mori.pdf
\usepackage{threeparttable} 

\usepackage{lmodern}
\usepackage{booktabs}
\usepackage{pgfplots}

\graphicspath{{../figuras/}}

\begin{document}
	
	%---------------------------------------------------------------------------
	% TITLE PAGE
	%---------------------------------------------------------------------------
	\doublespacing
	
	\title{Boletín COVID-19}
	\author{Autores}
	
	\date{}
	
	%\maketitle
	
	
	%\thispagestyle{empty}\baselineskip1.385\baselineskip \newpage{}
	
	\pagestyle{plain}\pagenumbering{arabic}
	
	%insertar el cover
	\includepdf[pages={1}]{../editorial/portada.pdf}
	\clearpage
	
	\pagestyle{plain}\pagenumbering{arabic}
	
	\clearpage
	
	\begin{center}
		
		{\large Gerencia Regional de Salud}
		
		\textbf{MSP. Javier Ramírez Escóbar}
		
		Gerente Regional \vspace{1.0cm}
		
		Dirección Ejecutiva de Inteligencia Sanitaria
		
		\textbf{MSP. Darío Francisco Navarro Mendoza}
		
		Director
		
		\vspace{1.5cm}
		\noindent
		\begin{minipage}[t]{.45\textwidth}
			\centering
			Dirección de Epidemiología e Investigación  \\
			\textbf{MSC. Fátima R. Concha Velasco}\\
			Directora \vspace{1.0cm}\\
			% Por orden alfabético del apellido
			\textit{Equipo de Epidemiología e Investigación }\vspace{.5cm}\\
			Bach. Eddie Briam Cassa Chavez \\
			Lic. Nadia Isabel Cáceres Pillco \\
			TAP. Edgar Waldo Capcha Salcedo \\
			M.S.P. Pablo Fidel Grajeda Ancca \\
			M.C. Alex Jaramillo Corrales \\ 
			M.C. Katia Luque Quispe \\
			M.C. Lucero Guadalupe Contreras Masias \\
			M.C. Jesus Kevin Perez Castilla \\
			Lic. Enf. Ruth Nelly Oscco Abarca \\
			Ing. Joel Wilfredo Sumerente Ayerbe \\
			Lic. Enf. Guinetta Margarita Yabar Herrera \vspace{1.5cm}\\	
		\end{minipage}
		\hfill
		\noindent
		\begin{minipage}[t]{.45\textwidth}
			\centering
			Dirección de Estadística, Informática y Telecomunicaciones\\
			\textbf{Ing. Santiago Quispe Peralta} \\
			Director \vspace{1.0cm} \\
			% Por orden alfabético del apellido
			\textit{Equipo de Estadística, Informática y Telecomunicaciones} \vspace{.5cm} \\
			Ing. Iván Atayupanqui Rondón \\
			Ing. Miguel Ángel Campana Alarcón \\
			Ing. Uriel Lacuta Farfán \\
			Ing. Jorge Fernando Lovatón Ramos \\
			Ing. Danny Robert Moscoso Sánchez \\
			Ing. Irvin Condori Champi \\
			Ing. Abel Rimasca Chacón
			 \vspace{1.5cm}\\
		\end{minipage}
		Secretaria: Sra. Ruth Baca Mendoza
	\end{center}
	\let\cleardoublepage\clearpage
	\tableofcontents
	\begin{center}
		Visite nuestro Dashboard interactivo sobre COVID-19 haciendo clic \href{https://sites.google.com/view/geresacusco/inicio}{AQUÍ}
	\end{center}
	
	%\mainmatter
	%---------------------------------------------------------------------------
	% CAPÍTULO: EDITORIAL
	%---------------------------------------------------------------------------
	
	\pagebreak
	
	\section*{Editorial}	\addcontentsline{toc}{chapter}{Editorial}
	\begin{wrapfigure}{l}{5cm}
		\label{wrap-fig:1}
		\includegraphics[width=5
		cm]{../editorial/editorial_karla}
		\caption*{
			\centering
			BLGA. CARLA CHAPARRO ZAMALLOA	
		
			\textit{Coordinadora de Biología SAMUE Cusco}
								}
	\end{wrapfigure}

	\noindent \textbf{EL SAMUE Y EL COVID-19 EN LA CIUDAD DEL CUSCO}
		
En Abril del año 2020 en la Ciudad del Cusco frente a la pandemia global que afrontaba el mundo, el Gobierno regional del Cusco implemento el \textbf{Equipo de Respuesta Rápida (ERR)} para la vigilancia epidemiológica de casos sospechosos de COVID-19 el cual estaba conformado por profesionales Médicos, enfermeras, Biólogos y conductores; regida bajo DS N.º 088-MINSA/2020/CDC y consistía básicamente en acudir a domicilio  según llamada telefónica a los responsables del área de Epidemiología para la realización de la búsqueda activa de casos sospechosos y a si establecer cercos epidemiológicos, el trabajo del equipo consistía en  investigar cada caso, entrevista familiar, evaluación, seguimiento de contactos, orientación, toma de muestras, conservación, transporte de muestras y entrega de resultados; todo esto en aquellos días donde regia el estado de emergencia Nacional  que inició el 16 de marzo del 2020 donde  se dispone la inmovilización social obligatoria de todas las personas en sus domicilios por lo cual para todos aquellos que conformábamos el equipo significaba transitar por las calles desiertas de Cusco, sin medios de transporte, con locales comerciales cerrados y utilizando incomodos atuendos, mascarillas y caretas en un panorama  desolador.

Como parte de la vigilancia epidemiológica, se controlaba a los viajeros que aun con las normas de restricción se trasladaban en un sinfín de direcciones, también se hacía seguimiento a pasajeros varados en hospedajes, hoteles y albergues temporales.  

Esta rutina de trabajo duro unos 2 meses puesto que los casos de pacientes sintomáticos respiratorios y casos positivos se elevaron exponencialmente frente a la primera ola, el equipo tuvo que adaptarse y además de realizar todo lo antes mencionado, se atendía pacientes sintomáticos que empezaron a asistir a nuestra base que hasta ahora se encuentra en el  \textbf{“Estadio Garcilaso”}, esto pronto se convirtió en colas diarias que no terminaban nunca a pesar del esfuerzo del equipo por atender a todos. Así se afrontó también la segunda ola.

En agosto del 2020 se implementó el “Servicio Médico de Apoyo de Atención Prehospitalaria” \textbf{(SAMUE CUSCO)}, el servicio que brinda, asistencia de manera oportuna y con calidad, cuando se presenta una urgencia o emergencia, en el lugar donde se encuentre de manera rápida, eficiente y gratuita.

El SAMUE CUSCO se fusionó con el ERR en marzo del 2021 para brindar una atención integral y eficiente a la población. Es así que, se realiza atención de llamadas de emergencia a través de la línea telefónica mediante regulación de una central telefónica (CRUE) cuyo número es 084-216464 o al 106 anexo 31 o 32, la atención de emergencias en el lugar donde éstas ocurran, por personal altamente calificado, traslados terrestres de pacientes que se encuentren en situación de emergencia, capacitación en primeros auxilios y brindando las pautas básicas para salvar la vida de la persona que lo necesita, hasta la llegada del equipo.

Las funciones del Equipo han ido cambiando; sin embargo, nunca se ha dejado la vigilancia, atención de pacientes sintomáticos con diagnósticos oportunos y seguimiento de pacientes COVID -19 resguardando la salud de la población de la Región del Cusco, puesto que desde el 2020 permanece el punto permanente de toma de muestra y el Equipo ha participado en diferentes campañas de descarte con las pruebas COVID disponibles, realizando barridos en los puntos más propensos como terminales terrestres, mercados y atendiendo solicitudes de diferentes Municipalidades de la Región así como de instituciones como la Policía Nacional y participando junto a los Militares en intervenciones para la protección de población vulnerable.

En los meses en los que muchos enfermos padecían por falta de oxígeno medicinal, y clamaban por conseguir este vital recurso, el equipo se dedicó además a dotar de oxigeno siendo un enlace importante entre la planta de oxigeno medicinal y la población que lo requería. 
Además, el equipo ha participado  en estudios de  seroprevalencia, prevalencia, junto a diferentes Instituciones como INS, ESSALUD, Universidades.

Todas las intervenciones que el equipo hace frente a la pandemia de COVID 19, es porque cuenta con un grupo de \textbf{profesionales Biólogos}, los cuales vienen afrontado de forma valiente soportando rigurosas jornadas de trabajo de varias horas, utilizando el equipo de protección personal reglamentario, tal vez la peor parte de los profesionales en primera línea contra el COVID 19 en Cusco, ya que llevan la función de tomar muestras, (hisopado nasofaríngeo y orofaríngeo) directamente del paciente por lo cual se exponen a la emisión de aerosoles y aunque siempre nos hemos ceñido a las normas de bioseguridad y el buen uso de lo EPPs, eso no evitaba el temor de los primeros meses y el afrontar tomar muestras de una gran cantidad de pacientes sintomáticos en comparación con profesionales que en esos tiempos difíciles hacían trabajo remoto, por temor al contagio y aun en los centros de salud donde solo atendía casos de emergencia, recordando además, lo difícil que es mantenerse trabajando varias horas utilizando de forma adecuada los equipos de protección. 

En cada uno de los picos más altos de las olas, los \textbf{Biólogos} hemos tenido que trabajar más horas de las normales, a fin de registrar los resultados de forma oportuna, en las plataformas virtuales, para así continuar con los demás procesos. 

\begin{center}
	\textbf{EL SAMUE Y EL COVID-19 EN LA CIUDAD DEL CUSCO}
\end{center}

\begin{center}
	\includegraphics[width=0.80\linewidth]{../editorial/imag_karla1}

\vspace{2mm}
\includegraphics[width=0.80\linewidth]{../editorial/imag_karla2}
\end{center}

	
		
	%---------------------------------------------------------------------------
	% CAPÍTULO: Metodología
	%-------------------------------- -------------------------------------------
	%insertar el cover del capitulo
	\includepdf[pages={1}]{../editorial/1.pdf}
	\clearpage	
	\section*{Metodología}
	\addcontentsline{toc}{chapter}{Metodología}
	
	
	
	\noindent El presente Boletín tiene el objetivo de informar sobre los principales indicadores epidemiológicos y
	de gestión hospitalaria, para hacer el seguimiento de la pandemia en nuestra región y tomar mejores decisiones. Este Boletín tiene una metodología de tipo descriptiva.
	
	La introdución de la variante ómicron ha sido causa de la tercera y cuarta ola en nuestra región, por ésta razon en esta edición del boletín se considera los datos desde el año 2021 hasta la
	semana epidemiológica  48 (03 de diciembre) del presente año, para que el lector pueda hacer las comparaciones e inferencia del comportamiento de la segunda, tercera y cuarta ola en nuestra región. Asimismo, en la descripción de cada indicador o figura se mencionará si el análisis de
	la información incluye otro periodo.
	
	Los datos analizados incluyen: a) características generales: sexo, edad, casos confirmados,
	fallecidos; b) características clínicas: síntomas reportados, casos confirmados sintomáticos, casos
	confirmados asintomáticos y comorbilidades; c) indicadores epidemiológicos: sistema de vigilancia
	epidemiológica, tasa de mortalidad, tasa de positividad de pruebas diagnósticas, casos activos –
	recuperados, y exceso de muerte por todas las causas, y d) indicadores de gestión hospitalaria: 
	ocupación de camas UCI y No UCI en la Región. En este boletín se considera como caso positivo de
	COVID-19, sólo a aquellos que tienen una prueba antigénica o molecular positiva, salvo en ciertas
	estimaciones, en cuya descripción se detalla si se utilizó otro tipo de examen diagnóstico.
	Las fuentes de información son las bases de datos de NOTI WEB (aplicativo del Sistema de
	Vigilancia Epidemiológica - COVID-19), SISCOVID (Sistema Integrado para COVID-19), SINADEF
	(Sistema Informático Nacional de Defunciones), SICOVAC-HIS MINSA (Base de datos de vacunación
	por COVID-19), Reporte de Disponibilidad de Camas de Hospitalización y datos de la Oficina de
	Referencias-Contrarreferencias de la Dirección de Emergencias y Desastres de GERESA-Cusco.

	Se usaron frecuencias absolutas y relativas para la descripción de los datos cualitativos. Para la
	descripción de datos cuantitativos se calcularon tasas (mortalidad, pruebas diagnósticas, incidencia de
	casos), promedios (ocupación de camas hospitalarias, fallecidos por COVID y fallecidos por todas las
	causas). Para describir la tendencia se representaron los datos cuantitativos y frecuencias relativas en
	intervalos de 7 días (semana epidemiológica). En las variables de sistema de vigilancia epidemiológica
		(1 prueba por 100,000 habitantes) y ocupación de cama (adecuado, menor a 70$\%$, moderado,
	entre 75 a 90$\%$ y limitado, más de 90$\%$), siendo todos los puntos de referencia sugeridos por la
	Organización Mundial de la Salud. Para el análisis de exceso de mortalidad, se usó la metodología
	descrita por C. Giattino, H. Ritchie, M. Roser, E. Ortiz-Ospina, y J. Hasell en el artículo .Excess
	mortality during the Coronavirus pandemic (COVID-19). Published online at OurWorldInData.org.
	La descripción de dichas variables se hace de manera regional y provincial. En la presente edición
	se hace una descripción de la tasa de incidencia, tasa de mortalidad, tasa de positividad por prueba
	molecular y antigénica, y exceso de defunciones de todas las provincias de nuestra región. El lector
	interesado en un análisis distrital de los casos y defunciones puede encontrar dicha información en
	los links correspondientes.
	
	%---------------------------------------------------------------------------
	% CAPÍTULO: CARACTERÍSTICAS GENERALES
	%-------------------------------- -------------------------------------------
	%insertar el cover del capitulo
	\includepdf[pages={1}]{../editorial/2.pdf}
	\clearpage	
	\section*{Características Generales}
	\addcontentsline{toc}{chapter}{Características Generales}
	\noindent En la Figura \ref{fig:casos_edad_sexo} se muestra la cantidad de casos confirmados de COVID-19, por prueba antigénica y molecular por grupo etario (en intervalos de 10 años) y sexo. Observamos que el grupo
	etario de 30 a 39 años presenta el mayor número de casos acumulados (14 467 casos acumulados), siendo mayor en el sexo femenino; como en todos los grupos etarios a partir de los 10 años 
	
	\begin{figure}[h]
		\caption{Casos Confirmados de COVID-19 según Grupo de Edad y Sexo en la Región Cusco hasta la SE 48-2022(*).}\label{fig:casos_edad_sexo}
		\begin{center}
			\includegraphics[width=0.75\linewidth]{../figuras/casos_etapavida_2022}
		\end{center}
		{\footnotesize {Fuente de datos: SISCOVID, NOTICOVID. (*) Sólo se incluye información del 2022.}}
	\end{figure}
	\pagebreak
	
	
	La Figura \ref{fig:fallecidos_edad_sexo}  muestra el número de muertes reportadas por COVID-19 conforme al grupo etario y sexo hasta la SE 48.  Se observa que el mayor número de muertes corresponde al grupo etario de 80 a 89 años, con predominio del sexo femenino (59 casos en mujeres y 42 casos en varones), no habiendo cambiado los valores en comparación al reporte anterior.
%	
	\begin{figure}[h]
		\caption{Casos fallecidos por COVID-19 según Grupo Etario y Sexo en la Región Cusco hasta la SE 48-2022(*).}\label{fig:fallecidos_edad_sexo}
		\begin{center}
			\includegraphics[width=0.75\linewidth]{../figuras/defunciones_etapavida_2022}
		\end{center}
		{\footnotesize {Fuente de datos: SISCOVID, NOTICOVID.(*) Sólo se incluye información del 2022.}}
	\end{figure}
%	
	
	
	\cleardoublepage
	%---------------------------------------------------------------------------
	% CAPÍTULO: CARACTERÍSTICAS CLÍNICAS
	%---------------------------------------------------------------------------
	%insertar el cover del capitulo
	\includepdf[pages={1}]{../editorial/3.pdf}
	
	\clearpage

	
	\section*{Características Clínicas}
	\addcontentsline{toc}{chapter}{Características Clínicas}	
	\noindent En la Figura \ref{fig:sintomas} se presentan los síntomas más frecuentes de COVID-19. La tos (22$\%$) y el dolor de garganta (20.1$\%$) continúan siendo los sintomas más reportados. Dentro de los signos más frecuentes (Figura \ref{fig:signos}), el exudado faríngeo se mantiene como signo más prevalente (85.1$\%$). 
	
	\noindent La Figura \ref{fig:comorbilidades} muestra la frecuencia de comorbilidades en pacientes con COVID-19, siendo las más prevalentes; la diabetes (25.4$\%$), obesidad (22$\%$) e inmunodeficiencia (19.7$\%$).
	  
	 
	
	\begin{figure}[h]
		\caption{Síntomas más frecuentes de los pacientes diagnosticados por COVID-19 en la Región Cusco hasta la SE 48-2022.  }\label{fig:sintomas}
		\begin{center}
			\includegraphics[width=0.85\linewidth]{../figuras/figura_sintoma.pdf}
		\end{center}
		{\footnotesize {Fuente de datos: SISCOVID, NOTICOVID.}}
	\end{figure}
	
	\begin{figure}[h]
		\caption{Signos más frecuentes de los pacientes diagnosticados por COVID-19 en la Región Cusco hasta la SE 48-2022.}\label{fig:signos}
		\begin{center}
			\includegraphics[width=0.65\linewidth]{../figuras/figura_signo.pdf}
		\end{center}
		{\footnotesize {Fuente de datos: NOTICOVID.}}
	\end{figure}
	
	
	  
	\begin{figure}[h]
				\caption{Comorbilidades más frecuentes de los pacientes diagnosticados por COVID-19 en la Región Cusco hasta la SE 48-2022. }\label{fig:comorbilidades}
		\begin{center}
			\includegraphics[width=0.65\linewidth]{../figuras/figura_comorbilidad.pdf}
		\end{center}
		{\footnotesize {Fuente de datos: NOTICOVID.}}
	\end{figure}
	\clearpage
	En la Figura \ref{fig:sintomaticos_asintomati} se evidencia la curva epidémica de casos sintomáticos y asintomáticos desde el año 2020 en comparación con los años 2021-2022. Para la SE 48, se observa que la curva de los casos sintomáticos, actualmente se encuentra en ascenso desde la SE 45. Situación similar se mostró el año 2020 para la semana homónima.
	
	\begin{figure}[h]
		\caption{Casos Sintomáticos y Asintomáticos de COVID-19 por Semana Epidemiológica en la Región Cusco, hasta la SE 48-2022.  }\label{fig:sintomaticos_asintomati}
		
		\begin{center}
			\includegraphics[width=0.95\linewidth]{../figuras/sintomaticos_20_21_22.png}
		\end{center}
		{\footnotesize {Fuente de datos: SISCOVID, NOTICOVID.}}
	\end{figure}
	\clearpage
	
	
	%---------------------------------------------------------------------------
	% CAPÍTULO: ANÁLISIS DE INDICADORES
	%---------------------------------------------------------------------------
	%insertar el cover del capitulo
	\includepdf[pages={1}]{../editorial/4.pdf}
	\clearpage
	
	\section*{Análisis de Indicadores}
	\addcontentsline{toc}{chapter}{Análisis de Indicadores}
	\subsection*{Tasa de Incidencia y Tasa de Positividad}
	\noindent La evolución de la tasa de incidencia a lo largo del tiempo se encuentra graficada en la Figura \ref{fig:incidencia}, se evidencia que la curva registra un ligero ascenso desde la SE 45.
	
	Para la SE 48 la tasa de incidencia fue de 81 casos / 1 000 000 habitantes
	
	
	\begin{figure}[h]
		\caption{Tasa de Incidencia de COVID-19 en la región Cusco hasta la SE 48-2022(*).  }\label{fig:incidencia}
		\begin{center}
			\includegraphics[width=0.90\linewidth]{../figuras/tasa_incidencia_2021_2022.png}
		\end{center}
		{\footnotesize {Fuente de datos: SISCOVID, NOTICOVID. (*) Se considera como caso positivo sólo a los pacientes con prueba molecular o antigénica positiva.}}
	\end{figure}
	
	\pagebreak
	
	
	La Figura \ref{fig:total_muestras_procesada} muestra un comparativo diario de las tasas de positividad ($\%$) de pruebas moleculares (PCR) y antigénicas (AG). Se puede evidenciar que para la SE 48, ambas tasas se encuentran en incremento.
	
	\begin{figure}[h]
		\caption{Tasa de positividad para muestras antigénicas y moleculares por COVID-19 en la región Cusco hasta la SE 48-2022. }\label{fig:total_muestras_procesada}
		\begin{center}
			\includegraphics[width=0.80\linewidth]{../figuras/positividad_diaria_2021_2022.png}
		\end{center}
		{\footnotesize {Fuente de datos: SISCOVID, NOTICOVID.}}
	\end{figure}
	
	
	
	Las Figuras \ref{fig:positividad_pcr} y \ref{fig:positividad_ag} muestran el número de positivos detectados por pruebas moleculares y antigénicas y sus tasas de positividad. Para la SE 48, la positividad de pruebas moleculares se encuentra en un 21$\%$, y 28.5$\%$ para las pruebas antigénicas.
	
	
	\begin{landscape}
		\begin{figure}[h]
			\caption{Positividad y Tasa de Positividad de pruebas moleculares tomadas por COVID-19 en la región Cusco hasta la SE 48-2022.}\label{fig:positividad_pcr}
			\begin{center}
				\includegraphics[width=0.90\linewidth]{../figuras/positividad_pcr.pdf}
			\end{center}
			{\footnotesize {Fuente de datos: SISCOVID, NOTICOVID.}}
		\end{figure}
	\end{landscape}
	\clearpage
	\begin{landscape}
		
		\begin{figure}[h]
			\caption{ Positividad y Tasa de Positividad de pruebas antigénicas tomadas por COVID-19 en la región Cusco hasta la SE 48-2022.}\label{fig:positividad_ag}
			\begin{center}
				\includegraphics[width=0.90\linewidth]{../figuras/positividad_ag.pdf}
			\end{center}
			{\footnotesize {Fuente de datos: SISCOVID, NOTICOVID.}}
		\end{figure}
	\end{landscape}
	%\clearpage
	%\subsection*{Análisis de Indicadores en población pediátrica}
	%\noindent Las figuras inferiores (Figura \ref{fig:niños_2021} y Figura \ref{fig:niño_2022} ) muestran el número de casos positivos(barras rosadas) y número de muertes (línea negra) en la población pediátrica para los quinquenios respectivos.
	%
	%Para el año 2020 el mayor número de casos positivos en niños se reportó en la SE 35 tras lo cuál el número de casos se ha mantenido variable en el resto del año. Con respecto a las defunciones,  el número máximo de muertes reportada fue de 1 por semana en cada grupo etario. 
	%
	%En el año 2021 el número de casos positivos se ha mantenido mas o menos constante a lo largo del año hasta la SE 52, tras lo cual se incrementa exponencialmente en todos los quinquenios. Del mismo modo el número de muertes no excedió de 1 muerte por semana en los tres grupos etarios.  
	%
	%Finalmente para el año 2022 se evidencia un incremento considerable en 
	%el número de casos positivos en población pediátrica hasta la SE 06, tras lo cuál se observa el descenso en el número de casos, a partir de las SE 07 no se reportaron muertes en el grupo etario de 0 a 15 años.  
	%
	%
	%En la Cuadro \ref{table:1} se evidencia el número de casos positivos, las defunciones y la tasa de letalidad de menores de 15 años, agrupados por quinquenios. Se evidencia que hasta la SE 12 del año 2022 se reportaron 2503 casos positivos y 6 defunciones en este grupo etario, teniendo una letalidad de 0,24 $\%$, porcentaje que es menor a lo reportado en las primeras olas de afección por COVID-19.   
	%
	%\begin{figure}[h]
	%	\caption{Casos y defunciones por quinquenio en población pediátrica 2020-2021.}
	%	\label{fig:niños_2021}
	%	\centering
	%	\begin{subfigure}[b]{0.45\textwidth}
		%		\centering
		%		\includegraphics[width=\textwidth]{../figuras/niños_2020_1.pdf}
		%		\caption{De 0 a 5 años - \textbf{2020}}
		%		%\label{fig:}
		%	\end{subfigure}
	%	\hfill
	%	\begin{subfigure}[b]{0.45\textwidth}
		%		\centering
		%		\includegraphics[width=\textwidth]{../figuras/niños_2021_1.pdf}
		%		\caption{De 0 a 5 años - \textbf{2021}}
		%		%\label{fig:70 a 79 años}
		%	\end{subfigure}
	%	
	%	\vspace{10mm}
	%	\begin{subfigure}[b]{0.45\textwidth}
		%		\centering
		%		\includegraphics[width=\textwidth]{../figuras/niños_2020_2.pdf}
		%		\caption{De 6 a 11 años - \textbf{2020}}
		%		%\label{fig:60 a 69 años}
		%	\end{subfigure}
	%	\hfill
	%	\begin{subfigure}[b]{0.45\textwidth}
		%		\centering
		%		\includegraphics[width=\textwidth]{../figuras/niños_2021_2.pdf}
		%		\caption{De 6 a 10 años - \textbf{2021}}
		%		%\label{fig:50 a 59 años}
		%	\end{subfigure}
	%	
	%	\vspace{10mm}
	%	\begin{subfigure}[b]{0.45\textwidth}
		%		\centering
		%		\includegraphics[width=\textwidth]{../figuras/niños_2020_3.pdf}
		%		\caption{De 11 a 15 años - \textbf{2020}}
		%		%\label{fig:40 a 49 años}
		%	\end{subfigure}
	%	\hfill
	%	\begin{subfigure}[b]{0.45\textwidth}
		%		\centering
		%		\includegraphics[width=\textwidth]{../figuras/niños_2021_3.pdf}
		%		\caption{De 11 a 15 años - \textbf{2021}}
		%		%\label{fig:40 a 49 años}
		%	\end{subfigure}
	%\end{figure}
	%
	%\begin{figure}[h]
	%	\caption{Casos y defunciones por quinquenio en población pediátrica - 2022.}
	%	\label{fig:niño_2022}
	%	\centering
	%	\begin{subfigure}[b]{0.45\textwidth}
		%		\centering
		%		\includegraphics[width=\textwidth]{../figuras/niños_2022_1.pdf}
		%		\caption{De 0 a 5 años - \textbf{2022}}
		%		%\label{fig:40 a 49 años}
		%	\end{subfigure}
	%	
	%	\centering
	%	\begin{subfigure}[b]{0.45\textwidth}
		%		\centering
		%		\includegraphics[width=\textwidth]{../figuras/niños_2022_2.pdf}
		%		\caption{De 6 a 10 años - \textbf{2022}}
		%		%\label{fig:40 a 49 años}
		%	\end{subfigure}
	%	
	%	\vspace{10mm}
	%	\begin{subfigure}[b]{0.45\textwidth}
		%		\centering
		%		\includegraphics[width=\textwidth]{../figuras/niños_2022_3.pdf}
		%		\caption{De 11 a 15 años - \textbf{2022}}
		%		%\label{fig:40 a 49 años}
		%	\end{subfigure}
	%\end{figure}
	%\clearpage
	%	\begin{table}[h]
		%			\caption{Tasa de letalidad de COVID-19 en población pediátrica 2020-2022.}
		%				\label{table:1}
		%		\resizebox{\textwidth}{!}{%
			%				\begin{tabular}{lcccc}
		\cline{2-5}
		\multicolumn{1}{l|}{} &
		\multicolumn{1}{c|}{\cellcolor[HTML]{ECF4FF}Etapa de Vida} &
		\multicolumn{1}{c|}{\cellcolor[HTML]{ECF4FF}Positivos} &
		\multicolumn{1}{c|}{\cellcolor[HTML]{ECF4FF}Defunciones} &
		\multicolumn{1}{c|}{\cellcolor[HTML]{ECF4FF}Letalidad(\%)} \\ \hline
		\multicolumn{1}{|l|}{} &
		\multicolumn{1}{c|}{0 a 5 años} &
		\multicolumn{1}{c|}{851} &
		\multicolumn{1}{c|}{4} &
		\multicolumn{1}{c|}{0.47} \\ \cline{2-5} 
		\multicolumn{1}{|l|}{} &
		\multicolumn{1}{c|}{6 a 10 años} &
		\multicolumn{1}{c|}{705} &
		\multicolumn{1}{c|}{2} &
		\multicolumn{1}{c|}{0.28} \\ \cline{2-5} 
		\multicolumn{1}{|l|}{} &
		\multicolumn{1}{c|}{11 a 15 años} &
		\multicolumn{1}{c|}{1146} &
		\multicolumn{1}{c|}{2} &
		\multicolumn{1}{c|}{0.17} \\ \cline{2-5} 
		\multicolumn{1}{|l|}{\multirow{-4}{*}{2020}} &
		\multicolumn{1}{c|}{\cellcolor[HTML]{ECF4FF}Total} &
		\multicolumn{1}{c|}{\cellcolor[HTML]{ECF4FF}2702} &
		\multicolumn{1}{c|}{\cellcolor[HTML]{ECF4FF}8} &
		\multicolumn{1}{c|}{\cellcolor[HTML]{ECF4FF}0.30} \\ \hline
		&
		\multicolumn{1}{l}{} &
		\multicolumn{1}{l}{} &
		\multicolumn{1}{l}{} &
		\multicolumn{1}{l}{} \\ \cline{2-5} 
		\multicolumn{1}{l|}{} &
		\multicolumn{1}{c|}{\cellcolor[HTML]{ECF4FF}Etapa de Vida} &
		\multicolumn{1}{c|}{\cellcolor[HTML]{ECF4FF}Positivos} &
		\multicolumn{1}{c|}{\cellcolor[HTML]{ECF4FF}Defunciones} &
		\multicolumn{1}{c|}{\cellcolor[HTML]{ECF4FF}Letalidad(\%)} \\ \hline
		\multicolumn{1}{|l|}{} &
		\multicolumn{1}{c|}{0 a 5 años} &
		\multicolumn{1}{c|}{517} &
		\multicolumn{1}{c|}{7} &
		\multicolumn{1}{c|}{1.4} \\ \cline{2-5} 
		\multicolumn{1}{|l|}{} &
		\multicolumn{1}{c|}{6 a 10 años} &
		\multicolumn{1}{c|}{484} &
		\multicolumn{1}{c|}{4} &
		\multicolumn{1}{c|}{0.83} \\ \cline{2-5} 
		\multicolumn{1}{|l|}{} &
		\multicolumn{1}{c|}{11 a 15 años} &
		\multicolumn{1}{c|}{1284} &
		\multicolumn{1}{c|}{3} &
		\multicolumn{1}{c|}{0.23} \\ \cline{2-5} 
		\multicolumn{1}{|l|}{\multirow{-4}{*}{2021}} &
		\multicolumn{1}{c|}{\cellcolor[HTML]{ECF4FF}Total} &
		\multicolumn{1}{c|}{\cellcolor[HTML]{ECF4FF}2285} &
		\multicolumn{1}{c|}{\cellcolor[HTML]{ECF4FF}14} &
		\multicolumn{1}{c|}{\cellcolor[HTML]{ECF4FF}0.61} \\ \hline
		&
		\multicolumn{1}{l}{} &
		\multicolumn{1}{l}{} &
		\multicolumn{1}{l}{} &
		\multicolumn{1}{l}{} \\ \cline{2-5} 
		\multicolumn{1}{l|}{} &
		\multicolumn{1}{c|}{\cellcolor[HTML]{ECF4FF}Etapa de Vida} &
		\multicolumn{1}{c|}{\cellcolor[HTML]{ECF4FF}Positivos} &
		\multicolumn{1}{c|}{\cellcolor[HTML]{ECF4FF}Defunciones} &
		\multicolumn{1}{c|}{\cellcolor[HTML]{ECF4FF}Letalidad(\%)} \\ \hline
		\multicolumn{1}{|l|}{} &
		\multicolumn{1}{c|}{0 a 5 años} &
		\multicolumn{1}{c|}{651} &
		\multicolumn{1}{c|}{5} &
		\multicolumn{1}{c|}{0.77} \\ \cline{2-5} 
		\multicolumn{1}{|l|}{} &
		\multicolumn{1}{c|}{6 a 10 años} &
		\multicolumn{1}{c|}{738} &
		\multicolumn{1}{c|}{1} &
		\multicolumn{1}{c|}{0.14} \\ \cline{2-5} 
		\multicolumn{1}{|l|}{} &
		\multicolumn{1}{c|}{11 a 15 años} &
		\multicolumn{1}{c|}{1114} &
		\multicolumn{1}{c|}{0} &
		\multicolumn{1}{c|}{0} \\ \cline{2-5} 
		\multicolumn{1}{|l|}{\multirow{-4}{*}{2022}} &
		\multicolumn{1}{c|}{\cellcolor[HTML]{ECF4FF}Total} &
		\multicolumn{1}{c|}{\cellcolor[HTML]{ECF4FF}2503} &
		\multicolumn{1}{c|}{\cellcolor[HTML]{ECF4FF}6} &
		\multicolumn{1}{c|}{\cellcolor[HTML]{ECF4FF}0.24} \\ \hline
	\end{tabular}
			%		}
		%	
		%	
		%		{\footnotesize Fuente de datos: NOTICOVID, SISCOVID, SINADEF. Actualizado a la SE 12-2022.}
		%	\end{table}	
	\clearpage
	
	\subsection*{Análisis de la Mortalidad}
	
	\noindent En la Figura \ref{fig:mortalidad_edad} se muestra la mortalidad semanal para las edades agrupadas en decenios,  se puede observar que en las últimas semanas, no se han notificado casos de muerte en ningún grupo etario.
	
	\begin{figure}[h]
		\caption{Tasa de Mortalidad por COVID-19 por Grupo Etario hasta la SE 48-2022.}\label{fig:mortalidad_edad}
		\begin{center}
			\includegraphics[width=0.65\linewidth]{../figuras/mortalidad_edad_2021_2022.pdf}
		\end{center}
		{\footnotesize Fuente de datos: SINADEF} 
	\end{figure}
	
	
	La Figura \ref{fig:mortalidad_grupo_edad} y 
	\ref{fig:mortalidad_grupo_edad_2},
	muestra la relación entre la tasa de mortalidad y la vacunación en todos los grupos etarios. Las líneas de referencia  representan la fecha de inicio de la vacunación (línea roja) para el correspondiente grupo etario y el inicio de la tercera y cuarta ola pandémica (líneas verdes). Si hacemos una comparación entre la tasa de mortalidad pre y post vacunación, evidenciamos una clara diferencia, siendo significativamente baja despues de la inmunización para todas las edades. No se han reportado muertes durante las últimas semanas, hasta la SE 48.
	
	
	\begin{figure}[h]
		\caption{Tasa de Mortalidad por COVID-19 por Grupo Etario hasta la SE 48-2022.}
		\label{fig:mortalidad_grupo_edad}
		\centering
		\begin{subfigure}[b]{0.45\textwidth}
			\centering
			\includegraphics[width=\textwidth]{../figuras/mortalidad_edad_80.pdf}
			\caption{Más de 80 años}
			%\label{fig:}
		\end{subfigure}
		\hfill
		\begin{subfigure}[b]{0.45\textwidth}
			\centering
			\includegraphics[width=\textwidth]{../figuras/mortalidad_edad_70.pdf}
			\caption{70 a 79 años}
			%\label{fig:70 a 79 años}
		\end{subfigure}
		
		\vspace{10mm}
		\begin{subfigure}[b]{0.45\textwidth}
			\centering
			\includegraphics[width=\textwidth]{../figuras/mortalidad_edad_60.pdf}
			\caption{60 a 69 años}
			%\label{fig:60 a 69 años}
		\end{subfigure}
		\hfill
		\begin{subfigure}[b]{0.45\textwidth}
			\centering
			\includegraphics[width=\textwidth]{../figuras/mortalidad_edad_50.pdf}
			\caption{50 a 59 años}
			%\label{fig:50 a 59 años}
		\end{subfigure}
		
		\vspace{10mm}
		\begin{subfigure}[b]{0.45\textwidth}
			\centering
			\includegraphics[width=\textwidth]{../figuras/mortalidad_edad_40.pdf}
			\caption{40 a 49 años}
			%\label{fig:40 a 49 años}
		\end{subfigure}
		\hfill
		\begin{subfigure}[b]{0.45\textwidth}
			\centering
			\includegraphics[width=\textwidth]{../figuras/mortalidad_edad_30.pdf}
			\caption{30 a 39 años}
			%\label{fig:40 a 49 años}
		\end{subfigure}
	\end{figure}
	
	\begin{figure}[h]
		\caption{Tasa de Mortalidad por COVID-19 por Grupo Etario hasta la SE 48-2022.}
		\label{fig:mortalidad_grupo_edad_2}
		\centering
		\begin{subfigure}[b]{0.45\textwidth}
			\centering
			\includegraphics[width=\textwidth]{../figuras/mortalidad_edad_20.pdf}
			\caption{20 a 29 años}
			%\label{fig:40 a 49 años}
		\end{subfigure}
		
		\centering
		\begin{subfigure}[b]{0.45\textwidth}
			\centering
			\includegraphics[width=\textwidth]{../figuras/mortalidad_edad_10.pdf}
			\caption{10 a 19 años}
			%\label{fig:40 a 49 años}
		\end{subfigure}
		
		\vspace{10mm}
		\begin{subfigure}[b]{0.45\textwidth}
			\centering
			\includegraphics[width=\textwidth]{../figuras/mortalidad_edad_0.pdf}
			\caption{0 a 09 años}
			%\label{fig:40 a 49 años}
		\end{subfigure}
	\end{figure}
	\clearpage	
	\subsection*{Exceso de Muertes por Todas las Causas}
	\noindent La Figura \ref{fig:exceso_regional} muestra la tendencia del exceso de muertes por todas las causas del año 2022 con respecto al año 2020.  Para la SE 48 se evidencia un exceso negativo de -108 muertes, es decir que se reporta 108 muertes menos en comparación con la SE 48 del año 2020.
	
	
	
	\begin{figure}[h]
		\caption{Exceso de Fallecidos por Todas las Causas en la Región Cusco hasta la SE 48-2022.}\label{fig:exceso_regional}
		\begin{center}
			\includegraphics[width=0.85\linewidth]{../figuras/exceso_region_2022.pdf}
		\end{center}
		{\footnotesize {Fuente de datos: SISCOVID, NOTICOVID.}}
	\end{figure}
	\clearpage
	
	\subsection*{Cobertura de Vacunación por COVID-19 en la Región Cusco, hasta la SE 48-2022.}
	\noindent La Figura \ref{fig:vacuna_edad} muestra la cobertura de vacunación (1ra, 2da y 3ra dosis), por grupo etario en la Región Cusco. Se evidencia que la cobertura de vacunación va incrementando en cada reporte, con un promedio de 54.3$\%$ para la tercera dosis. Siendo mayor en el grupo etario de 70 a 79 años (72.9$\%$) y menor en el grupo etario de 5-11 años (7.1$\%$), debido al comienzo tardío de la vacunación en este grupo.
	
	La Figura \ref{fig:Cobertura_Vacunacion_Provincias} muestra el avance de vacunación (2da y 3era dosis aplicadas) por provincia de la región Cusco. La provincia con mejor cobertura es la provincia de Cusco llegando casi al 80$\%$ de cobertura respecto a la 3ra dosis, seguida de la provincia de Anta, La Convención, Canchis y Urubamba; mientras que las provincias con menor cobertura continúan siendo las provincias de Canas, Espinar y Paucartambo. 
	\begin{figure}[h]
		\caption{Cobertura de Vacunación por Grupo Etario en la Región Cusco hasta la SE 48-2022. }\label{fig:vacuna_edad}
		\begin{center}
			\includegraphics[width=1\linewidth]{../figuras/vacunacion_grupo_edad_dosis.pdf}
		\end{center}
		{\footnotesize {Fuente de datos: SICOVAC, HIS-MINSA.}}
	\end{figure}
	\clearpage
	
	\begin{landscape}
		\begin{figure}[ht]
			\caption{Cobertura de segunda y tercera dosis aplicadas por provincia en la región Cusco-2022}\label{fig:Cobertura_Vacunacion_Provincias}
			\begin{center}
				\includegraphics[width=0.90\linewidth]{../sala_covid/../sala_nacional/Cobertura_Vacunacion_Provincias.jpg}
			\end{center}
			{\footnotesize {Fuente de datos: SICOVAC, HIS-MINSA.}}
		\end{figure}
	\end{landscape}
	
	
	%La Figura \ref{fig:cobertura_vacunaci_provincia}  muestra la cobertura de vacunación en cada una de las provincias de Cusco por grupo etario. Es preciso señalar que la provincia de Espinar tiene la cobertura más baja de la región, en los grupos etarios desde los 50 años en adelante.
	%
	%\begin{figure}[h]
	%	\caption{Cobertura de Vacunación por Provincia y por Grupo Etario en la Región Cusco, hasta la SE 51.}
	%	\label{fig:cobertura_vacunaci_provincia}
	%	\centering
	%	\begin{subfigure}[b]{0.45\textwidth}
		%		\centering
		%		\includegraphics[width=\textwidth]{../figuras/vacunacion_provincial_edad_1}
		%		\caption{ De 12 a 19 años}
		%		%\label{fig:}
		%	\end{subfigure}
	%	\hfill
	%	\begin{subfigure}[b]{0.45\textwidth}
		%		\centering
		%		\includegraphics[width=\textwidth]{../figuras/vacunacion_provincial_edad_2}
		%		\caption{De 20 a 29 años}
		%		%\label{fig:70 a 79 años}
		%	\end{subfigure}
	%	\begin{subfigure}[b]{0.45\textwidth}
		%		\centering
		%		\includegraphics[width=\textwidth]{../figuras/vacunacion_provincial_edad_3}
		%		\caption{De 30 a 39 años}
		%		%\label{fig:60 a 69 años}
		%	\end{subfigure}
	%	\hfill
	%	\begin{subfigure}[b]{0.45\textwidth}
		%		\centering
		%		\includegraphics[width=\textwidth]{../figuras/vacunacion_provincial_edad_4}
		%		\caption{De 40 a 49 años}
		%		%\label{fig:50 a 59 años}
		%	\end{subfigure}
	%	\begin{subfigure}[b]{0.45\textwidth}
		%		\centering
		%		\includegraphics[width=\textwidth]{../figuras/vacunacion_provincial_edad_5}
		%		\caption{De 50 a 59 años}
		%		%\label{fig:40 a 49 años}
		%	\end{subfigure}
	%	\hfill
	%	\begin{subfigure}[b]{0.45\textwidth}
		%		\centering
		%		\includegraphics[width=\textwidth]{../figuras/vacunacion_provincial_edad_6}
		%		\caption{De 60 a 69 años}
		%		%\label{fig:40 a 49 años}
		%	\end{subfigure}
	%\begin{subfigure}[b]{0.45\textwidth}
	%	\centering
	%	\includegraphics[width=\textwidth]{../figuras/vacunacion_provincial_edad_6}
	%	\caption{De 70 a 79 años}
	%	%\label{fig:40 a 49 años}
	%\end{subfigure}
	%\hfill
	%\begin{subfigure}[b]{0.45\textwidth}
	%	\centering
	%	\includegraphics[width=\textwidth]{../figuras/vacunacion_provincial_edad_7}
	%	\caption{Más de 80 años}
	%	%\label{fig:40 a 49 años}
	%\end{subfigure}
	%\end{figure}
	
	%\clearpage
	%\subsection*{Análisis de Supervivencia y Vacunas en Hospitalizados con COVID-19 de la Región Cusco}
	%\noindent Las curvas de sobrevida (Figura \ref{fig:supervivencia_2}) mostraron que los hospitalizados por COVID-19 con tres o dos dosis completas presentan menor probabilidad de muerte, a partir de su ingreso hasta el alta o defunción. Habiendo fallecido con tres dosis el 0.31$\%$, con dos dosis el 5.35$\%$ versus el 89.14$\%$ que no tuvo vacunación. 
	%Las curvas de sobrevida en hospitalizados por COVID-19 comenzaron a divergir en el 70$\%$ de eventos de muerte en día 18 después del ingreso (long rank test <0,0001).
	%
	%\begin{figure}[h]
	%	\caption{Defunciones en vacunados durante la hospitalización en la Región Cusco hasta la SE 07-2022.}\label{fig:supervivencia_2}
	%	\begin{center}
		%		\includegraphics[width=0.90\linewidth]{../figuras/supervivencia_1.png}
		%	\end{center}
	%	{\footnotesize {Fuente de datos: SICOVAC, Referencias y contrareferencias.}}
	%\end{figure}
	%
	%Las curvas de sobrevida (Figura \ref{fig:supervivencia_2}) mostraron que, conforme al grupo etario de un hospitalizado por COVID-19, se tuvo mayor sobrevida en los grupos de 0 a 17 años, siendo el grupo de mayores de 60 años el que tuvo menor sobrevida durante la hospitalización (long rank test <0,0001).
	%
	%\begin{figure}[h]
	%	\caption{Defunciones en vacunados durante la hospitalización conforme al grupo etario, Región Cusco hasta la SE 07-2022. }\label{fig:supervivencia_2}
	%	\begin{center}
		%		\includegraphics[width=0.90\linewidth]{../figuras/supervivencia_2.png}
		%	\end{center}
	%	{\footnotesize {Fuente de datos: SICOVAC, Referencias y contrareferencias}}
	%\end{figure}
	%
	%Para la evaluación de muerte relacionada con COVID-19 a partir de la última dosis de vacunación (Figura \ref{fig:supervivencia_4}), se verificó diferencias entre la aplicación de dosis incompleta (1 dosis) versus dosis completas (2 dosis), siendo muy escaso el número de hospitalizados con 3 dosis. Se observó divergencia en la presentación de muerte en el 88.5$\%$ alrededor del día 14 después de la última dosis de vacunación para las personas que presentaron dosis incompletas y completas. (long rank test <0.001)
	
	%\begin{figure}[h]
	%	\caption{Defunciones en vacunados durante la hospitalización a partir de la última dosis de vacunación, Región Cusco hasta la SE 07-2022. }\label{fig:supervivencia_4}
	%	\begin{center}
		%		\includegraphics[width=0.90\linewidth]{../figuras/supervivencia_4.png}
		%	\end{center}
	%	{\footnotesize {Fuente de datos: SICOVAC, Referencias y contrareferencias}}
	%\end{figure}
	
	Las figuras \ref{fig:covertura_vacunación_grupo etario_provincias}, \ref{fig:covertura_vacunación_grupo etario_provincias_2}, \ref{fig:covertura_vacunación_grupo etario_provincias_3}, \ref{fig:covertura_vacunación_grupo etario_provincias_4},
	muestran la cobertura de vacunación por grupo etario, en cada una de las 13 provincias de nuestra región Cusco. Se evidencia que el mayor porcentaje de vacunados con las 3 dosis se mantiene en el grupo etario de 70-79 años, en las provincias de Acomayo (87.9$\%$), Anta (74.1$\%$), Calca (72.5$\%$), Canas (67.2$\%$), Canchis (74.5$\%$), Chumbivilcas (60.8$\%$), La Convención (67.2$\%$), Paruro (60.9$\%$), Paucartambo (65.3$\%$), Quispicanchis (65.6$\%$) y Urubamba (66.9$\%$); mientras que en la provincia de Espinar la edad de 30-39 años es la que cuenta con mayor porcentaje de vacunación con las 3 dosis (41.8$\%$), sin embargo, este valor está por debajo del promedio de la región. En la provincia de Cusco el mayor porcentaje de vacunados con la 3ra dosis, se encuentran en los grupos etarios de 18-29 años y 60-69 años (93.3$\%$ y 90.1$\%$ respectivamente).
	
	\begin{figure}[h]
		\caption{Cobertura de vacunación COVID-19 por grupo etario en las 13 provincias de la región Cusco hasta la SE 48-2022.}
		\label{fig:covertura_vacunación_grupo etario_provincias}
		\centering
		\begin{subfigure}[b]{0.65\textwidth}
			\centering
			\includegraphics[width=\textwidth]{../figuras/vacunacion__provincias_1.pdf}
			\caption{Acomayo}
			%\label{fig:Acomayo}
		\end{subfigure}
		
		\vspace{5mm}
		\begin{subfigure}[b]{0.65\textwidth}
			\centering
			\includegraphics[width=\textwidth]{../figuras/vacunacion__provincias_2.pdf}
			\caption{Anta}
			%\label{fig:Anta}
		\end{subfigure}
	\end{figure}
	
	\begin{figure}[h]
		\caption{Cobertura de vacunación COVID-19 por grupo etario en las 13 provincias de la región Cusco hasta la SE 48-2022.}
		\label{fig:covertura_vacunación_grupo etario_provincias_2}
		\centering	
		\begin{subfigure}[b]{0.63\textwidth}
			\centering
			\includegraphics[width=\textwidth]{../figuras/vacunacion__provincias_3.pdf}
			\caption{Calca}
			%\label{fig:Calca}
		\end{subfigure}
		
		\vspace{3mm}
		\begin{subfigure}[b]{0.63\textwidth}
			\centering
			\includegraphics[width=\textwidth]{../figuras/vacunacion__provincias_4.pdf}
			\caption{Canas}
			%\label{fig:Canas}
		\end{subfigure}
		
		\vspace{3mm}
		\begin{subfigure}[b]{0.63\textwidth}
			\centering
			\includegraphics[width=\textwidth]{../figuras/vacunacion__provincias_5.pdf}
			\caption{Canchis}
			%\label{fig:Canchis}
		\end{subfigure}
		
		\vspace{3mm}
		\begin{subfigure}[b]{0.63\textwidth}
			\centering
			\includegraphics[width=\textwidth]{../figuras/vacunacion__provincias_6.pdf}
			\caption{Chumbivilcas}
			%\label{fig:Chumbivilcas}
		\end{subfigure}
	\end{figure}
	
	\begin{figure}[h]
		\caption{Cobertura de vacunación COVID-19 por grupo etario en las 13 provincias de la región Cusco hasta la SE 48-2022.}
		\label{fig:covertura_vacunación_grupo etario_provincias_3}
		\centering
		\begin{subfigure}[b]{0.63\textwidth}
			\centering
			\includegraphics[width=\textwidth]{../figuras/vacunacion__provincias_7.pdf}
			\caption{Cusco}
			%\label{fig:Cusco}
		\end{subfigure}
		
		\vspace{3mm}
		\begin{subfigure}[b]{0.63\textwidth}
			\centering
			\includegraphics[width=\textwidth]{../figuras/vacunacion__provincias_8.pdf}
			\caption{Espinar}
			%\label{fig:Espinar}
		\end{subfigure}
		
		\vspace{3mm}
		\begin{subfigure}[b]{0.63\textwidth}
			\centering
			\includegraphics[width=\textwidth]{../figuras/vacunacion__provincias_9.pdf}
			\caption{La Convención}
			%\label{fig:La Convención}
		\end{subfigure}
		
		\vspace{3mm}
		\begin{subfigure}[b]{0.63\textwidth}
			\centering
			\includegraphics[width=\textwidth]{../figuras/vacunacion__provincias_10.pdf}
			\caption{Paruro}
			%\label{fig:Paruro}
		\end{subfigure}
	\end{figure}
	
	\begin{figure}[h]
		\caption{Cobertura de vacunación COVID-19 por grupo etario en las 13 provincias de la región Cusco hasta la SE 48-2022.}
		\label{fig:covertura_vacunación_grupo etario_provincias_4}
		\centering
		\begin{subfigure}[b]{0.65\textwidth}
			\centering
			\includegraphics[width=\textwidth]{../figuras/vacunacion__provincias_11.pdf}
			\caption{Paucartambo}
			%\label{fig:Paucartambo}
		\end{subfigure}
		
		\vspace{5mm}
		\begin{subfigure}[b]{0.65\textwidth}
			\centering
			\includegraphics[width=\textwidth]{../figuras/vacunacion__provincias_12.pdf}
			\caption{Quispicanchis}
			%\label{fig:Quispicanchis}
		\end{subfigure}
		
		\vspace{5mm}
		\begin{subfigure}[b]{0.65\textwidth}
			\centering
			\includegraphics[width=\textwidth]{../figuras/vacunacion__provincias_13.pdf}
			\caption{Urubamba}
			%\label{fig:Urubamba}
		\end{subfigure}
	\end{figure}
	
	\clearpage
\subsection*{Análisis de pacientes post vacunación COVID-19 en la región Cusco, hasta la SE 48-2022.}
\noindent  Del total de fallecidos (Figura \ref{fig:defunción_postvacunación}) el 89.6$\%$ fueron aquellos pacientes que no contaban con ninguna dosis, el 3.9$\%$ en aquellos que tenían una dosis, 5.1$\%$ pertenecieron a los que contaban con dos dosis y tan solo el 1.5$\%$ de muertes fueron en pacientes que contaban con tres dosis de vacunación.


\begin{figure}[h]
	\caption{Proporción de defunción en relación a la vacunación COVID-19 en la región Cusco, hasta la SE 48-2022.}\label{fig:defunción_postvacunación}
\begin{center}
		\includegraphics[width=0.85\linewidth]{../figuras/post_vacunas_def.png}
	\end{center}
	{\footnotesize {Fuente de datos: SISCOVAC, SISCOVID, NOTICOVID, SINADEF.}}
\end{figure}

	En la figura \ref{fig:altas_postvacunación} se observa que el número de altas hospitalarias es mayor (64.3$\%$) en el grupo de pacientes con tres dosis de la vacuna COVID-19. Así mismo, se evidencia que el mayor porcentaje de camas UCI (60$\%$) se encuentran ocupadas por aquellos pacientes que no poseen ninguna dosis de la vacuna (Figura \ref{fig:uci_postvacunación}).

\begin{figure}[h]
	\caption{Proporción de altas hospitalarias en relación a la vacunación COVID-19 en la región Cusco, hasta la SE 48-2022.}\label{fig:altas_postvacunación}
	\begin{center}
		\includegraphics[width=0.71\linewidth]{../figuras/post_vacunas_altmed.png}
	\end{center}
	{\footnotesize {Fuente de datos: SISCOVAC, SISCOVID, NOTICOVID.}}
\end{figure}

\begin{figure}[h]
	\caption{Proporción de pacientes hospitalizados en el servicio de UCI y no UCI, en relación a la vacunación COVID-19 en la región Cusco, hasta la SE 48-2022.}\label{fig:uci_postvacunación}
	\begin{center}
		\includegraphics[width=0.71\linewidth]{../figuras/post_vacunas_uci.png}
	\end{center}
	{\footnotesize {Fuente de datos: SISCOVAC, SISCOVID, NOTICOVID.}}
\end{figure}
	
		
	\clearpage
	\subsection*{Ocupación de Camas}
	\noindent La disponibilidad y ocupación de camas UCI se ve resumida en la Figura \ref{fig:ocupacion_uci}, se evidencia que despues de una pequeña fase de meseta desde la SE 41, nos encontramos nuevamente con una evidente elevación de la curva. Actualmente, para la SE 48 se cuenta con 8 camas disponibles, con un porcentaje de ocupación de 25$\%$.
	
	\begin{figure}[h]
		\caption{Ocupación de Camas UCI COVID-19 en la Región Cusco hasta la SE 48-2022.}\label{fig:ocupacion_uci}
		\begin{center}
			\includegraphics[width=0.95\linewidth]{../figuras/uci.pdf}
		\end{center}
		{\footnotesize {Fuente de datos: REFERENCIAS Y CONTRAREFERENCIAS.}}
	\end{figure}
	\cleardoublepage
	
	En la Figura \ref{fig:ocupacion_3_nivel}, se plasma el porcentaje de ocupación y número de camas no UCI-COVID en el III nivel Hospitalario. Se observa que desde la SE 45, la curva se encuentra en ascenso, contando con 28 camas disponibles y 54$\%$ de porcentaje de ocupación de camas para la SE 48.
	


	\begin{figure}[htpb]
		\caption{Ocupación de Camas no UCI COVID-19 en el nivel III en la Región Cusco hasta la SE 48-2022.}\label{fig:ocupacion_3_nivel}
		\begin{center}
			\includegraphics[width=0.95\linewidth]{../figuras/nivel_3.pdf}
		\end{center}
		{\footnotesize {Fuente de datos: REFERENCIAS Y CONTRAREFERENCIAS.}}
	\end{figure}
	
	\clearpage
	
	En la Figura \ref{fig:ocupacion_2nivel}, se observa el número de camas disponibles y su porcentaje de ocupación en el II Nivel. Para la SE 48 tenemos 0 camas disponibles.
	
	\begin{figure}[h]
		\caption{Disponibilidad y Ocupación de Camas-COVID a Nivel de Hospitales del Nivel II en la Región Cusco hasta la SE 48-2022.}\label{fig:ocupacion_2nivel}
		\begin{center}
			\includegraphics[width=0.95\linewidth]{../figuras/nivel_2.pdf}
		\end{center}
		{\footnotesize {Fuente de datos: REFERENCIAS Y CONTRAREFERENCIAS.}}
	\end{figure}
	\clearpage
	\begin{landscape}
		
		\subsection*{Evaluación Provincial de Defunciones por COVID-19 para el año 2022.} 
		
		\begin{tabular}{lrrrrr}
	\rowcolor[HTML]{ECF4FF} 
	\textbf{Provincias}                   & \multicolumn{1}{l}{\cellcolor[HTML]{ECF4FF}\textbf{Población}} & \multicolumn{1}{l}{\cellcolor[HTML]{ECF4FF}\textbf{Pruebas Totales}} & \multicolumn{1}{l}{\cellcolor[HTML]{ECF4FF}\textbf{Defunciones}} & \multicolumn{1}{l}{\cellcolor[HTML]{ECF4FF}\textbf{Tasa de letalidad}} & \multicolumn{1}{l}{\cellcolor[HTML]{ECF4FF}\textbf{\begin{tabular}[c]{@{}l@{}}Tasa de mortalidad x \\   100.000 hab\end{tabular}}} \\
	\cellcolor[HTML]{FD6864}CANCHIS       & 105,049                                                        & 2,599                                                                & 19                                                             & 0.7\%                                                                  & 18.1                                                                                                                               \\
	\cellcolor[HTML]{FD6864}QUISPICANCHI  & 92,566                                                         & 1,192                                                                & 14                                                             & 1.2\%                                                                  & 15.1                                                                                                                               \\
	\cellcolor[HTML]{FFCE93}LA CONVENCIÓN & 185,793                                                        & 3,565                                                                & 22                                                             & 0.6\%                                                                  & 11.8                                                                                                                               \\
	\cellcolor[HTML]{FFCE93}CUSCO         & 463,656                                                        & 21,550                                                               & 50                                                             & 0.2\%                                                                  & 10.8                                                                                                                               \\
	\cellcolor[HTML]{FFFC9E}URUBAMBA      & 66,439                                                         & 1,230                                                                & 6                                                              & 0.5\%                                                                  & 9.0                                                                                                                                \\
	\cellcolor[HTML]{FFFC9E}PAUCARTAMBO   & 52,989                                                         & 465                                                                  & 4                                                              & 0.9\%                                                                  & 7.5                                                                                                                                \\
	\cellcolor[HTML]{FFFC9E}CHUMBIVILCAS  & 84,925                                                         & 905                                                                  & 6                                                              & 0.7\%                                                                  & 7.1                                                                                                                                \\
	\cellcolor[HTML]{FFFC9E}ANTA          & 57,731                                                         & 732                                                                  & 4                                                              & 0.5\%                                                                  & 6.9                                                                                                                                \\
	\cellcolor[HTML]{FFFC9E}ESPINAR       & 71,304                                                         & 937                                                                  & 4                                                              & 0.4\%                                                                  & 5.6                                                                                                                                \\
	\cellcolor[HTML]{FFFC9E}CALCA         & 76,462                                                         & 713                                                                  & 4                                                              & 0.6\%                                                                  & 5.2                                                                                                                                \\
	\cellcolor[HTML]{FFFC9E}CANAS         & 40,420                                                         & 488                                                                  & 2                                                              & 0.4\%                                                                  & 4.9                                                                                                                                \\
	\cellcolor[HTML]{9AFF99}ACOMAYO       & 28,477                                                         & 273                                                                  & 1                                                              & 0.4\%                                                                  & 3.5                                                                                                                                \\
	\cellcolor[HTML]{9AFF99}PARURO        & 31,264                                                         & 224                                                                  & 1                                                              & 0.4\%                                                                  & 3.2                                                                                                                                \\
	&                                                                &                                                                      &                                                                &                                                                        &                                                                                                                                    \\
	\rowcolor[HTML]{ECF4FF} 
	\textbf{Totales generales}            & \textbf{1,357,075}                                             & \textbf{34,873}                                                      & \textbf{137}                                                   & \textbf{0,39\%}                                                        & \textbf{10.1}                                                                                                                     
\end{tabular}
		
		
		{\footnotesize Fuente de datos: NOTICOVID, SISCOVID, SINADEF. Actualizado a la SE 48-2022.}
		
		\noindent 
		
	\end{landscape}
	%---------------------------------------------------------------------------
	% CAPÍTULO: EVALUACIÓN DE PROVINCIAS
	%---------------------------------------------------------------------------
	
	%insertar el cover del capitulo
	\includepdf[pages={1}]{../editorial/5.pdf}
	\clearpage
	
	\section*{Evaluación de Priorización y riesgo para COVID-19 por provincias}
	\addcontentsline{toc}{chapter}{Evaluación para Provincias Priorizadas}
	\noindent La Figura \ref{fig:incidencia_provincias} muestra las tasas de incidencia acumulada por provincia desde el 01 de enero hasta el 05 de diciembre del 2022, ordenadas de mayor a menor. Se evidencia que la mayor tasa de incidencia acumulada es, como es de esperarse, la que corresponde a la provincia de Cusco (807.4 casos / 10 000 personas), seguida de la provincia de Canchis (355.4 casos/ 10 000 personas) y en tercer lugar La Convención (313.5 casos/ 10 000 personas).
	
	\begin{figure}[!htpb]
		\caption{Tasa de Incidencia Acumulada por Provincia en la Región Cusco, hasta el 05 de diciembre del 2022*. }\label{fig:incidencia_provincias}
		\begin{center}
			\includegraphics[width=0.60\linewidth]{../figuras/incidencia_provincial_2022.png}
		\end{center}
		{\footnotesize {
				Fuente de datos: SISCOVID, NOTICOVID.(*)Se considera como caso positivo sólo a pacientes con prueba molecular o antigénica positiva}}
	\end{figure}
	
	La Figura \ref{fig:mortalidad_ordenada} muestra a las provincias de la región ordenadas de mayor a menor según la tasa de mortalidad acumulada hasta la SE 48; siendo las provincias con mayor tasa de mortalidad;  Canchis (2.5 defunciones/ 10 000 hab), Quispicanchis (2,4 defunciones/ 10 000 hab), La Convención (2 defunciones/ 10 000 hab), Urubamba y Cusco (2 defunciones/ 10 000 hab), no habiendo una gran variación en las últimas semanas.
	
	\begin{figure}[h]
		\caption{Tasa de Mortalidad Acumulada por Provincia en la Región Cusco, hasta la SE 48-2022. }\label{fig:mortalidad_ordenada}
		\begin{center}
			\includegraphics[width=0.60\linewidth]{../figuras/mortalidad_provincial_2022.png}
		\end{center}
		{\footnotesize {Fuente de datos: SISCOVID, NOTICOVID.}}
	\end{figure}
	
	La Figura \ref{fig:incidencia_provincial} muestra la tendencia de la incidencia acumulada a través del año 2022. Podemos observar que la tasa provincial de incidencia acumulada se encuentra con muy poca variación en los últimos meses. Se evidencia también, como ya se ha mencionado, que Cusco tiene mayor tasa de incidencia acumulada.
	
	\begin{figure}[h]
		\caption{Tendencia Provincial de Incidencia acumulada de COVID-19 hasta la SE 48-2022. }\label{fig:incidencia_provincial}
		\begin{center}
			\includegraphics[width=0.60\linewidth]{../figuras/incidencia_provincial_acumulada_2022.pdf}
		\end{center}
		{\footnotesize {Fuente de datos: NOTICOVID, SISCOVID.}}
	\end{figure}
	
	\clearpage
	
	\section*{Evaluación Provincial de 5 Indicadores}
	\noindent El objetivo de estas figuras es comparar a cada provincia consigo misma de acuerdo a su historia en la primera ola (en el año 2020). Se evaluaron los siguientes indicadores: incidencia (tomando en cuenta pruebas moleculares y antigénicas), tasa de mortalidad, tasa de positividad por prueba molecular, tasa de positividad por prueba antigénica, y exceso de defunciones para cada provincia.
	
	\subsection*{Provincia de Acomayo}
	\noindent La Figura \ref{fig:inc_mort_acomayo} muestra la tasa de incidencia y mortalidad de la provincia de Acomayo, se evidencia que posterior a la SE 33 tenemos escasos reportes de casos y ningún caso de muerte. Si comparamos con el año 2020, vemos que la curva de las semanas epidemiológicas homónimas se encuentran con un pequeño pico en la tasa de mortalidad y la tasa de incidencia. La tasa de positividad (Figura \ref{fig:positividad_acomayo}) para las pruebas antigénicas, se encuentra con un pico posterior a la SE 45, sin embargo, no se ve registro de casos positivos en cuanto a las pruebas por PCR. 
		
	La Figura \ref{fig:exceso_acomayo} muestra el exceso de defunciones para la SE 48, con un exceso de -3 muertes, significa que tenemos 3 muertes menos en comparación con la misma SE del año 2020. Así mismo, si comparamos con el año 2021, tenemos también menos cantidad de muertes en la misma SE del presente año.
	\begin{figure}[h]
		\caption{Tasa de Incidencia y Mortalidad Comparativa en la Provincia de Acomayo hasta la SE 48-2022.}\label{fig:inc_mort_acomayo}
		\begin{center}
			\includegraphics[width=0.70\linewidth]{../figuras/incidencia_mortalidad_20_21_1.png}
		\end{center}
		{\footnotesize {Fuente de datos: NOTICOVID, SISCOVID, SINADEF.}}
	\end{figure}
	
	\begin{figure}[h]
		\caption{Tasa de Positividad de Prueba Molecular y Antigénica Comparativa en la Provincia de Acomayo hasta la SE 48-2022. }\label{fig:positividad_acomayo}
		\begin{center}
			\includegraphics[width=0.7\linewidth]{../figuras/positividad_20_21_1.png}
		\end{center}
		{\footnotesize {Fuente de datos: NOTICOVID, SISCOVID.}}
	\end{figure}
	
	\begin{figure}[h]
		\caption{Exceso de Defunciones Comparativo en la Provincia de Acomayo hasta la SE 48-2022.}\label{fig:exceso_acomayo}
		\begin{center}
			\includegraphics[width=0.7\linewidth]{../figuras/exceso_1.pdf}
		\end{center}
		{\footnotesize {Fuente de datos: SINADEF.}}
	\end{figure}
	
	% Anta
	\clearpage
	
	\subsection*{Provincia de Anta}
	\noindent En la Figura \ref{fig:inc_mort_anta} se observa una tasa de mortalidad con valores cercanos a cero y la tasa de incidencia se encuentra con una ligera elevación, para la SE 48. Si lo comparamos con el año 2020, vemos que en la SE homónima se encuentran aún con algunos casos positivos y casos de muerte.
	\noindent La Figura \ref{fig:positividad_anta} muestra un incremento tanto en la tasa de positividad de pruebas antigénicas como de las pruebas moleculares, en la SE 48. Se registró una  curva similar en la SE homónima del año 2020.
	
	En la Figura \ref{fig:exceso_anta} se aprecia el exceso de defunciones para la SE 48, siendo -6 (exceso negativo); es decir, tenemos 6 muertes menos con respecto a la SE 48 del año 2020. Si observamos el gráfico del 2021, podemos observar que  presentamos mucho menos muertes en el presente año en comparación con la SE homónima del 2021.
	
	\begin{figure}[h]
		\caption{Tasa de Incidencia y Mortalidad Comparativa en la Provincia de Anta hasta la SE 48-2022.}\label{fig:inc_mort_anta}
		\begin{center}
			\includegraphics[width=0.85\linewidth]{../figuras/incidencia_mortalidad_20_21_2.png}
		\end{center}
		{\footnotesize {Fuente de datos: NOTICOVID, SISCOVID, SINADEF.}}
	\end{figure}
	
	\begin{figure}[h]
		\caption{Tasa de Positividad de Prueba Molecular y Antigénica Comparativa en la Provincia de Anta hasta la SE 48-2022.}\label{fig:positividad_anta}
		\begin{center}
			\includegraphics[width=0.7\linewidth]{../figuras/positividad_20_21_2.png}
		\end{center}
		{\footnotesize {Fuente de datos: NOTICOVID, SISCOVID.}}
	\end{figure}
	
	\begin{figure}[h]
		\caption{Exceso de Defunciones Comparativo en la Provincia de Anta hasta la SE 48-2022.}\label{fig:exceso_anta}
		\begin{center}
			\includegraphics[width=0.7\linewidth]{../figuras/exceso_2.pdf}
		\end{center}
		{\footnotesize {Fuente de datos: SINADEF.}}
	\end{figure}
	
	% Calca
	\clearpage
	
	\subsection*{Provincia de Calca}
	\noindent La Figura \ref{fig:inc_mort_calca} se evidencia que la tasa de mortalidad e incidencia en las últimas semanas tienen valores cercanos a cero.  Comparando con el 2020 se observa varios casos y algunas muertes en la misma SE. La tasa de positividad de pruebas antigénicas (Figura \ref{fig:positividad_calca}) muestra un aumento desde la SE 45.
	
	La Figura \ref{fig:exceso_calca} muestra el exceso de defunciones para la SE 48, siendo de -6 defunciones en comparación al 2020. Así también, se puede apreciar que tenemos significativamente menor número de defunciones en comparación con el año 2021, para la misma SE.
	
	\begin{figure}[h]
		\caption{Tasa de Incidencia y Mortalidad Comparativa en la Provincia de Calca hasta la SE 48-2022.}\label{fig:inc_mort_calca}
		\begin{center}
			\includegraphics[width=0.85\linewidth]{../figuras/incidencia_mortalidad_20_21_3.png}
		\end{center}
		{\footnotesize {Fuente de datos: NOTICOVID, SISCOVID, SINADEF.}}
	\end{figure}
	
	\begin{figure}[h]
		\caption{Tasa de Positividad de Prueba Molecular y Antigénica Comparativa en la Provincia de Calca hasta la SE 48-2022.}\label{fig:positividad_calca}
		\begin{center}
			\includegraphics[width=0.7\linewidth]{../figuras/positividad_20_21_3.png}
		\end{center}
		{\footnotesize {Fuente de datos: NOTICOVID, SISCOVID.}}
	\end{figure}
	
	\begin{figure}[h]
		\caption{Exceso de Defunciones Comparativo en la Provincia de Calca hasta la SE 48-2022.}\label{fig:exceso_calca}
		\begin{center}
			\includegraphics[width=0.7\linewidth]{../figuras/exceso_3.pdf}
		\end{center}
		{\footnotesize {Fuente de datos: SINADEF.}}
	\end{figure}
	
	% Canas
	\clearpage
	
	\subsection*{Provincia de Canas}
	\noindent Las figuras de abajo (Figura \ref{fig:inc_mort_canas}, \ref{fig:positividad_canas}) muestran el comportamiento de la tasa de incidencia, mortalidad y  positividad de  la provincia de Calca. Se puede evidenciar que durante las últimas semanas, no hemos tenido reportes de casos positivos ni de muertes. Se observa una distribución similar en cuanto a la tasa de mortalidad para la SE 48 del año 2020. La tasa de positividad por prueba antigénica se encuentra en ascenso desde la SE 45.

	
	En la Figura \ref{fig:exceso_canas} se muestra un exceso de muertes de -5 en relación al año 2020, lo cual indica que tenemos 5 muertes menos en este año en la SE 48 con respecto al 2020. Por otro lado, si comparamos la curva con el año 2021 observamos que el presente año registramos menos muertes en la misma SE.
	
	
	\begin{figure}[h]
		\caption{Tasa de Incidencia y Mortalidad Comparativa en la Provincia de Canas hasta la SE 48-2022.}\label{fig:inc_mort_canas}
		\begin{center}
			\includegraphics[width=0.85\linewidth]{../figuras/incidencia_mortalidad_20_21_4.pdf}
		\end{center}
		{\footnotesize {Fuente de datos: NOTICOVID, SISCOVID, SINADEF.}}
	\end{figure}
	
	\begin{figure}[h]
		\caption{Tasa de Positividad de Prueba Molecular y Antigénica Comparativa en la Provincia de Canas hasta la SE 48-2022.}\label{fig:positividad_canas}
		\begin{center}
			\includegraphics[width=0.7\linewidth]{../figuras/positividad_20_21_4.pdf}
		\end{center}
		{\footnotesize {Fuente de datos: NOTICOVID, SISCOVID.}}
	\end{figure}
	
	\begin{figure}[h]
		\caption{Exceso de Defunciones Comparativo en la Provincia de Canas hasta la SE 48-2022.}\label{fig:exceso_canas}
		\begin{center}
			\includegraphics[width=0.7\linewidth]{../figuras/exceso_4.pdf}
		\end{center}
		{\footnotesize {Fuente de datos: SINADEF.}}
	\end{figure}
	
	% Canchis
	\clearpage
	
	\subsection*{Provincia de Canchis}
	\noindent La Figura \ref{fig:inc_mort_canchis} muestra valores cercanos a cero, tanto en la tasa de incidencia como en la de mortalidad para la SE 48;  si comparamos con la SE homónima del año 2020, encontramos algunas muertes y varios casos reportados en ese año.
	\noindent La Figura \ref{fig:positividad_canchis} muestra el ascenso de la tasa de positividad de las pruebas antigénicas; sin embargo, vemos 0 casos positivos por PCR. 
	En la Figura \ref{fig:exceso_canchis} se evidencia exceso negativo de -8 defunciones con respecto al año 2020 para la SE 48, y también mucho menos casos de muerte en comparación con el año 2021 para la misma SE.
	
	\begin{figure}[h]
		\caption{Tasa de Incidencia y Mortalidad Comparativa en la Provincia de Canchis hasta la SE 48-2022.}\label{fig:inc_mort_canchis}
		\begin{center}
			\includegraphics[width=0.85\linewidth]{../figuras/incidencia_mortalidad_20_21_5.png}
		\end{center}
		{\footnotesize {Fuente de datos: NOTICOVID, SISCOVID, SINADEF.}}
	\end{figure}
	
	\begin{figure}[h]
		\caption{Tasa de Positividad de Prueba Molecular y Antigénica Comparativa en la Provincia de Canchis hasta la SE 48-2022.}\label{fig:positividad_canchis}
		\begin{center}
			\includegraphics[width=0.7\linewidth]{../figuras/positividad_20_21_5.png}
		\end{center}
		{\footnotesize {Fuente de datos: NOTICOVID, SISCOVID.}}
	\end{figure}
	
	\begin{figure}[h]
		\caption{Exceso de Defunciones Comparativo en la Provincia de Canchis hasta la SE 48-2022.}\label{fig:exceso_canchis}
		\begin{center}
			\includegraphics[width=0.7\linewidth]{../figuras/exceso_5.pdf}
		\end{center}
		{\footnotesize {Fuente de datos: SINADEF.}}
	\end{figure}
	
	\clearpage
	
	% Chumbivilcas
	\subsection*{Provincia de Chumbivilcas}
	\noindent En la Figura \ref{fig:inc_mort_chumbivilcas} se evidencia que para la SE 48, la tasa de incidencia y la tasa de mortalidad de la provincia de Chumbivilcas se encuentran con valores irrelevantes, cercanos a cero, desde la SE 40. Si comparamos ambas tasas con la SE 48 del 2020, vemos que la tasa de mortalidad se encuentra con una distribución similar, mientras que la tasa de incidencia se encuentra ligeramente mas elevada.
	
	\noindent La Figura \ref{fig:positividad_chumbivilcas} muestra un aumento de la tasa de positividad por pruebas antigénicas y moleculares desde la SE 45. Se observa que para la SE homónima del año 2020, la tasa de positividad por PCR se encuentra nula.
	En la Figura \ref{fig:exceso_chumbivilcas} se muestra el exceso de defunciones hasta la SE 48. Se evidencia un exceso negativo de -8 defunciones con respecto al año 2020. Por otro lado si comparamos con el año 2021, el presente año registra un número mucho menor de muertes para la misma SE.
	
	\begin{figure}[h]
		\caption{Tasa de Incidencia y Mortalidad Comparativa en la Provincia de Chumbivilcas hasta la SE 48-2022.}\label{fig:inc_mort_chumbivilcas}
		\begin{center}
			\includegraphics[width=0.85\linewidth]{../figuras/incidencia_mortalidad_20_21_6.png}
		\end{center}
		{\footnotesize {Fuente de datos: NOTICOVID, SISCOVID, SINADEF.}}
	\end{figure}
	
	\begin{figure}[h]
		\caption{Tasa de Positividad de Prueba Molecular y Antigénica Comparativa en la Provincia de Chumbivilcas hasta la SE 48-2022.}\label{fig:positividad_chumbivilcas}
		\begin{center}
			\includegraphics[width=0.7\linewidth]{../figuras/positividad_20_21_6.png}
		\end{center}
		{\footnotesize {Fuente de datos: NOTICOVID, SISCOVID.}}
	\end{figure}
	
	\begin{figure}[h]
		\caption{Exceso de Defunciones Comparativo en la Provincia de Chumbivilcas hasta la SE 48-2022.}\label{fig:exceso_chumbivilcas}
		\begin{center}
			\includegraphics[width=0.7\linewidth]{../figuras/exceso_6.pdf}
		\end{center}
		{\footnotesize {Fuente de datos: SINADEF.}}
	\end{figure}
	
	% Cusco
	\clearpage
	
	\subsection*{Provincia de Cusco}
	\noindent En la Figura \ref{fig:inc_mort_cusco} se evidencia que la tasa de mortalidad permanece con valores escasos desde la SE 37, sin embargo, la tasa de incidencia está sufriendo un ligero incremento en la curva desde la SE 45. Vemos lo opuesto en la figura de la izquierda, en el año 2020 se registraron varios casos positivos y defunciones para la misma semana epidemiológica.
	\noindent La  Figura \ref{fig:positividad_cusco} muestra un ascenso importante de la tasa de positividad para ambas pruebas desde la SE 45.
	
	En la Figura \ref{fig:exceso_cusco} se muestra el exceso de defunciones para la SE 48, donde indica un exceso negativo de -18 defunciones respecto al año 2020. También registramos menos casos de muerte en la SE 48 del presente año en comparación a la del año 2021.
	
	\begin{figure}[h]
		\caption{Tasa de Incidencia y Mortalidad Comparativa en la Provincia de Cusco hasta la SE 48-2022.}\label{fig:inc_mort_cusco}
		\begin{center}
			\includegraphics[width=0.85\linewidth]{../figuras/incidencia_mortalidad_20_21_7.png}
		\end{center}
		{\footnotesize {Fuente de datos: NOTICOVID, SISCOVID, SINADEF.}}
	\end{figure}
	
	\begin{figure}[h]
		\caption{Tasa de Positividad de Prueba Molecular y Antigénica Comparativa en la Provincia de Cusco hasta la SE 48-2022.}\label{fig:positividad_cusco}
		\begin{center}
			\includegraphics[width=0.7\linewidth]{../figuras/positividad_20_21_7.png}
		\end{center}
		{\footnotesize {Fuente de datos: NOTICOVID, SISCOVID.}}
	\end{figure}
	
	\begin{figure}[h]
		\caption{Exceso de Defunciones Comparativo en la Provincia de Cusco hasta la SE 48-2022.}\label{fig:exceso_cusco}
		\begin{center}
			\includegraphics[width=0.7\linewidth]{../figuras/exceso_7.pdf}
		\end{center}
		{\footnotesize {Fuente de datos: SINADEF.}}
	\end{figure}
	
	% Espinar
	\clearpage
	
	\subsection*{Provincia de Espinar}
	\noindent Las figuras (Figura \ref{fig:inc_mort_espinar}, \ref{fig:positividad_espinar}) muestran el comportamiento de la tasa de incidencia, mortalidad y positividad de la provincia de Espinar. Se evidencia la tendencia a cero de la tasa de incidencia y mortalidad desde la SE 37 hasta la actualidad. Se evidencia una curva similar en cuanto a la tasa de mortalidad para la SE homonima del año 2020; pero presenta una gran diferencia en cuanto a la tasa de incidencia, que presentó abundantes casos para la SE homónima del año 2020. La tasa de positividad de la prueba antigénica se encuentra con un ligero pico desde la SE 45.
	
	En la Figura \ref{fig:exceso_espinar} se muestra un exceso negativo de -2 defunciones respecto al año 2020 para la SE 48. Al realizar la comparación con el 2021, vemos que tenemos también menos cantidad de defunciones para la SE homónima.
	
	\begin{figure}[h]
		\caption{Tasa de Incidencia y Mortalidad Comparativa en la Provincia de Espinar hasta la SE 48-2022.}\label{fig:inc_mort_espinar}
		\begin{center}
			\includegraphics[width=0.85\linewidth]{../figuras/incidencia_mortalidad_20_21_8.png}
		\end{center}
		{\footnotesize {Fuente de datos: NOTICOVID, SISCOVID, SINADEF.}}
	\end{figure}
	
	\begin{figure}[h]
		\caption{Tasa de Positividad de Prueba Molecular y Antigénica Comparativa en la Provincia de Espinar hasta la SE 48-2022.}\label{fig:positividad_espinar}
		\begin{center}
			\includegraphics[width=0.7\linewidth]{../figuras/positividad_20_21_8.png}
		\end{center}
		{\footnotesize {Fuente de datos: NOTICOVID, SISCOVID.}}
	\end{figure}
	
	\begin{figure}[h]
		\caption{Exceso de Defunciones Comparativo en la Provincia de Espinar hasta la SE 48-2022.}\label{fig:exceso_espinar}
		\begin{center}
			\includegraphics[width=0.7\linewidth]{../figuras/exceso_8.pdf}
		\end{center}
		{\footnotesize {Fuente de datos: SINADEF.}}
	\end{figure}
	
	% La Convención
	\clearpage
	
	\subsection*{Provincia de La Convención}
	\noindent Las figuras inferiores (Figura \ref{fig:inc_mort_laconv}, \ref{fig:positividad_laconv}) muestran el comportamiento de la tasa de incidencia, mortalidad y positividad de la provincia de La Convención. Ambas tasas se encuentran con valores cercanos a cero desde la SE 37; mostrando así una gran diferencia con las semanas epidemiológicas homónimas del año 2020, donde nos encontrábamos con varios casos y varias defunciones reportadas. En cuanto a la tasa de positividad de las pruebas antigénicas, vemos un incremento en la curva desde la SE 45, no siendo así, para las pruebas por PCR que se encunetran en 0.
		
	En la Figura \ref{fig:exceso_laconv} muestra que hay exceso negativo de -22 defunciones respecto al año 2020 para la SE 48, también podemos afirmar que registramos mucho menor cantidad de muertes en comparación a la SE homónima del año 2021.    
	
	\begin{figure}[h]
		\caption{Tasa de Incidencia y Mortalidad Comparativa en la Provincia de La Convención hasta la SE 48-2022.}\label{fig:inc_mort_laconv}
		\begin{center}
			\includegraphics[width=0.85\linewidth]{../figuras/incidencia_mortalidad_20_21_9.png}
		\end{center}
		{\footnotesize {Fuente de datos: NOTICOVID, SISCOVID, SINADEF.}}
	\end{figure}
	
	\begin{figure}[h]
		\caption{Tasa de Positividad de Prueba Molecular y Antigénica Comparativa en la Provincia de La Convención hasta la SE 48-2022.}\label{fig:positividad_laconv}
		\begin{center}
			\includegraphics[width=0.7\linewidth]{../figuras/positividad_20_21_9.png}
		\end{center}
		{\footnotesize {Fuente de datos: NOTICOVID, SISCOVID.}}
	\end{figure}
	
	\begin{figure}[h]
		\caption{Exceso de Defunciones Comparativo en la Provincia de La Convención hasta la SE 48-2022.}\label{fig:exceso_laconv}
		\begin{center}
			\includegraphics[width=0.7\linewidth]{../figuras/exceso_9.pdf}
		\end{center}
		{\footnotesize {Fuente de datos: SINADEF.}}
	\end{figure}
	
	% Paruro
	\clearpage
	
	\subsection*{Provincia de Paruro}
	\noindent Las figuras de abajo (Figura \ref{fig:inc_mort_paruro}, \ref{fig:positividad_paruro}) muestran el comportamiento de la tasa de incidencia, mortalidad y positividad de la provincia de Paruro. Ambas tasas muestran valores cercanos a cero para la SE 48. La tasa de positividad por prueba antigénica y por prueba molecular se encuentran con valor nulo, desde la SE 42.
	
	En la Figura \ref{fig:exceso_paruro} muestra que hubo un exceso negativo de -5 muertes con respecto al año 2020 para la SE 48; asimismo registramos menos defunciones respecto a la misma SE del año 2021.
	
	\begin{figure}[h]
		\caption{Tasa de Incidencia y Mortalidad Comparativa en la Provincia de Paruro, hasta la SE 48-2022.}\label{fig:inc_mort_paruro}
		\begin{center}
			\includegraphics[width=0.85\linewidth]{../figuras/incidencia_mortalidad_20_21_10.png}
		\end{center}
		{\footnotesize {Fuente de datos: NOTICOVID, SISCOVID, SINADEF.}} 
	\end{figure}
	
	\begin{figure}[h]
		\caption{Tasa de Positividad de Prueba Molecular y Antigénica Comparativa en la Provincia de Paruro hasta la SE 48-2022.}\label{fig:positividad_paruro}
		\begin{center}
			\includegraphics[width=0.7\linewidth]{../figuras/positividad_20_21_10.png}
		\end{center}
		{\footnotesize {Fuente de datos: NOTICOVID, SISCOVID.}}
	\end{figure}
	
	\begin{figure}[h]
		\caption{Exceso de Defunciones Comparativo en la Provincia de Paruro hasta la SE 48-2022.}\label{fig:exceso_paruro}
		\begin{center}
			\includegraphics[width=0.7\linewidth]{../figuras/exceso_10.pdf}
		\end{center}
		{\footnotesize {Fuente de datos: SINADEF.}}
	\end{figure}
	
	
	% Paucartambo
	\clearpage
	
	\subsection*{Provincia de Paucartambo}
	\noindent Las figuras (Figura \ref{fig:inc_mort_paucartam}, \ref{fig:positividad_paucartam}) muestran el comportamiento de la tasa de incidencia, mortalidad y positividad de la provincia de Paucartambo. Se evidencia valores cercanos a cero de la tasa de incidencia y mortalidad para la SE 48; presenta una diferencia con el año 2020, donde se registraron algunos casos de muerte y la tasa de incidencia tuvo un pico en la SE 44 y 45. La tasa de positividad respecto a ambas pruebas se encuentra en un ascenso importante desde la SE 45, curva similar a la SE homónima del año 2020 para las pruebas por PCR.
	En la Figura \ref{fig:exceso_paucartam} se evidencia un exceso negativo de -4 defunciones respecto al año 2020 para la SE 48, si comparamos con el año 2021 para la misma SE, notamos que el presente año reportó mucho menos defunciones.
	\begin{figure}[h]
		\caption{Tasa de Incidencia y Mortalidad Comparativa en la Provincia de Paucartambo hasta la SE 48-2022.}\label{fig:inc_mort_paucartam}
		\begin{center}
			\includegraphics[width=0.85\linewidth]{../figuras/incidencia_mortalidad_20_21_11.png}
		\end{center}
		{\footnotesize {Fuente de datos: NOTICOVID, SISCOVID, SINADEF.}}
	\end{figure}
	
	\begin{figure}[h]
		\caption{Tasa de Positividad de Prueba Molecular y Antigénica Comparativa en la Provincia de Paucartambo hasta la SE 48-2022.}\label{fig:positividad_paucartam}
		\begin{center}
			\includegraphics[width=0.7\linewidth]{../figuras/positividad_20_21_11.png}
		\end{center}
		{\footnotesize {Fuente de datos: NOTICOVID, SISCOVID.}}
	\end{figure}
	
	\begin{figure}[h]
		\caption{Exceso de Defunciones Comparativo en la Provincia de Paucartambo hasta la SE 48-2022.}\label{fig:exceso_paucartam}
		\begin{center}
			\includegraphics[width=0.7\linewidth]{../figuras/exceso_11.pdf}
		\end{center}
		{\footnotesize {Fuente de datos: SINADEF.}}
	\end{figure}
	
	% Quispicanchi
	\clearpage
	
	\subsection*{Provincia de Quispicanchis}
	\noindent Las figuras (Figura \ref{fig:inc_mort_quisp}, \ref{fig:positividad_quisp}) muestran el comportamiento de la tasa de incidencia, mortalidad y positividad de la provincia de Quispicanchis. Tanto la tasa de incidencia como la tasa de mortalidad, se encuentran con valores cercanos a cero, sin embargo, en la última semana epidemiológica, observamos un leve incremento en la tasa de incidencia; notamos una clara diferencia con el año 2020, donde se reportaron varios casos positivos y defunciones. Con respecto a la tasa de positividad de ambas pruebas, se muestra un ascenso de la curva, desde la SE 45.
	
	En la Figura \ref{fig:exceso_quisp} se muestra un exceso negativo de -12 defunciones respecto al año 2020, para la SE 48. Vemos también que tenemos menos reportes de muerte en comparación al 2021. 
	
	\begin{figure}[h]
		\caption{Tasa de Incidencia y Mortalidad Comparativa en la Provincia de Quispicanchis hasta la SE 48-2022.}\label{fig:inc_mort_quisp}
		\begin{center}
			\includegraphics[width=0.85\linewidth]{../figuras/incidencia_mortalidad_20_21_12.png}
		\end{center}
		{\footnotesize {Fuente de datos: NOTICOVID, SISCOVID, SINADEF.}}
	\end{figure}
	
	\begin{figure}[h]
		\caption{Tasa de Positividad de Prueba Molecular y Antigénica Comparativa en la Provincia de Quispicanchi hasta la SE 48-2022.}\label{fig:positividad_quisp}
		\begin{center}
			\includegraphics[width=0.7\linewidth]{../figuras/positividad_20_21_12.png}
		\end{center}
		{\footnotesize {Fuente de datos: NOTICOVID, SISCOVID.}}
	\end{figure}
	
	\begin{figure}[h]
		\caption{Exceso de Defunciones Comparativo en la Provincia de Quispicanchis hasta la SE 48-2022.}\label{fig:exceso_quisp}
		\begin{center}
			\includegraphics[width=0.7\linewidth]{../figuras/exceso_12.pdf}
		\end{center}
		{\footnotesize {Fuente de datos: SINADEF.}}
	\end{figure}
	
	% Urubamba
	\clearpage
	
	\subsection*{Provincia de Urubamba}
	\noindent Las figuras (Figura \ref{fig:inc_urub}, \ref{fig:positividad_urub}) muestran el comportamiento de la tasa de incidencia, mortalidad y positividad de la provincia de Urubamba. Para la SE 48, se evidencia valores cercanos a 0 de la tasa de mortalidad, sin embargo, respecto a la tasa de incidencia observamos una pequeña elevación en la curva. Siendo muy diferente para el año 2020, donde ambas tasas registran curvas mas elevadas. La tasa de positividad de la provincia de Urubamba para las pruebas antigénicas muestra incremento desde la SE 45. Por otro lado, la tasa de positividad por pruebas moleculares se encuentra con valor nulo.
	
	En la Figura \ref{fig:exceso_urub} se muestra que hay exceso de -4 defunciones (exceso negativo) en comparación con el año 2020 para la SE 48. Si comparamos con el año 2021, observamos que reportamos mucho menos defunciones en la misma SE del presente año.
	
	\begin{figure}[h]
		\caption{Tasa de Incidencia y Mortalidad Comparativa en la Provincia de Urubamba hasta la SE 48-2022.}\label{fig:inc_urub}
		\begin{center}
			\includegraphics[width=0.85\linewidth]{../figuras/incidencia_mortalidad_20_21_13.png}
		\end{center}
		{\footnotesize {Fuente de datos: NOTICOVID, SISCOVID, SINADEF.}}
	\end{figure}
	
	\begin{figure}[h]
		\caption{Tasa de Positividad de Prueba Molecular y Antigénica Comparativa en la Provincia de Urubamba hasta la SE 48-2022.}\label{fig:positividad_urub}
		\begin{center}
			\includegraphics[width=0.7\linewidth]{../figuras/positividad_20_21_13.png}
		\end{center}
		{\footnotesize {Fuente de datos: NOTICOVID, SISCOVID.}}
	\end{figure}
	
	\begin{figure}[h]
		\caption{Exceso de Defunciones Comparativo en la Provincia de Urubamba hasta la SE 48-2022.}\label{fig:exceso_urub}
		\begin{center}
			\includegraphics[width=0.7\linewidth]{../figuras/exceso_13.pdf}
		\end{center}
		{\footnotesize {Fuente de datos: SINADEF.}}
	\end{figure}
	
	\clearpage
	%---------------------------------------------------------------------------
	% CAPÍTULO: VARIANTES DE COVID-19
	%---------------------------------------------------------------------------
	%insertar el cover del capitulo
	\includepdf[pages={1}]{../editorial/6.pdf}
	\clearpage
	
	\section* {Variantes de COVID-19 en la Región Cusco}
	\addcontentsline{toc}{chapter}{Variantes de COVID-19}
	\noindent La aparición de la variante ómicron generó las últimas olas de COVID-19 en el Perú debido a su gran transmisibilidad. Asimismo es importante resaltar la identificación de las nuevas subvariantes BA.4 y BA.5 de ómicron desde el mes de julio. En la Figura \ref{fig:variantes} y \ref{fig:subvariantes} se observa que en la región de Cusco, la variante ómicron continúa siendo la única prevalente en los últimos meses (100$\%$) y la más importante durante todo el año 2022; así como también, las subvariantes BA.4, BA.5, BQ.1, XBB.2, BF.7 y BA.2  son las principales causantes de los casos reportados actualmente, con predominio de la subvariante BA.5 para el mes de octubre y noviembre.
		 
	La vigilancia genómica viene siendo realizada desde el mes de junio del presente año por la GERESA-Cusco, a través del laboratorio referencial que procesa todas las muestras de secuenciamiento genómico para COVID-19.
	
	\begin{figure}[h]
		\caption{Prevalencia de las variantes de SARS Cov-2 aisladas en la región de Cusco, hasta Noviembre-2022. }\label{fig:variantes}
		\begin{center}
			\includegraphics[width=0.85\linewidth]{../figuras/variantes.pdf}
		\end{center}
		{\footnotesize {Fuente de datos: INS-NETLAB, UPCH, UNSAAC, Laboratorio referencial GERESA-Cusco}}
	\end{figure}

	\begin{figure}[h]
	\caption{Prevalencia de las variantes de SARS Cov-2 aisladas en la región de Cusco, hasta Noviembre-2022. }\label{fig:subvariantes}
	\begin{center}
		\includegraphics[width=0.85\linewidth]{../figuras/subvariantes.pdf}
	\end{center}
	{\footnotesize {Fuente de datos: INS-NETLAB, UPCH, UNSAAC, Laboratorio referencial GERESA-Cusco}}
\end{figure}
	
	Asimismo, la Figura \ref{fig:mapa_variantes} muestra las variantes de COVID-19 aisladas por provincias. Se evidencia la amplia distribución de la variante Ómicron en la región, reportándose casos por esta variante en el total de provincias (13) de la región Cusco, y en 2do lugar tenemos a la variante Lambda que abarca 11 de las 13 provincias de la región.
	
	\begin{figure}[h]
		\caption{Distribución provincial de las variantes de SARS-CoV-2 aisladas en la Región Cusco hasta la SE 48-2022.}
		\label{fig:mapa_variantes}
		\centering
		\begin{subfigure}[b]{0.40\textwidth}
			\centering
			\includegraphics[width=\textwidth]{../figuras/variantes_provincial_lambda.pdf}
			\caption{Variante Lambda}
			%\label{fig:}
		\end{subfigure}
		\hfill
		\begin{subfigure}[b]{0.40\textwidth}
			\centering
			\includegraphics[width=\textwidth]{../figuras/variantes_provincial_gamma.pdf}
			\caption{Variante Gamma}
			%\label{fig:70 a 79 años}
		\end{subfigure}
		
		\begin{subfigure}[b]{0.40\textwidth}
			\centering
			\includegraphics[width=\textwidth]{../figuras/variantes_provincial_delta.pdf}
			\caption{Variante Delta}
			%\label{fig:60 a 69 años}
		\end{subfigure}
		\vspace{0.5mm}
		\hspace{25mm}
		\begin{subfigure}[b]{0.40\textwidth}
			\centering
			\includegraphics[width=\textwidth]{../figuras/variantes_provincial_omicron.png}
			\caption{Variante Ómicron}
			%\label{fig:60 a 69 años}
		\end{subfigure}
	{\footnotesize {Fuente de datos: INS-NETLAB, UPCH, UNSAAC, Laboratorio referencial GERESA-Cusco}}
	\end{figure}
	
	\clearpage
	%---------------------------------------------------------------------------
	% CAPÍTULO: DEFUNCIONES CERO
	%-------------------------------------------
	
	%insertar el cover del capitulo
	
	\includepdf[pages={1}]{../editorial/7.pdf}
	\clearpage
	\section*{Semanas con Cero Defunciones por COVID-19 por Semana a Nivel Provincial}\addcontentsline{toc}{chapter}{Defunciones Cero}
	
	\noindent En la tabla inferior se muestran las provincias con cero defunciones reportadas (casillas en ámbar) por cada semana epidemiológica, desde la SE 41 hasta la SE 48. 
	Se observa que no se registró ninguna defunción en ninguna provincia durante el último mes.
	
	\begin{table}[h]		\caption{Defunciones Cero por COVID-19 a nivel Provincial hasta la SE 48-2022.}
		\resizebox{\textwidth}{!}{%
			\begin{tabular}{lccccccccc}
	\textbf{}              	  & \multicolumn{1}{l}{}                        & \multicolumn{1}{l}{}      & \multicolumn{1}{l}{}                         & \multicolumn{1}{l}{}                         & \multicolumn{1}{l}{}                         & \multicolumn{1}{l}{}                        & \multicolumn{1}{l}{}                         & \multicolumn{1}{l}{}                         & \multicolumn{1}{l}{}     \\
	\textbf{}                                                                               
	&\textbf{SE-03}
	&\textbf{SE-04}								&\textbf{SE-05}	
	&\textbf{SE-06}								&\textbf{SE-07}				&\textbf{SE-08}
	&\textbf{SE-09}								&\textbf{SE-10}
	&\textbf{SE-11}\\
	\textbf{}              	  	
	&\textbf{16ene-22ene}						&\textbf{23ene-29ene}						&\textbf{30ene-05feb}
	&\textbf{05feb-12feb}						&\textbf{13feb-19feb}
	&\textbf{20feb-26feb}						&\textbf{27feb-05mar}
	&\textbf{06mar-12mar}						&\textbf{13mar-19mar}\\
	\textbf{Acomayo}                        	
	&\cellcolor[HTML]{FCC46C}				    &\cellcolor[HTML]{FCC46C}
	&\cellcolor[HTML]{FCC46C}					&\cellcolor[HTML]{FCC46C}
	&\cellcolor[HTML]{FCC46C}					&1
	&\cellcolor[HTML]{FCC46C}					&\cellcolor[HTML]{FCC46C} 
	&\cellcolor[HTML]{FCC46C}\\
	\textbf{Anta}                                                          				
	&\cellcolor[HTML]{FCC46C}					&2 				
	&1											&\cellcolor[HTML]{FCC46C}					&2
	&\cellcolor[HTML]{FCC46C}					&\cellcolor[HTML]{FCC46C}					&1
	&\cellcolor[HTML]{FCC46C}\\
	\textbf{Calca}      				       								            &\cellcolor[HTML]{FCC46C}					&\cellcolor[HTML]{FCC46C}
	&1 											&1	
	&\cellcolor[HTML]{FCC46C}					&1											&\cellcolor[HTML]{FCC46C} 				&1											&1\\             			
	\textbf{Canas}                              		
	&\cellcolor[HTML]{FCC46C}					&1
	&1											&\cellcolor[HTML]{FCC46C}
	&1											&\cellcolor[HTML]{FCC46C}
	&\cellcolor[HTML]{FCC46C}					&\cellcolor[HTML]{FCC46C}
	&\cellcolor[HTML]{FCC46C} \\
	\textbf{Canchis}                             		
	&4											&3
	&2											&4
	&1											&1
	&\cellcolor[HTML]{FCC46C}					&3
	&1\\
	\textbf{Chumbivilcas}                      			
	&\cellcolor[HTML]{FCC46C} 					&1
	&\cellcolor[HTML]{FCC46C}					&3
	&4											&\cellcolor[HTML]{FCC46C}
	&\cellcolor[HTML]{FCC46C}					&\cellcolor[HTML]{FCC46C}
	&1\\
	\textbf{Cusco}                            										
	&11											&9 	
	&14 										&4
	&8											&3
	&1											&1
	&1\\
	\textbf{Espinar}       					             								
	 &\cellcolor[HTML]{FCC46C}
	&1											&1
	&\cellcolor[HTML]{FCC46C}					&1
	&2											&\cellcolor[HTML]{FCC46C}	
	&\cellcolor[HTML]{FCC46C} 					&\cellcolor[HTML]{FCC46C}\\
	\textbf{La Convención}                      					
	&3
	&4											&5
	&2											&3
	&1 											&1 
	&1											&\cellcolor[HTML]{FCC46C}\\
	\textbf{Paruro}                            					
	&1
	&\cellcolor[HTML]{FCC46C}					&\cellcolor[HTML]{FCC46C}
	&\cellcolor[HTML]{FCC46C} 					&1
	&1											&1
	&\cellcolor[HTML]{FCC46C}					&\cellcolor[HTML]{FCC46C}\\
	\textbf{Paucartambo}               		                       					
	&1											&1		
	&1											&\cellcolor[HTML]{FCC46C}
	&\cellcolor[HTML]{FCC46C}
	&\cellcolor[HTML]{FCC46C}					&\cellcolor[HTML]{FCC46C}
	&1											&\cellcolor[HTML]{FCC46C}\\
	\textbf{Quispicanchi}                                         	                 	
	&3											&5
	&4											&\cellcolor[HTML]{FCC46C}
	&1											&\cellcolor[HTML]{FCC46C}
	&1											&\cellcolor[HTML]{FCC46C}
	&\cellcolor[HTML]{FCC46C}\\
	\textbf{Urubamba}                                                          			
	&1
	&\cellcolor[HTML]{FCC46C}					&3
	&\cellcolor[HTML]{FCC46C}					&1
	&2											&\cellcolor[HTML]{FCC46C}
	&\cellcolor[HTML]{FCC46C}					&\cellcolor[HTML]{FCC46C}\\	
	&\multicolumn{1}{l}{}                       &\multicolumn{1}{l}{}            &\multicolumn{1}{l}{}                         
	&\multicolumn{1}{l}{}                       &\multicolumn{1}{l}{}            &\multicolumn{1}{l}{}                       &\multicolumn{1}{l}{}                       &\multicolumn{1}{l}{}            &\multicolumn{1}{l}{}    
\end{tabular}
		}
		{\footnotesize {Fuente de datos: SINADEF.}}
	\end{table}
	\pagebreak
	\section*{Resumen de Indicadores COVID-19}\addcontentsline{toc}{chapter}{Resumen de Indicadores Covid19}
	\begin{table}[h]		\caption{Tabla de Letalidad, Mortalidad e Incidencia a nivel Regional por Covid19, 2020 - 2022 (SE 48)}
		\resizebox{\textwidth}{!}{%
				\begin{tabular}{lccc|cccccc|}
		\cline{5-10}
		&
		\multicolumn{1}{l}{} &
		&
		&
		\multicolumn{6}{c|}{\cellcolor[HTML]{F2F2F2}} \\
		&
		\multicolumn{1}{l}{} &
		\multicolumn{1}{l}{} &
		\multicolumn{1}{l|}{} &
		\multicolumn{6}{c|}{\multirow{-2}{*}{\cellcolor[HTML]{F2F2F2}\textbf{Etapa de Vida}}} \\ \cline{5-10} 
		&
		\multicolumn{1}{l}{} &
		\multicolumn{1}{l}{} &
		\multicolumn{1}{l|}{} &
		\multicolumn{1}{c|}{\cellcolor[HTML]{F2F2F2}\textbf{Niño}} &
		\multicolumn{1}{l|}{\cellcolor[HTML]{F2F2F2}\textbf{Adolescente}} &
		\multicolumn{1}{l|}{\cellcolor[HTML]{F2F2F2}\textbf{Joven}} &
		\multicolumn{1}{l|}{\cellcolor[HTML]{F2F2F2}\textbf{Adulto}} &
		\multicolumn{1}{l|}{\cellcolor[HTML]{F2F2F2}\textbf{Adulto Mayor}} &
		\cellcolor[HTML]{F2F2F2}\textbf{Total} \\ \cline{2-10} 
		\multicolumn{1}{l|}{} &
		\multicolumn{1}{c|}{\cellcolor[HTML]{ECF4FF}} &
		\multicolumn{1}{c|}{\cellcolor[HTML]{ECF4FF}} &
		\cellcolor[HTML]{ECF4FF}\textbf{Tasa (\%)} &
		\multicolumn{1}{c|}{\cellcolor[HTML]{ECF4FF}0.4} &
		\multicolumn{1}{c|}{\cellcolor[HTML]{ECF4FF}0.049} &
		\multicolumn{1}{c|}{\cellcolor[HTML]{ECF4FF}0.12} &
		\multicolumn{1}{c|}{\cellcolor[HTML]{ECF4FF}0.57} &
		\multicolumn{1}{c|}{\cellcolor[HTML]{ECF4FF}7.9} &
		\cellcolor[HTML]{ECF4FF}1.3 \\ \cline{4-10} 
		\multicolumn{1}{l|}{} &
		\multicolumn{1}{c|}{\cellcolor[HTML]{ECF4FF}} &
		\multicolumn{1}{c|}{\multirow{-2}{*}{\cellcolor[HTML]{ECF4FF}\textbf{Letalidad}}} &
		\cellcolor[HTML]{ECF4FF} &
		\multicolumn{1}{c|}{\cellcolor[HTML]{ECF4FF}} &
		\multicolumn{1}{c|}{\cellcolor[HTML]{ECF4FF}} &
		\multicolumn{1}{c|}{\cellcolor[HTML]{ECF4FF}} &
		\multicolumn{1}{c|}{\cellcolor[HTML]{ECF4FF}} &
		\multicolumn{1}{c|}{\cellcolor[HTML]{ECF4FF}} &
		\cellcolor[HTML]{ECF4FF} \\ \cline{3-3}
		\multicolumn{1}{l|}{} &
		\multicolumn{1}{c|}{\cellcolor[HTML]{ECF4FF}} &
		\multicolumn{1}{c|}{\cellcolor[HTML]{ECF4FF}} &
		\multirow{-2}{*}{\cellcolor[HTML]{ECF4FF}\textbf{Defunciones}} &
		\multicolumn{1}{c|}{\multirow{-2}{*}{\cellcolor[HTML]{ECF4FF}07}} &
		\multicolumn{1}{c|}{\multirow{-2}{*}{\cellcolor[HTML]{ECF4FF}01}} &
		\multicolumn{1}{c|}{\multirow{-2}{*}{\cellcolor[HTML]{ECF4FF}29}} &
		\multicolumn{1}{c|}{\multirow{-2}{*}{\cellcolor[HTML]{ECF4FF}375}} &
		\multicolumn{1}{c|}{\multirow{-2}{*}{\cellcolor[HTML]{ECF4FF}973}} &
		\multirow{-2}{*}{\cellcolor[HTML]{ECF4FF}1385} \\ \cline{4-10} 
		\multicolumn{1}{l|}{} &
		\multicolumn{1}{c|}{\cellcolor[HTML]{ECF4FF}} &
		\multicolumn{1}{c|}{\multirow{-2}{*}{\cellcolor[HTML]{ECF4FF}\textbf{Mortalidad}}} &
		\cellcolor[HTML]{ECF4FF}\textbf{Tasa*} &
		\multicolumn{1}{c|}{\cellcolor[HTML]{ECF4FF}5.2} &
		\multicolumn{1}{c|}{\cellcolor[HTML]{ECF4FF}0.74} &
		\multicolumn{1}{c|}{\cellcolor[HTML]{ECF4FF}21} &
		\multicolumn{1}{c|}{\cellcolor[HTML]{ECF4FF}276} &
		\multicolumn{1}{c|}{\cellcolor[HTML]{ECF4FF}717} &
		\cellcolor[HTML]{ECF4FF}1020 \\ \cline{3-10} 
		\multicolumn{1}{l|}{} &
		\multicolumn{1}{c|}{\cellcolor[HTML]{ECF4FF}} &
		\multicolumn{1}{c|}{\cellcolor[HTML]{ECF4FF}} &
		\cellcolor[HTML]{ECF4FF}\textbf{Casos +} &
		\multicolumn{1}{c|}{\cellcolor[HTML]{ECF4FF}1749} &
		\multicolumn{1}{c|}{\cellcolor[HTML]{ECF4FF}2029} &
		\multicolumn{1}{c|}{\cellcolor[HTML]{ECF4FF}25091} &
		\multicolumn{1}{c|}{\cellcolor[HTML]{ECF4FF}66024} &
		\multicolumn{1}{c|}{\cellcolor[HTML]{ECF4FF}12255} &
		\cellcolor[HTML]{ECF4FF}107148 \\ \cline{4-10} 
		\multicolumn{1}{l|}{} &
		\multicolumn{1}{c|}{\multirow{-6}{*}{\cellcolor[HTML]{ECF4FF}\textbf{2020}}} &
		\multicolumn{1}{c|}{\multirow{-2}{*}{\cellcolor[HTML]{ECF4FF}\textbf{Incidencia}}} &
		\cellcolor[HTML]{ECF4FF}\textbf{Tasa*} &
		\multicolumn{1}{c|}{\cellcolor[HTML]{ECF4FF}1288} &
		\multicolumn{1}{c|}{\cellcolor[HTML]{ECF4FF}1495} &
		\multicolumn{1}{c|}{\cellcolor[HTML]{ECF4FF}18483} &
		\multicolumn{1}{c|}{\cellcolor[HTML]{ECF4FF}48637} &
		\multicolumn{1}{c|}{\cellcolor[HTML]{ECF4FF}9028} &
		\cellcolor[HTML]{ECF4FF}7.9 \\ \cline{2-10} 
		\multicolumn{1}{l|}{} &
		\multicolumn{1}{c|}{\cellcolor[HTML]{FFFFC7}} &
		\multicolumn{1}{c|}{\cellcolor[HTML]{FFFFC7}} &
		\cellcolor[HTML]{FFFFC7}\textbf{Tasa (\%)} &
		\multicolumn{1}{c|}{\cellcolor[HTML]{FFFFC7}0.94} &
		\multicolumn{1}{c|}{\cellcolor[HTML]{FFFFC7}0.087} &
		\multicolumn{1}{c|}{\cellcolor[HTML]{FFFFC7}0.13} &
		\multicolumn{1}{c|}{\cellcolor[HTML]{FFFFC7}1.9} &
		\multicolumn{1}{c|}{\cellcolor[HTML]{FFFFC7}19} &
		\cellcolor[HTML]{FFFFC7}3.8 \\ \cline{4-10} 
		\multicolumn{1}{l|}{} &
		\multicolumn{1}{c|}{\cellcolor[HTML]{FFFFC7}} &
		\multicolumn{1}{c|}{\multirow{-2}{*}{\cellcolor[HTML]{FFFFC7}\textbf{Letalidad}}} &
		\cellcolor[HTML]{FFFFC7} &
		\multicolumn{1}{c|}{\cellcolor[HTML]{FFFFC7}} &
		\multicolumn{1}{c|}{\cellcolor[HTML]{FFFFC7}} &
		\multicolumn{1}{c|}{\cellcolor[HTML]{FFFFC7}} &
		\multicolumn{1}{c|}{\cellcolor[HTML]{FFFFC7}} &
		\multicolumn{1}{c|}{\cellcolor[HTML]{FFFFC7}} &
		\cellcolor[HTML]{FFFFC7} \\ \cline{3-3}
		\multicolumn{1}{l|}{} &
		\multicolumn{1}{c|}{\cellcolor[HTML]{FFFFC7}} &
		\multicolumn{1}{c|}{\cellcolor[HTML]{FFFFC7}} &
		\multirow{-2}{*}{\cellcolor[HTML]{FFFFC7}\textbf{Defunciones}} &
		\multicolumn{1}{c|}{\multirow{-2}{*}{\cellcolor[HTML]{FFFFC7}11}} &
		\multicolumn{1}{c|}{\multirow{-2}{*}{\cellcolor[HTML]{FFFFC7}04}} &
		\multicolumn{1}{c|}{\multirow{-2}{*}{\cellcolor[HTML]{FFFFC7}25}} &
		\multicolumn{1}{c|}{\multirow{-2}{*}{\cellcolor[HTML]{FFFFC7}826}} &
		\multicolumn{1}{c|}{\multirow{-2}{*}{\cellcolor[HTML]{FFFFC7}2127}} &
		\multirow{-2}{*}{\cellcolor[HTML]{FFFFC7}2993} \\ \cline{4-10} 
		\multicolumn{1}{l|}{} &
		\multicolumn{1}{c|}{\cellcolor[HTML]{FFFFC7}} &
		\multicolumn{1}{c|}{\multirow{-2}{*}{\cellcolor[HTML]{FFFFC7}\textbf{Mortalidad}}} &
		\cellcolor[HTML]{FFFFC7}\textbf{Tasa*} &
		\multicolumn{1}{c|}{\cellcolor[HTML]{FFFFC7}8.1} &
		\multicolumn{1}{c|}{\cellcolor[HTML]{FFFFC7}2.9} &
		\multicolumn{1}{c|}{\cellcolor[HTML]{FFFFC7}18} &
		\multicolumn{1}{c|}{\cellcolor[HTML]{FFFFC7}608} &
		\multicolumn{1}{c|}{\cellcolor[HTML]{FFFFC7}1567} &
		\cellcolor[HTML]{FFFFC7}2205 \\ \cline{3-10} 
		\multicolumn{1}{l|}{} &
		\multicolumn{1}{c|}{\cellcolor[HTML]{FFFFC7}} &
		\multicolumn{1}{c|}{\cellcolor[HTML]{FFFFC7}} &
		\cellcolor[HTML]{FFFFC7}\textbf{Casos +} &
		\multicolumn{1}{c|}{\cellcolor[HTML]{FFFFC7}1173} &
		\multicolumn{1}{c|}{\cellcolor[HTML]{FFFFC7}4573} &
		\multicolumn{1}{c|}{\cellcolor[HTML]{FFFFC7}19526} &
		\multicolumn{1}{c|}{\cellcolor[HTML]{FFFFC7}43215} &
		\multicolumn{1}{c|}{\cellcolor[HTML]{FFFFC7}11129} &
		\cellcolor[HTML]{FFFFC7}79616 \\ \cline{4-10} 
		\multicolumn{1}{l|}{} &
		\multicolumn{1}{c|}{\multirow{-6}{*}{\cellcolor[HTML]{FFFFC7}\textbf{2021}}} &
		\multicolumn{1}{c|}{\multirow{-2}{*}{\cellcolor[HTML]{FFFFC7}\textbf{Incidencia}}} &
		\cellcolor[HTML]{FFFFC7}\textbf{Tasa*} &
		\multicolumn{1}{c|}{\cellcolor[HTML]{FFFFC7}864} &
		\multicolumn{1}{c|}{\cellcolor[HTML]{FFFFC7}3369} &
		\multicolumn{1}{c|}{\cellcolor[HTML]{FFFFC7}14304} &
		\multicolumn{1}{c|}{\cellcolor[HTML]{FFFFC7}31834} &
		\multicolumn{1}{c|}{\cellcolor[HTML]{FFFFC7}8198} &
		\cellcolor[HTML]{FFFFC7}58649 \\ \cline{2-10} 
		\multicolumn{1}{l|}{} &
		\multicolumn{1}{c|}{\cellcolor[HTML]{E2EFDA}} &
		\multicolumn{1}{c|}{\cellcolor[HTML]{E2EFDA}} &
		\cellcolor[HTML]{E2EFDA}\textbf{Tasa(\%)} &
		\multicolumn{1}{c|}{\cellcolor[HTML]{E2EFDA}0.35} &
		\multicolumn{1}{c|}{\cellcolor[HTML]{E2EFDA}0.099} &
		\multicolumn{1}{c|}{\cellcolor[HTML]{E2EFDA}0.028} &
		\multicolumn{1}{c|}{\cellcolor[HTML]{E2EFDA}0.13} &
		\multicolumn{1}{c|}{\cellcolor[HTML]{E2EFDA}3.7} &
		\cellcolor[HTML]{E2EFDA}0.48 \\ \cline{4-10} 
		\multicolumn{1}{l|}{} &
		\multicolumn{1}{c|}{\cellcolor[HTML]{E2EFDA}} &
		\multicolumn{1}{c|}{\multirow{-2}{*}{\cellcolor[HTML]{E2EFDA}\textbf{Letalidad}}} &
		\cellcolor[HTML]{E2EFDA} &
		\multicolumn{1}{c|}{\cellcolor[HTML]{E2EFDA}} &
		\multicolumn{1}{c|}{\cellcolor[HTML]{E2EFDA}} &
		\multicolumn{1}{c|}{\cellcolor[HTML]{E2EFDA}} &
		\multicolumn{1}{c|}{\cellcolor[HTML]{E2EFDA}} &
		\multicolumn{1}{c|}{\cellcolor[HTML]{E2EFDA}} &
		\cellcolor[HTML]{E2EFDA} \\ \cline{3-3}
		\multicolumn{1}{l|}{} &
		\multicolumn{1}{c|}{\cellcolor[HTML]{E2EFDA}} &
		\multicolumn{1}{c|}{\cellcolor[HTML]{E2EFDA}} &
		\multirow{-2}{*}{\cellcolor[HTML]{E2EFDA}\textbf{Defunciones}} &
		\multicolumn{1}{c|}{\multirow{-2}{*}{\cellcolor[HTML]{E2EFDA}07}} &
		\multicolumn{1}{c|}{\multirow{-2}{*}{\cellcolor[HTML]{E2EFDA}02}} &
		\multicolumn{1}{c|}{\multirow{-2}{*}{\cellcolor[HTML]{E2EFDA}04}} &
		\multicolumn{1}{c|}{\multirow{-2}{*}{\cellcolor[HTML]{E2EFDA}37}} &
		\multicolumn{1}{c|}{\multirow{-2}{*}{\cellcolor[HTML]{E2EFDA}193}} &
		\multirow{-2}{*}{\cellcolor[HTML]{E2EFDA}234} \\ \cline{4-10} 
		\multicolumn{1}{l|}{} &
		\multicolumn{1}{c|}{\cellcolor[HTML]{E2EFDA}} &
		\multicolumn{1}{c|}{\multirow{-2}{*}{\cellcolor[HTML]{E2EFDA}\textbf{Mortalidad}}} &
		\cellcolor[HTML]{E2EFDA}\textbf{Tasa *} &
		\multicolumn{1}{c|}{\cellcolor[HTML]{E2EFDA}5.2} &
		\multicolumn{1}{c|}{\cellcolor[HTML]{E2EFDA}1.5} &
		\multicolumn{1}{c|}{\cellcolor[HTML]{E2EFDA}2.9} &
		\multicolumn{1}{c|}{\cellcolor[HTML]{E2EFDA}27} &
		\multicolumn{1}{c|}{\cellcolor[HTML]{E2EFDA}142} &
		\cellcolor[HTML]{E2EFDA}172 \\ \cline{3-10} 
		\multicolumn{1}{l|}{} &
		\multicolumn{1}{c|}{\cellcolor[HTML]{E2EFDA}} &     
		\multicolumn{1}{c|}{\cellcolor[HTML]{E2EFDA}} &
		\cellcolor[HTML]{E2EFDA}\textbf{Casos +} &
		\multicolumn{1}{c|}{\cellcolor[HTML]{E2EFDA}2001} &
		\multicolumn{1}{c|}{\cellcolor[HTML]{E2EFDA}2025} &
		\multicolumn{1}{c|}{\cellcolor[HTML]{E2EFDA}14174} &
		\multicolumn{1}{c|}{\cellcolor[HTML]{E2EFDA}27487} &
		\multicolumn{1}{c|}{\cellcolor[HTML]{E2EFDA}5178} &
		\cellcolor[HTML]{E2EFDA}50865 \\ \cline{4-10} 
		\multicolumn{1}{l|}{} &
		\multicolumn{1}{c|}{\multirow{-6}{*}{\cellcolor[HTML]{E2EFDA}\textbf{2022}}} &
		\multicolumn{1}{c|}{\multirow{-2}{*}{\cellcolor[HTML]{E2EFDA}\textbf{Incidencia}}} &
		\cellcolor[HTML]{E2EFDA}\textbf{Tasa} &
		\multicolumn{1}{c|}{\cellcolor[HTML]{E2EFDA}1474} &
		\multicolumn{1}{c|}{\cellcolor[HTML]{E2EFDA}1492} &
		\multicolumn{1}{c|}{\cellcolor[HTML]{E2EFDA}10441} &
		\multicolumn{1}{c|}{\cellcolor[HTML]{E2EFDA}20248} &
		\multicolumn{1}{c|}{\cellcolor[HTML]{E2EFDA}3814} &
		\cellcolor[HTML]{E2EFDA}37470 \\ \cline{2-10} 
	\end{tabular}
		}
		{ \footnotesize {Fuente de datos: NOTICOVID, SISCOVID, SINADEF. 
				
				Tasa de mortalidad ajustada 1 000 000 de habitantes* - Tasa de incidencia ajustada 1 000 000 de habitantes*}}
	\end{table}
	
	\clearpage
	
	
	
	
	
	%---------------------------------------------------------------------------
	% CAPÍTULO: AGRADECIMIENTOS
	%--------------------------------------------------------------------------	\section*{Agradecimientos}
	\addcontentsline{toc}{chapter}{Agradecimientos}
	
	\centering
	{\large El presente Boletín Epidemiológico COVID-19 se ha elaborado gracias a la información y esfuerzo conjunto de los Equipos de Inteligencia Sanitaria de los Hospitales y Redes de la GERESA Cusco:
		
		\vspace{0.5cm}
		\noindent
		\begin{minipage}[t]{.45\textwidth}
			\centering
			Hospital Regional del Cusco \\
			M.S.P. Marina Ochoa Linares \vspace{0.5cm}\\
			Hospital Antonio Lorena \\
			Dr. Homero Dueñas \vspace{.5cm}\\
			Hospital Nacional Adolfo Guevara Velasco\\
			M.S.P. Lucio Velasquez Cuentas \vspace{.5cm}\\
			Red de Salud Norte \\
			Lic. Rosa Luz Quispe Sullcahuaman \vspace{0.5cm}\\
			Red de Salud Sur\\
			Lic. Luz Marina Bernable Villasante \vspace{0.5cm}\\	
		\end{minipage}
		\hfill
		\noindent
		\begin{minipage}[t]{.45\textwidth}
			\centering
			Red de Salud La Convención\\
			Lic. Gina Mejía Huacac\vspace{0.5cm}\\
			Red de Salud Chumbivilcas\\
			Lic. Eduarda Benito Calderón \vspace{.5cm}\\
			Red de Salud Canas Canchis Espinar\\
			Lic Gladys Martha Loaiza Ayala \vspace{.5cm}\\
			Red de Salud Kimbiri Pichari \\
			Blgo. Miguel Huayta Rivera\vspace{0.5cm}\\	
		\end{minipage}
		%---------------------------------------------------------------------------
		% CAPÍTULO: AGRADECIMIENTOS
		%---------------------------------------------------------------------------
		\chapter*{Diseño y Edición}
		\addcontentsline{toc}{chapter}{Diseño y Edición}
		\begin{center}
			
			% Como siempre, por orden alfabético del apellid0
			
			MSC. Fátima R. Concha Velasco
			
	MC. Lucero G. Contreras Masias
			
			Ing. Joel Wilfredo Sumerente Ayerbe
		\end{center}
		
		%insertar la última página
		\includepdf[pages={1}]{../editorial/pagina_final.pdf}
		\clearpage
		
	\end{document}
