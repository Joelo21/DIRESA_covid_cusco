%

%Note: LaTeX Beamer taken from the website: https://www.sharelatex.com/templates/presentations/conference-presentation
%Note: to find out the RGB code for dolphin: https://tex.stackexchange.com/questions/66465/how-to-get-actual-values-of-colour-theme-colours-in-beamer
%Note: to convert eps files to pdf: https://docupub.com/pdfconvert/

\documentclass[xcolor=table]{beamer}
\usepackage{appendixnumberbeamer}

\usepackage{subcaption}

% Change the margin
%\setbeamersize{text margin left = 2pt, text margin right = 2pt}

\setbeamertemplate{footline}[frame number]
\setbeamertemplate{headline}{}
% There are many different themes available for Beamer. A comprehensive
% list with examples is given here:
% http://deic.uab.es/~iblanes/beamer_gallery/index_by_theme.html
% You can uncomment the themes below if you would like to use a different
% one:
%\usetheme{AnnArbor}
%\usetheme{Antibes}
%\usetheme{Bergen}
%\usetheme{Berkeley}
%\usetheme{Berlin}
%\usetheme{Boadilla}
%\usetheme{boxes}
%\usetheme{CambridgeUS}
%\usetheme{Copenhagen}
%\usetheme{Darmstadt}
\usetheme{default}
%\usetheme{Frankfurt}
%\usetheme{Goettingen}
%\usetheme{Hannover}
%\usetheme{Ilmenau}
%\usetheme{JuanLesPins}
%\usetheme{Luebeck}
%\usetheme{Madrid}
%\usetheme{Malmoe}
%\usetheme{Marburg}
%\usetheme{Montpellier}
%\usetheme{PaloAlto}
%\usetheme{Pittsburgh}
%\usetheme{Rochester}
%\usetheme{Singapore}
%\usetheme{Szeged}
%\usetheme{Warsaw}
\makeatletter
\definecolor{beamer@blendedblue}{RGB}{164, 92, 92} % changed this

%\setbeamercolor{normal text}{fg=black,bg=white}
%\setbeamercolor{alerted text}{fg=red}
%\setbeamercolor{example text}{fg=green!50!black}

\setbeamercolor{structure}{fg=beamer@blendedblue}
%\useinnertheme{rounded}

%\renewcommand{\labelitemi}{$\bullet$}
%\renewcommand{\labelitemii}{$\cdot$}
%\renewcommand{\labelitemiii}{$\diamond$}
%\renewcommand{\labelitemiv}{$\ast$}

%colors
\usecolortheme{dolphin}
\usepackage{color}
\definecolor{mycolor1}{RGB}{184, 184, 184}
\definecolor{mycolor2}{RGB}{252, 196, 108}
\definecolor{mycolor3}{RGB}{164, 92, 92}
\definecolor{mycolor4}{RGB}{76, 60, 52}
\definecolor{mycolor5}{RGB}{164, 248, 208}
\definecolor{mycolor6}{RGB}{116, 112, 112}
\definecolor{mycolor7}{RGB}{116, 116, 52}

\usepackage{fnpct}

\usepackage{subcaption}

\usepackage{newpxtext}


%fonts
\usefonttheme{professionalfonts}

%Other packages
\usepackage{bm}
\setbeamercovered{transparent}
\usepackage[section]{placeins}
\usepackage{epstopdf}
\usepackage[table]{xcolor}
\usepackage{graphicx} %Useful to resize large tables into the frame using \resizebox 
\usepackage{grffile} %Documentation: http://ctan.sharelatex.com/tex-archive/macros/latex/contrib/oberdiek/grffile.pdf
\usepackage{marvosym}
\usepackage[style=british]{csquotes}
\def\signed #1{{\leavevmode\unskip\nobreak\hfil\penalty50\hskip2em
		\hbox{}\nobreak\hfill #1
		\parfillskip=0pt \finalhyphendemerits=0 \endgraf}}

\newsavebox\mybox
\newenvironment{aquote}[1]
{\savebox\mybox{#1}\begin{quote}}
	{\vspace*{1mm}\signed{\usebox\mybox}\end{quote}}

\usepackage{verse}
\newcommand{\attrib}[1]{
	\nopagebreak{\raggedleft\footnotesize #1\par}}

%for checkmarks
\usepackage{tikz}

%For equations
\usepackage{amsmath, amssymb}
\usepackage{bbm}

	%Tables
	\usepackage{tabularx}
	\usepackage{multirow}
	\usepackage{multicol} 
	\usepackage{booktabs}%\usepackage{booktabs, calc} %This is the package to use to have nice-looking tables. More documentation on the tables in LateX: https://www.tug.org/pracjourn/2007-1/mori/mori.pdf
	\usepackage{threeparttable}  
	\usepackage[table]{xcolor}
	\usepackage{colortbl}
	\usepackage{lmodern}
	\usepackage{booktabs}
	\usepackage{pgfplots}

% Count slides
\setcounter{MaxMatrixCols}{10}

\graphicspath{{../figuras/}}
%\graphicspath{{../tablas/}}

% Commands -
\newcommand{\extractRGB}[1]{\extractcolorspecs{#1}{\model}{\dolphin} \convertcolorspec{\model}{\dolphin}{RGB}\printcol \printcol}
\setbeamertemplate{caption}{\raggedright\insertcaption\par} %to prevent beamer from putting "figure" in front of a caption
\setbeamertemplate{navigation symbols}{}

%Code to create sections with title pages in Beamer slides
\AtBeginSection[]{
	\begin{frame}[plain]
		\vfill
		\centering
		\begin{beamercolorbox}[sep=8pt,center,shadow=true,rounded=true]{title}
			\usebeamerfont{title}\insertsectionhead\par%
		\end{beamercolorbox}
		\vfill
	\end{frame}
}

\title{Sala Situacional COVID-19, Región Cusco\footnote{Análisis con información de la región hasta el 17 de Diciembre del 2022}}
\author{Fátima Concha \& Joel Sumerente} 
\institute{Dirección de Epidemiología e Investigación \\ \textbf{\color{mycolor3}DEIS, GERESA - Cusco}}
\date{Semana Epidemiológica 50, 2022}

\begin{document}
	%------------------------------------------------------------------------------------------------------------------------------------------------------------------------------------------------------------------------------------------
	% TITLE PAGE 
	%------------------------------------------------------------------------------------------------------------------------------------------------------------------------------------------------------------------------------------------
	\setbeamercovered{invisible}
	\begin{frame}[plain]
		\titlepage
	\end{frame}

	%------------------------------------------------------------------------------------------------------------------------------------------------------------------------------------------------------------------------------------------
	% INTRODUCTION 
	%------------------------------------------------------------------------------------------------------------------------------------------------------------------------------------------------------------------------------------------
	
	\setcounter{subsection}{1}
	\begin{frame}[label=indice]
		\frametitle{Índice}
		\vspace{-.5cm}
		\begin{itemize}
			\item Indicadores epidemiológicos.
			\begin{itemize}
%				\item Nacional: Variantes, cobertura vacuna, mortalidad, RT, ocupación de camas UCI y no UCI a nivel regional. \hyperlink{epi_nacional}{\beamergotobutton{epi. nacional.}}
				\item Regional: Curva epidémica, incidencia, tasa de positividad, variantes ( \hyperlink{variantes}{\beamergotobutton{variantes}}), propagación, defunciones, RT por distrito, exceso de defunciones, letalidad, mortalidad, y mortalidad por grupos de edad e hitos de vacunación.  \hyperlink{epi_cusco}{\beamergotobutton{epi. cusco.}} 
			\end{itemize} 
			\item Cobertura de Vacunación ( \hyperlink{cobertura_vacuna}{\beamergotobutton{cobertura}}). 
			\item Indicadores de gestión hospitalaria.
			\begin{itemize}
				\item Ocupación camas UCI y no UCI a nivel regional y por hospital. \hyperlink{camas}{\beamergotobutton{camas hosp.}} 
			\end{itemize}
			\item Indicadores por provincias.
			\begin{itemize}
				\item Incidencia, tasa de mortalidad, tasa de letalidad, tasa de positividad PCR, AG, y exceso de mortalidad. \hyperlink{provincias}{\beamergotobutton{provincias}}
			\end{itemize}
			\item Resumen y recomendaciones. \hyperlink{Resumen}{\beamergotobutton{resumen}} \hyperlink{recomendaciones}{\beamergotobutton{recomendaciones}}
			\item Links útiles. \hyperlink{links}{\beamergotobutton{links}} \hfill \hyperlink{vacunas_90}{\beamergotobutton{apendice}}
		\end{itemize}
	\end{frame}
	
	%------------------------------------------------------------------------------------------------------------------------------------------------------------------------------------------------------------------------------------------
	% SECCIÓN 1: Indicadores Epidemiológicos, Nivel Nacional
	%------------------------------------------------------------------------------------------------------------------------------------------------------------------------------------------------------------------------------------------
%	\section{Indicadores Epidemiológicos, Nivel Nacional}
	
%	\begin{frame}[label=epi_nacional]
%	\frametitle{Tendencia de Variantes en Lima y la Sierra Selva Sur}
%	\vspace{-.5cm}
%	\begin{figure}
%		\centering
%		\begin{subfigure}[b]{0.40\textwidth}
%			\centering
%			\includegraphics[width=\textwidth]{../sala_nacional/variantes_lima.png}
%			\caption{Lima}
%			\label{fig:}
%		\end{subfigure}
%		\hfill
%		\begin{subfigure}[b]{0.42\textwidth}
%			\centering
%			\includegraphics[width=\textwidth]{../sala_nacional/variantes_sierra_sur.png}
%			\caption{Sierra Selva Sur}
%			\label{fig:70 a 79 años}
%		\end{subfigure}
%	\end{figure}
%	{\tiny Fuente: Lescano, Orellana, Fano, Pino, y Flores. Situación Epidemiológica de la COVID-19 al 26 de Febrero del 2022.\\} 
%	\vspace{0.01cm}
%	$\rightarrow$ 100\% ómicron en Lima y Callao en la SE05, 33 y 8 muestras respectivamente.\\
%	$\rightarrow$ Omicrón detectada en Lima: 3/90 muestras.\\
%	$\rightarrow$ Resultados de 2-3 semanas atrás. Vacíos en 10 regiones.\\	
%\end{frame}
%
%\begin{frame}[label=epi_nacional]
%	\frametitle{Casos nacionales bajan 36.7\%, cuarta caída fuerte?}
%	\vspace{.1cm}
%	\begin{figure}
%		\centering
%		\begin{subfigure}[b]{0.35\textwidth}
%			\centering
%			\includegraphics[width=\textwidth]{../sala_nacional/Casos_Nacionales_3ra_ola.jpg}
%			\label{fig:}
%		\end{subfigure}
%		\begin{tikzpicture}[overlay]
%		\draw[mycolor2, thick] (-2.05,2.73) circle [x radius=2.5cm, y radius=.15cm, rotate=0];
%		\end{tikzpicture}
%		\hfill
%		\begin{subfigure}[b]{0.55\textwidth}
%			\centering
%			\includegraphics[width=\textwidth]{../sala_nacional/Casos_Nacionales_3ra_ola_02.jpg}
%			\label{fig:70 a 79 años}
%		\end{subfigure}
%		\begin{tikzpicture}[overlay]
%		\draw[mycolor2, ultra thick] (-4.30,3.20) circle [x radius=.47cm, y radius=.47cm, rotate=0];
%		\end{tikzpicture}
%	\end{figure}
%	{\tiny Fuente: Lescano, Orellana, Fano, Pino, y Flores. Situación Epidemiológica de la COVID-19 al 19 de Febrero del 2022.\\} 
%	\vspace{0.01cm}
%	$\rightarrow$ Sólo sube Tumbes, 139\%, crece nueve semanas.\\
%	$\rightarrow$ La costa norte y Centro caen menos fuertemente.\\
%	$\rightarrow$ Todas las regiones menos Tumbes crecen >22\%.\\
%	$\rightarrow$ Casos caen masiva, simultanea y proporcionalmente en multiples regiones.\\
%\end{frame}
%
%\begin{frame}[label=epi_nacional]
%	\frametitle{Positividad antigénica cae 7.20\%, cuarta y fuerte baja}
%	\vspace{-.1cm}
%	\begin{figure}
%		\centering
%		\begin{subfigure}[b]{0.35\textwidth}
%			\centering
%			\includegraphics[width=\textwidth]{../sala_nacional/Positividad_Antigenica_3ra_ola.jpg}
%			\label{fig:}
%		\end{subfigure}
%		\begin{tikzpicture}[overlay]
%			\draw[mycolor2, thick] (-2.10,4.45) circle [x radius=2.4cm, y radius=.15cm, rotate=0];
%		\end{tikzpicture}
%		\hfill
%		\begin{subfigure}[b]{0.60\textwidth}
%			\centering
%			\includegraphics[width=\textwidth]{../sala_nacional/Porcentaje_Positivos_3ra_ola_regiones.jpg.png}
%			\label{fig:70 a 79 años}
%		\end{subfigure}
%		\begin{tikzpicture}[overlay]
%			\draw[mycolor2, ultra thick] (-4.70,3.60) circle [x radius=.50cm, y radius=.50cm, rotate=0];
%		\end{tikzpicture}
%	\end{figure}
%	{\tiny Fuente: Lescano, Orellana, Fano, Pino, y Flores. Situación Epidemiológica de la COVID-19 al 19 de Febrero del 2022.\\} 
%	\vspace{0.01cm}
%		$\rightarrow$ Todas las regiones bajan por al menos tres semanas.\\
%\end{frame}

%	\begin{frame}
%		\frametitle{Vacunación COVID 19 en el Perú}
%		\vspace{-.5cm}
%		\begin{center}
%			\includegraphics[width=1\linewidth]{../sala_nacional/vacunas_nacional.jpeg} 
%		\end{center}
%	
%	%	\begin{tikzpicture}[overlay]
%	%		\draw[mycolor5, very thick] (9.5,3.40) circle [x radius=3.0cm, y radius=.58cm, rotate=90];
%	%	\end{tikzpicture}
%		{\tiny Fuente:REUNIS (Repositorio Único Nacional de Información en Salud). Consultado el 27 de Agosto del 2022.\\} 
%		\vspace{0.01cm}
%	%	$\searrow$ Cusco pasa al puesto 15.
%	\end{frame}
%
%	\begin{frame}
%			\frametitle{Mortalidad por Regiones}
%			\vspace{-.2cm}
%			\begin{center}
%				\includegraphics[width=0.75\linewidth]{../sala_nacional/mortalidad_regional.png}
%			\end{center}
%			\begin{tikzpicture}[overlay]
%			\draw[mycolor2, thick] (3.5,4.55) circle [x radius=2.cm, y radius=.13cm, rotate=0];
%			\end{tikzpicture}
%			{\tiny Fuente: Lescano, Orellana, Fano, Pino, y Flores. Situación Epidemiológica de la COVID-19 al 27 de Agosto del 2022.\\}
%	\vspace{0.01cm}
%	$\nearrow$ Cusco se mantiene en el puesto 8, con $ 3,483 $ defunciones por COVID-19 por millón de habitantes. \\
%	\end{frame}
%
%	\begin{frame}
%		\frametitle{Defunciones Semanales por Regiones}
%		\vspace{-.2cm}
%		\begin{center}
%			\includegraphics[width=0.55\linewidth]{../sala_nacional/defunciones_regional.png}
%		\end{center}
%		\begin{tikzpicture}[overlay]
%			\draw[mycolor2, ultra thick] (5.2,3.34) circle [x radius=3.8cm, y radius=.1cm, 	rotate=0];
%		\end{tikzpicture}
%	{\tiny Fuente: Lescano, Orellana, Fano, Pino, y Flores. Situación Epidemiológica de la COVID-19 al 27 de Agosto del 2022.\\
%	\textbf{Riesgos}:\\
%				Resto del país sube 11.8\%. \\
%				Fuerte incremento en Ica. \\
%				Costa Norte, Sierra/Selva Centro y Selva Baja suben de 2-5 fallecidos. \\
%				Mayores de 70 años suben. \\
%				Apurímac con la mayor tasa de fallecidos (1.99). \\
%	\textbf{Menor Riesgo}: \\
%				Lima metropolitana disminuye luego de 4 semanas en alza. \\
%				Todas las DIRIS disminuyen o se estancan. \\
%				}
%	\vspace{0.01cm}
%	$\rightarrow$ {\color{mycolor5}Cusco}: Las defunciones representan un $3.80\% $ del pico de la segunda ola.\\
%	\end{frame}
%	
%	\begin{frame}
%		\frametitle{Número Reproductivo Efectivo por Regiones}
%		\vspace{-.5cm}
%		\begin{center}
%			\includegraphics[width=0.71\linewidth]{../sala_nacional/rt_regional.png}
%		\end{center}
%		\begin{tikzpicture}[overlay]
%		\draw[mycolor2, ultra thick] (5.35,3.47) circle [x radius=3.4cm, y radius=.18cm, 	rotate=0];
%		\end{tikzpicture}
%		{\tiny Fuente: CDC MINSA. Reporte de Vigilancia COVID-19, Perú 2022. Actualización: 2022-08-27.\\}
%		\vspace{0.01cm}
%		$\rightarrow$ El número reproductivo efectivo (RT) indica la dinámica de la enfermedad; si es menor a $ 1 $, existe control de la epidemia. \\
%	$\nearrow$ Cusco se encuentra a la posición 15 con un RT de $0.84$ $ [0.80-0.88]$.
%	\end{frame}
%
%  \begin{frame}
%	\frametitle{Ocupación de Camas UCI Comparativo por Regiones}
%	\vspace{-.2cm}
%		\begin{center}
%			\includegraphics[width=0.60\linewidth]{../sala_nacional/camas_regional.png}
%		\end{center}
%	\begin{tikzpicture}[overlay]
%	\draw[mycolor2, ultra thick] (4.08,3.75) circle [x radius=.50cm, y radius=.50cm, rotate=0];
%	\end{tikzpicture}
%	{\tiny Fuente: Lescano, Orellana, Fano, Pino, y Flores. Situación Epidemiológica de la COVID-19 al 27 de Agosto del 2022. \\
%		\textbf Perú: 873 camas libres, sin cambio 0\% -  Lima: 329 camas libres, baja 0.9\%. \\
%		\textbf >Suben*: \\
%		Costa: Piura, Lima región, Arequipa, Lambayeque, Tumbes y Ancash.\\
%		Sierra/Selva Norte: San Martin, Ucayali, Loreto, Cajamarca, y Amazonas.\\
%		Sierra Sur: Cusco, Ayacucho y Apurímac. - Sierra Centro: Junín.\\
%		\textbf >Siete regiones suben 2-3 semanas*:\\
%		Cajamarca sube 5.41\% en 2 sem.- Cusco sube 7.55\% en 2 sem. - Junín sube 13.81\% en 3 sem.\\
%		Lima región sube 13.37\% en 2 sem. - Loreto sube 23.08\% en 3 sem.\\
%		Piura sube 11.86 en 2 sem. - San Martin  sube 17.65 en 2 sem.\\ 
%		\textbf >Que media:\\
%		Ayacucho (50.7\%), Tumbes (48.2\%), Arequipa (47.2\%), Ancash (45.1\%), Apurímac (32.1\%), La Libertad (31.5\%), Cusco (27.3\%), Huánuco (23.5\%), Loreto (23.1\%), Lima región (20.4\%), Piura (19.9\%) y Callao (18.5\%).\\
%		} 
%		\vspace{0.01cm}
%		$\nearrow$ Nueve de estas 12 regiones suben
%	\end{frame}
%
%\begin{frame}
%	\frametitle{Ocupación de Camas UCI por Regiones}
%	\vspace{-.5cm}
%	\begin{center}
%		\includegraphics[width=0.75\linewidth]{../sala_nacional/uci_regional.png}
%	\end{center}
%	\begin{tikzpicture}[overlay]
%	\draw[mycolor2, thick] (5.6,2.57) circle [x radius=3.9cm, y radius=.15cm, rotate=0];
%	\end{tikzpicture}
%	{\tiny Fuente: Lescano, Orellana, Fano, Pino, y Flores. Situación Epidemiológica de la COVID-19 al 27 de Agosto del 2022.\\}
%	\vspace{0.1cm}
%	$\searrow$ Cusco se mantiene en la posición 21.
%\end{frame}
%	
%\begin{frame}
%	\frametitle{Ocupación de Camas No UCI por Regiones}
%	\vspace{-.5cm}
%	\begin{center}
%		\includegraphics[width=0.75\linewidth]{../sala_nacional/nouci_regional.png} 
%	\end{center}
%	\begin{tikzpicture}[overlay]
%	\draw[mycolor2, thick] (3.8,2.90) circle [x radius=2.8cm, y radius=.15cm, rotate=0];
%	\end{tikzpicture}
%	{\tiny Fuente: Lescano, Orellana, Fano, Pino, y Flores. Situación Epidemiológica de la COVID-19 al 27 de Agosto del 2022. \\} 
%	\vspace{0.01cm}
%	$\nearrow$ Cusco baja a la posición 19.	
%	\hyperlink{indice}{\beamergotobutton{Índice}} 
%\end{frame}


	%------------------------------------------------------------------------------------------------------------------------------------------------------------------------------------------------------------------------------------------
% SECCIÓN 2: Indicadores Epidemiológicos, Región Cusco
%------------------------------------------------------------------------------------------------------------------------------------------------------------------------------------------------------------------------------------------
\section{Indicadores Epidemiológicos, Región Cusco}
	\begin{frame}
		\frametitle{Curva de Sintomáticos y Asintomáticos Semanales}
		\vspace{-.5cm}
		\begin{center}
			\includegraphics[width=0.9\linewidth]{../figuras/sintomaticos_20_21_22.pdf}
		\end{center} 
		{\tiny Fuente de datos: SISCOVID, NOTICOVID.}
	\end{frame}
	
	\begin{frame}[label=epi_cusco]
		\frametitle{Curva del Total de Casos Positivos Semanales}
		\vspace{-.5cm}
		\begin{center}
			\includegraphics[width=.9\linewidth]{../figuras/positivos_semanales_20_21_22.pdf}
		\end{center}
		{\tiny Fuente de datos: SISCOVID, NOTICOVID.} \\
%			Nota: {\color{mycolor1} --- ---: Navidad}, {\color{mycolor1} - -: Año Nuevo}, {\color{mycolor2} - -: Semana Santa}, {\color{mycolor3} - -: Elecciones Primera Vuelta}, {\color{mycolor4} $- \cdot$: Elecciones Segunda Vuelta}. \\}	
	\end{frame}
	
	\begin{frame}[label=TipoPrueba]
		\frametitle{Curva Epidémica de Sintomáticos por Tipo de Prueba}
		\vspace{-.15cm}
		\begin{center}
			\includegraphics[width=0.7\linewidth, trim={0cm 0cm 0cm 0cm},clip]{../figuras/sinto_prueba_20_21_22.pdf}
		\end{center}
	\vspace{-.5cm}
		{\tiny Fuente de datos: SISCOVID, NOTICOVID.}\\
		Ver detalles de sintomáticos por tipo de prueba para cada provincia haciendo click en los siguientes enlaces:
		\hyperlink{Acomayo}{\beamergotobutton{Acomayo}} 
		\hyperlink{Anta}{\beamergotobutton{Anta}} 
		\hyperlink{Calca}{\beamergotobutton{Calca}} 
		\hyperlink{Canas}{\beamergotobutton{Canas}} \hyperlink{Chumbivilcas}{\beamergotobutton{Chumbivilcas}}
		\hyperlink{Canchis}{\beamergotobutton{Canchis}} 
		\hyperlink{Cusco}{\beamergotobutton{Cusco}}
		\hyperlink{Espinar}{\beamergotobutton{Espinar}}
		\hyperlink{laconvecion}{\beamergotobutton{La Convencion}}
		\hyperlink{Paruro}{\beamergotobutton{Paruro}} \hyperlink{Paucartambo}{\beamergotobutton{Paucartambo}}
		\hyperlink{Quispicanchi}{\beamergotobutton{Quispicanchi}}
		\hyperlink{Urubamba}{\beamergotobutton{Urubamba}}
	\end{frame}
	
	\begin{frame}
		\frametitle{Tasa de Crecimiento del Total de Casos por Semana}
		\vspace{-.5cm}
		\begin{center}
			\includegraphics[width=0.9\linewidth]{../figuras/positivos_crecimiento_2021_2022.pdf}
		\end{center} 
		{\tiny Fuente de datos: SISCOVID, NOTICOVID. \\
	Nota:{\color{mycolor1} - -: Año Nuevo}, {\color{mycolor2} - -: Semana Santa}, {\color{mycolor3} - -: Elecciones Primera Vuelta}, {\color{mycolor4} $- \cdot$: Elecciones Segunda Vuelta}, {\color{mycolor7} - -: Año 2022}. \\}
	\end{frame}
	
	\begin{frame}
			\frametitle{Tasa de Positividad Semanal por Tipo de Prueba, 2021-2022}
			\vspace{-.5cm}
			\begin{center}
				\includegraphics[width=0.8\linewidth]{../figuras/positividad_pcr.pdf}
			\end{center}
			{\tiny Fuente de datos: SISCOVID, NOTICOVID.}
	\end{frame}

	\begin{frame}
		\frametitle{Tasa de Positividad Semanal por Tipo de Prueba, 2021-2022}
		\vspace{-.5cm}
		\begin{center}
			\includegraphics[width=0.8\linewidth]{../figuras/positividad_ag.pdf}
		\end{center}
		{\tiny Fuente de datos: SISCOVID, NOTICOVID.}
	\end{frame}

	\begin{frame}[label=variantes]
		\frametitle{Tendencia de Variantes en la Región Cusco, 2021-2022}
		\vspace{-.5cm}
		\begin{center}
			\includegraphics[width=0.9\linewidth]{../figuras/variantes.pdf}
		\end{center}
		{\tiny Fuente de datos: NETLAB Cusco, UPCH.}
	\end{frame}

	\begin{frame}[label=subvariantes]
		\frametitle{Tendencia de Subvariantes Omicron en la Región Cusco, 2022}
		\vspace{-.5cm}
		\begin{center}
			\includegraphics[width=0.85\linewidth]{../figuras/subvariantes.pdf}
		\end{center}
		{\tiny Fuente de datos: NETLAB Cusco, UPCH.}
	\end{frame}

	\begin{frame}[label=mapa_variantes]
		\frametitle{Cantidad de Casos Variantes y Dispersión Geográfica en las Provincias de Cusco, 2021-2022}
		\begin{center}
		\includegraphics[width=0.4\linewidth]{../figuras/variantes_provincial.pdf}
		\end{center}
		{\tiny Fuente de datos: NETLAB Cusco, UNSAAC, UPCH.}	
		Ver detalles para cada  Provincia de Cusco, Distritos de la Región, y por Tipo de variantes haciendo clic en los siguientes enlaces:
		\hyperlink{mapa_provincia_cusco}{\beamergotobutton{Prov. Cusco}} \hyperlink{mapa_distrital}{\beamergotobutton{Distritos de Cusco}} \hyperlink{mapa_lambda}{\beamergotobutton{Lambda}}
		\hyperlink{mapa_gamma}{\beamergotobutton{Gamma}}
		\hyperlink{mapa_delta}{\beamergotobutton{Delta}}
		\hyperlink{mapa_delta}{\beamergotobutton{Omicron}}
	\end{frame}

\begin{frame}
	\frametitle{Propagación a Nivel Departamental}
	\vspace{-.5cm}
	\begin{center}
		\includegraphics[width=0.75\linewidth, trim={0cm .5cm 0cm 0.2cm},clip]{../sala_nacional/rt_cusco.png}
	\end{center}
	{\tiny Fuente: CDC MINSA. Reporte de Vigilancia COVID-19, Perú 2022. Actualización: 2022-12-19. Reporte de número reproductivo efectivo (Rt) de COVID-19. Disponible haciendo clic en el siguiente enlace: \href{https://www.dge.gob.pe/portalnuevo/informacion-publica/reporte-de-numero-reproductivo-efectivo-rt/}{CDC-Rt}. \\}
	\vspace{0.01cm}
	$\rightarrow$ Para ver el número exacto de RT por distrito, haga clic en el siguiente link: \href{https://www.dge.gob.pe/portalnuevo/informacion-publica/reporte-de-numero-reproductivo-efectivo-rt/}{\color{mycolor3}reporte-vigilancia-minsa}. \\
\end{frame}
	
\begin{frame}
	\frametitle{Defunciones Semanales por COVID-19}
	\vspace{-.5cm}
	\begin{center}
		\includegraphics[width=0.9\linewidth, trim={0cm .5cm 0cm 0.2cm},clip]{../figuras/defunciones_semanales_20_21_22.pdf}
	\end{center}
	{\tiny Fuente de datos: SINADEF - NOTICOVID.\\
%	Nota: {\color{mycolor1} --- ---: Navidad}, {\color{mycolor1} - -: Año Nuevo}, {\color{mycolor2} - -: Semana Santa}, {\color{mycolor3} - -: Elecciones Primera Vuelta}, {\color{mycolor4} $- \cdot$: Elecciones Segunda Vuelta},
	{\color{mycolor7} $- \cdot$: Año 2022}. 
	$\rightarrow$ El $86.4\%$ de los fallecidos son mayores de 60 años, siendo el grupo entre 80-89 años la mayoría ($91.4\%$).
	$\rightarrow$ El $2.5\%$ están entre los 0-11 años\\}
\end{frame}
	
\begin{frame}
	\frametitle{Promedio diario de Casos y Defunciones}
	\vspace{-.5cm}
	\begin{center}
		\includegraphics[width=0.9\linewidth, trim={0cm .5cm 0cm 0.2cm},clip]{../figuras/promedio_casos_defuncion_2020_2021_2022.png}
	\end{center}
	{\tiny Fuente de datos: SINADEF - NOTICOVID.}\\
\end{frame}
	
\begin{frame}
	\frametitle{Tasa de Crecimiento de Defunciones por Semana}
	\vspace{-.5cm}
	\begin{center}
		\includegraphics[width=0.9\linewidth]{../figuras/defunciones_tasa_crecimiento_21_22.pdf}
	\end{center} 
	{\tiny Fuente de datos: SINADEF - NOTICOVID. \\
		Nota: {\color{mycolor1} - -: Año Nuevo}, {\color{mycolor2} - -: Semana Santa}, {\color{mycolor3} - -: Elecciones Primera Vuelta}, {\color{mycolor4} $- \cdot$: Elecciones Segunda Vuelta}, {\color{mycolor7} $- \cdot$: Año 2022}. \\}
\end{frame}

\begin{frame}
	\frametitle{Exceso de Defunciones por Todas las Causas}
	\vspace{-.5cm}
	\begin{center}
		\includegraphics[width=0.9\linewidth]{../figuras/exceso_region_2022.pdf}
	\end{center}
	{\tiny Fuente de datos: SINADEF - NOTICOVID.} 
\end{frame}
	
\begin{frame}
	\frametitle{Mortalidad por Grupos de Edad}
	\vspace{-.1cm}
	\begin{center}
		\includegraphics[width=0.9\linewidth]{../figuras/mortalidad_edad_2021_2022.pdf}
	\end{center}
	{\tiny Fuente de datos: SINADEF - NOTICOVID, Dirección Ejecutiva de Atención Integral de Salud} 
\end{frame}

\begin{frame}
	\frametitle{Mortalidad por Grupos de Edad (e Hitos de Vacunación)}
	\vspace{.5cm}
	\begin{figure}
		\centering
		\begin{subfigure}[b]{0.3\textwidth}
			\centering
			\includegraphics[width=\textwidth]{../figuras/mortalidad_edad_80.pdf}
			\caption{Más de 80 años}
			%\label{fig:}
		\end{subfigure}
		\hfill
		\begin{subfigure}[b]{0.3\textwidth}
			\centering
			\includegraphics[width=\textwidth]{../figuras/mortalidad_edad_70.pdf}
			\caption{70 a 79 años}
			%\label{fig:70 a 79 años}
		\end{subfigure}
		\hfill
		\begin{subfigure}[b]{0.3\textwidth}
			\centering
			\includegraphics[width=\textwidth]{../figuras/mortalidad_edad_60.pdf}
			\caption{60 a 69 años}
			%\label{fig:60 a 69 años}
		\end{subfigure}
	\vspace{10mm}
		\begin{subfigure}[b]{0.3\textwidth}
		\centering
		\includegraphics[width=\textwidth]{../figuras/mortalidad_edad_50.pdf}
		\caption{50 a 59 años}
		%\label{fig:50 a 59 años}
	\end{subfigure}
		\hfill
	\begin{subfigure}[b]{0.3\textwidth}
		\centering
		\includegraphics[width=\textwidth]{../figuras/mortalidad_edad_40.pdf}
		\caption{40 a 49 años}
		%\label{fig:40 a 49 años}
	\end{subfigure}
	\hfill
	\begin{subfigure}[b]{0.3\textwidth}
		\centering
		\includegraphics[width=\textwidth]{../figuras/mortalidad_edad_30.pdf}
		\caption{30 a 39 años}
		%\label{fig:40 a 49 años}
	\end{subfigure}
	
		%\caption{Three simple graphs}
		%\label{fig:three graphs}
	\end{figure}
\end{frame}

\begin{frame}
	\frametitle{Mortalidad por Grupos de Edad (e Hitos de Vacunación)}
	\vspace{-.2cm}
	\hfill
	\begin{figure}
		\begin{subfigure}[b]{0.3\textwidth}

			\includegraphics[width=\textwidth]{../figuras/mortalidad_edad_20.pdf}
			\caption{20 a 29 años}
			%\label{fig:}
		\end{subfigure}
		\centering
		\begin{subfigure}[b]{0.3\textwidth}
			\includegraphics[width=\textwidth]{../figuras/mortalidad_edad_10.pdf}
			\caption{10 a 19 años}
			%\label{fig:70 a 79 años}
		\end{subfigure}
		\begin{subfigure}[b]{0.3\textwidth}

			\includegraphics[width=\textwidth]{../figuras/mortalidad_edad_0.pdf}
			\caption{0 a 9 años}
			%\label{fig:60 a 69 años}
		\end{subfigure}
		\vspace{10mm}	
		%\caption{Three simple graphs}
		%\label{fig:three graphs}
	\end{figure}
	\vspace{-.8cm} 
	{\tiny Fuente de datos: SINADEF - NOTICOVID, Dirección Ejecutiva de Atención Integral de Salud.\\}
	{\tiny Nota: Líneas punteadas (- -) es el inicio de la primera dosis de vacuna contra COVID-19 y línea continua (--) es el inicio de la segunda dosis en el respectivo grupo de edad. Nota 2: Las escalas en el eje y son diferentes.\\}
	$\rightarrow$ {\small La tasa de mortalidad presenta descenso en el número de fallecidos. Se observa que en los grupos etarios que accedieron a la \textbf{\color{mycolor3}vacunación anticipada} presentaron menos número de fallecidos}  \hyperlink{indice}{\beamergotobutton{Índice}}
\end{frame}

%\begin{frame}
%	\frametitle{Defunciones, tasa de mortalidad y letalidad, Región Cusco, 2020 - 2022}
%	\vspace{-1.0cm}
%	\begin{table}[]
%		\resizebox{\textwidth}{!}{%
%				\begin{tabular}{lccc|cccccc|}
		\cline{5-10}
		&
		\multicolumn{1}{l}{} &
		&
		&
		\multicolumn{6}{c|}{\cellcolor[HTML]{F2F2F2}} \\
		&
		\multicolumn{1}{l}{} &
		\multicolumn{1}{l}{} &
		\multicolumn{1}{l|}{} &
		\multicolumn{6}{c|}{\multirow{-2}{*}{\cellcolor[HTML]{F2F2F2}\textbf{Etapa de Vida}}} \\ \cline{5-10} 
		&
		\multicolumn{1}{l}{} &
		\multicolumn{1}{l}{} &
		\multicolumn{1}{l|}{} &
		\multicolumn{1}{c|}{\cellcolor[HTML]{F2F2F2}\textbf{Niño}} &
		\multicolumn{1}{l|}{\cellcolor[HTML]{F2F2F2}\textbf{Adolescente}} &
		\multicolumn{1}{l|}{\cellcolor[HTML]{F2F2F2}\textbf{Joven}} &
		\multicolumn{1}{l|}{\cellcolor[HTML]{F2F2F2}\textbf{Adulto}} &
		\multicolumn{1}{l|}{\cellcolor[HTML]{F2F2F2}\textbf{Adulto Mayor}} &
		\cellcolor[HTML]{F2F2F2}\textbf{Total} \\ \cline{2-10} 
		\multicolumn{1}{l|}{} &
		\multicolumn{1}{c|}{\cellcolor[HTML]{ECF4FF}} &
		\multicolumn{1}{c|}{\cellcolor[HTML]{ECF4FF}} &
		\cellcolor[HTML]{ECF4FF}\textbf{Tasa (\%)} &
		\multicolumn{1}{c|}{\cellcolor[HTML]{ECF4FF}0.4} &
		\multicolumn{1}{c|}{\cellcolor[HTML]{ECF4FF}0.049} &
		\multicolumn{1}{c|}{\cellcolor[HTML]{ECF4FF}0.12} &
		\multicolumn{1}{c|}{\cellcolor[HTML]{ECF4FF}0.57} &
		\multicolumn{1}{c|}{\cellcolor[HTML]{ECF4FF}7.9} &
		\cellcolor[HTML]{ECF4FF}1.3 \\ \cline{4-10} 
		\multicolumn{1}{l|}{} &
		\multicolumn{1}{c|}{\cellcolor[HTML]{ECF4FF}} &
		\multicolumn{1}{c|}{\multirow{-2}{*}{\cellcolor[HTML]{ECF4FF}\textbf{Letalidad}}} &
		\cellcolor[HTML]{ECF4FF} &
		\multicolumn{1}{c|}{\cellcolor[HTML]{ECF4FF}} &
		\multicolumn{1}{c|}{\cellcolor[HTML]{ECF4FF}} &
		\multicolumn{1}{c|}{\cellcolor[HTML]{ECF4FF}} &
		\multicolumn{1}{c|}{\cellcolor[HTML]{ECF4FF}} &
		\multicolumn{1}{c|}{\cellcolor[HTML]{ECF4FF}} &
		\cellcolor[HTML]{ECF4FF} \\ \cline{3-3}
		\multicolumn{1}{l|}{} &
		\multicolumn{1}{c|}{\cellcolor[HTML]{ECF4FF}} &
		\multicolumn{1}{c|}{\cellcolor[HTML]{ECF4FF}} &
		\multirow{-2}{*}{\cellcolor[HTML]{ECF4FF}\textbf{Defunciones}} &
		\multicolumn{1}{c|}{\multirow{-2}{*}{\cellcolor[HTML]{ECF4FF}07}} &
		\multicolumn{1}{c|}{\multirow{-2}{*}{\cellcolor[HTML]{ECF4FF}01}} &
		\multicolumn{1}{c|}{\multirow{-2}{*}{\cellcolor[HTML]{ECF4FF}29}} &
		\multicolumn{1}{c|}{\multirow{-2}{*}{\cellcolor[HTML]{ECF4FF}375}} &
		\multicolumn{1}{c|}{\multirow{-2}{*}{\cellcolor[HTML]{ECF4FF}973}} &
		\multirow{-2}{*}{\cellcolor[HTML]{ECF4FF}1385} \\ \cline{4-10} 
		\multicolumn{1}{l|}{} &
		\multicolumn{1}{c|}{\cellcolor[HTML]{ECF4FF}} &
		\multicolumn{1}{c|}{\multirow{-2}{*}{\cellcolor[HTML]{ECF4FF}\textbf{Mortalidad}}} &
		\cellcolor[HTML]{ECF4FF}\textbf{Tasa*} &
		\multicolumn{1}{c|}{\cellcolor[HTML]{ECF4FF}5.2} &
		\multicolumn{1}{c|}{\cellcolor[HTML]{ECF4FF}0.74} &
		\multicolumn{1}{c|}{\cellcolor[HTML]{ECF4FF}21} &
		\multicolumn{1}{c|}{\cellcolor[HTML]{ECF4FF}276} &
		\multicolumn{1}{c|}{\cellcolor[HTML]{ECF4FF}717} &
		\cellcolor[HTML]{ECF4FF}1020 \\ \cline{3-10} 
		\multicolumn{1}{l|}{} &
		\multicolumn{1}{c|}{\cellcolor[HTML]{ECF4FF}} &
		\multicolumn{1}{c|}{\cellcolor[HTML]{ECF4FF}} &
		\cellcolor[HTML]{ECF4FF}\textbf{Casos +} &
		\multicolumn{1}{c|}{\cellcolor[HTML]{ECF4FF}1749} &
		\multicolumn{1}{c|}{\cellcolor[HTML]{ECF4FF}2029} &
		\multicolumn{1}{c|}{\cellcolor[HTML]{ECF4FF}25091} &
		\multicolumn{1}{c|}{\cellcolor[HTML]{ECF4FF}66024} &
		\multicolumn{1}{c|}{\cellcolor[HTML]{ECF4FF}12255} &
		\cellcolor[HTML]{ECF4FF}107148 \\ \cline{4-10} 
		\multicolumn{1}{l|}{} &
		\multicolumn{1}{c|}{\multirow{-6}{*}{\cellcolor[HTML]{ECF4FF}\textbf{2020}}} &
		\multicolumn{1}{c|}{\multirow{-2}{*}{\cellcolor[HTML]{ECF4FF}\textbf{Incidencia}}} &
		\cellcolor[HTML]{ECF4FF}\textbf{Tasa*} &
		\multicolumn{1}{c|}{\cellcolor[HTML]{ECF4FF}1288} &
		\multicolumn{1}{c|}{\cellcolor[HTML]{ECF4FF}1495} &
		\multicolumn{1}{c|}{\cellcolor[HTML]{ECF4FF}18483} &
		\multicolumn{1}{c|}{\cellcolor[HTML]{ECF4FF}48637} &
		\multicolumn{1}{c|}{\cellcolor[HTML]{ECF4FF}9028} &
		\cellcolor[HTML]{ECF4FF}7.9 \\ \cline{2-10} 
		\multicolumn{1}{l|}{} &
		\multicolumn{1}{c|}{\cellcolor[HTML]{FFFFC7}} &
		\multicolumn{1}{c|}{\cellcolor[HTML]{FFFFC7}} &
		\cellcolor[HTML]{FFFFC7}\textbf{Tasa (\%)} &
		\multicolumn{1}{c|}{\cellcolor[HTML]{FFFFC7}0.94} &
		\multicolumn{1}{c|}{\cellcolor[HTML]{FFFFC7}0.087} &
		\multicolumn{1}{c|}{\cellcolor[HTML]{FFFFC7}0.13} &
		\multicolumn{1}{c|}{\cellcolor[HTML]{FFFFC7}1.9} &
		\multicolumn{1}{c|}{\cellcolor[HTML]{FFFFC7}19} &
		\cellcolor[HTML]{FFFFC7}3.8 \\ \cline{4-10} 
		\multicolumn{1}{l|}{} &
		\multicolumn{1}{c|}{\cellcolor[HTML]{FFFFC7}} &
		\multicolumn{1}{c|}{\multirow{-2}{*}{\cellcolor[HTML]{FFFFC7}\textbf{Letalidad}}} &
		\cellcolor[HTML]{FFFFC7} &
		\multicolumn{1}{c|}{\cellcolor[HTML]{FFFFC7}} &
		\multicolumn{1}{c|}{\cellcolor[HTML]{FFFFC7}} &
		\multicolumn{1}{c|}{\cellcolor[HTML]{FFFFC7}} &
		\multicolumn{1}{c|}{\cellcolor[HTML]{FFFFC7}} &
		\multicolumn{1}{c|}{\cellcolor[HTML]{FFFFC7}} &
		\cellcolor[HTML]{FFFFC7} \\ \cline{3-3}
		\multicolumn{1}{l|}{} &
		\multicolumn{1}{c|}{\cellcolor[HTML]{FFFFC7}} &
		\multicolumn{1}{c|}{\cellcolor[HTML]{FFFFC7}} &
		\multirow{-2}{*}{\cellcolor[HTML]{FFFFC7}\textbf{Defunciones}} &
		\multicolumn{1}{c|}{\multirow{-2}{*}{\cellcolor[HTML]{FFFFC7}11}} &
		\multicolumn{1}{c|}{\multirow{-2}{*}{\cellcolor[HTML]{FFFFC7}04}} &
		\multicolumn{1}{c|}{\multirow{-2}{*}{\cellcolor[HTML]{FFFFC7}25}} &
		\multicolumn{1}{c|}{\multirow{-2}{*}{\cellcolor[HTML]{FFFFC7}826}} &
		\multicolumn{1}{c|}{\multirow{-2}{*}{\cellcolor[HTML]{FFFFC7}2127}} &
		\multirow{-2}{*}{\cellcolor[HTML]{FFFFC7}2993} \\ \cline{4-10} 
		\multicolumn{1}{l|}{} &
		\multicolumn{1}{c|}{\cellcolor[HTML]{FFFFC7}} &
		\multicolumn{1}{c|}{\multirow{-2}{*}{\cellcolor[HTML]{FFFFC7}\textbf{Mortalidad}}} &
		\cellcolor[HTML]{FFFFC7}\textbf{Tasa*} &
		\multicolumn{1}{c|}{\cellcolor[HTML]{FFFFC7}8.1} &
		\multicolumn{1}{c|}{\cellcolor[HTML]{FFFFC7}2.9} &
		\multicolumn{1}{c|}{\cellcolor[HTML]{FFFFC7}18} &
		\multicolumn{1}{c|}{\cellcolor[HTML]{FFFFC7}608} &
		\multicolumn{1}{c|}{\cellcolor[HTML]{FFFFC7}1567} &
		\cellcolor[HTML]{FFFFC7}2205 \\ \cline{3-10} 
		\multicolumn{1}{l|}{} &
		\multicolumn{1}{c|}{\cellcolor[HTML]{FFFFC7}} &
		\multicolumn{1}{c|}{\cellcolor[HTML]{FFFFC7}} &
		\cellcolor[HTML]{FFFFC7}\textbf{Casos +} &
		\multicolumn{1}{c|}{\cellcolor[HTML]{FFFFC7}1173} &
		\multicolumn{1}{c|}{\cellcolor[HTML]{FFFFC7}4573} &
		\multicolumn{1}{c|}{\cellcolor[HTML]{FFFFC7}19526} &
		\multicolumn{1}{c|}{\cellcolor[HTML]{FFFFC7}43215} &
		\multicolumn{1}{c|}{\cellcolor[HTML]{FFFFC7}11129} &
		\cellcolor[HTML]{FFFFC7}79616 \\ \cline{4-10} 
		\multicolumn{1}{l|}{} &
		\multicolumn{1}{c|}{\multirow{-6}{*}{\cellcolor[HTML]{FFFFC7}\textbf{2021}}} &
		\multicolumn{1}{c|}{\multirow{-2}{*}{\cellcolor[HTML]{FFFFC7}\textbf{Incidencia}}} &
		\cellcolor[HTML]{FFFFC7}\textbf{Tasa*} &
		\multicolumn{1}{c|}{\cellcolor[HTML]{FFFFC7}864} &
		\multicolumn{1}{c|}{\cellcolor[HTML]{FFFFC7}3369} &
		\multicolumn{1}{c|}{\cellcolor[HTML]{FFFFC7}14304} &
		\multicolumn{1}{c|}{\cellcolor[HTML]{FFFFC7}31834} &
		\multicolumn{1}{c|}{\cellcolor[HTML]{FFFFC7}8198} &
		\cellcolor[HTML]{FFFFC7}58649 \\ \cline{2-10} 
		\multicolumn{1}{l|}{} &
		\multicolumn{1}{c|}{\cellcolor[HTML]{E2EFDA}} &
		\multicolumn{1}{c|}{\cellcolor[HTML]{E2EFDA}} &
		\cellcolor[HTML]{E2EFDA}\textbf{Tasa(\%)} &
		\multicolumn{1}{c|}{\cellcolor[HTML]{E2EFDA}0.35} &
		\multicolumn{1}{c|}{\cellcolor[HTML]{E2EFDA}0.099} &
		\multicolumn{1}{c|}{\cellcolor[HTML]{E2EFDA}0.028} &
		\multicolumn{1}{c|}{\cellcolor[HTML]{E2EFDA}0.13} &
		\multicolumn{1}{c|}{\cellcolor[HTML]{E2EFDA}3.7} &
		\cellcolor[HTML]{E2EFDA}0.48 \\ \cline{4-10} 
		\multicolumn{1}{l|}{} &
		\multicolumn{1}{c|}{\cellcolor[HTML]{E2EFDA}} &
		\multicolumn{1}{c|}{\multirow{-2}{*}{\cellcolor[HTML]{E2EFDA}\textbf{Letalidad}}} &
		\cellcolor[HTML]{E2EFDA} &
		\multicolumn{1}{c|}{\cellcolor[HTML]{E2EFDA}} &
		\multicolumn{1}{c|}{\cellcolor[HTML]{E2EFDA}} &
		\multicolumn{1}{c|}{\cellcolor[HTML]{E2EFDA}} &
		\multicolumn{1}{c|}{\cellcolor[HTML]{E2EFDA}} &
		\multicolumn{1}{c|}{\cellcolor[HTML]{E2EFDA}} &
		\cellcolor[HTML]{E2EFDA} \\ \cline{3-3}
		\multicolumn{1}{l|}{} &
		\multicolumn{1}{c|}{\cellcolor[HTML]{E2EFDA}} &
		\multicolumn{1}{c|}{\cellcolor[HTML]{E2EFDA}} &
		\multirow{-2}{*}{\cellcolor[HTML]{E2EFDA}\textbf{Defunciones}} &
		\multicolumn{1}{c|}{\multirow{-2}{*}{\cellcolor[HTML]{E2EFDA}07}} &
		\multicolumn{1}{c|}{\multirow{-2}{*}{\cellcolor[HTML]{E2EFDA}02}} &
		\multicolumn{1}{c|}{\multirow{-2}{*}{\cellcolor[HTML]{E2EFDA}04}} &
		\multicolumn{1}{c|}{\multirow{-2}{*}{\cellcolor[HTML]{E2EFDA}37}} &
		\multicolumn{1}{c|}{\multirow{-2}{*}{\cellcolor[HTML]{E2EFDA}193}} &
		\multirow{-2}{*}{\cellcolor[HTML]{E2EFDA}234} \\ \cline{4-10} 
		\multicolumn{1}{l|}{} &
		\multicolumn{1}{c|}{\cellcolor[HTML]{E2EFDA}} &
		\multicolumn{1}{c|}{\multirow{-2}{*}{\cellcolor[HTML]{E2EFDA}\textbf{Mortalidad}}} &
		\cellcolor[HTML]{E2EFDA}\textbf{Tasa *} &
		\multicolumn{1}{c|}{\cellcolor[HTML]{E2EFDA}5.2} &
		\multicolumn{1}{c|}{\cellcolor[HTML]{E2EFDA}1.5} &
		\multicolumn{1}{c|}{\cellcolor[HTML]{E2EFDA}2.9} &
		\multicolumn{1}{c|}{\cellcolor[HTML]{E2EFDA}27} &
		\multicolumn{1}{c|}{\cellcolor[HTML]{E2EFDA}142} &
		\cellcolor[HTML]{E2EFDA}172 \\ \cline{3-10} 
		\multicolumn{1}{l|}{} &
		\multicolumn{1}{c|}{\cellcolor[HTML]{E2EFDA}} &     
		\multicolumn{1}{c|}{\cellcolor[HTML]{E2EFDA}} &
		\cellcolor[HTML]{E2EFDA}\textbf{Casos +} &
		\multicolumn{1}{c|}{\cellcolor[HTML]{E2EFDA}2001} &
		\multicolumn{1}{c|}{\cellcolor[HTML]{E2EFDA}2025} &
		\multicolumn{1}{c|}{\cellcolor[HTML]{E2EFDA}14174} &
		\multicolumn{1}{c|}{\cellcolor[HTML]{E2EFDA}27487} &
		\multicolumn{1}{c|}{\cellcolor[HTML]{E2EFDA}5178} &
		\cellcolor[HTML]{E2EFDA}50865 \\ \cline{4-10} 
		\multicolumn{1}{l|}{} &
		\multicolumn{1}{c|}{\multirow{-6}{*}{\cellcolor[HTML]{E2EFDA}\textbf{2022}}} &
		\multicolumn{1}{c|}{\multirow{-2}{*}{\cellcolor[HTML]{E2EFDA}\textbf{Incidencia}}} &
		\cellcolor[HTML]{E2EFDA}\textbf{Tasa} &
		\multicolumn{1}{c|}{\cellcolor[HTML]{E2EFDA}1474} &
		\multicolumn{1}{c|}{\cellcolor[HTML]{E2EFDA}1492} &
		\multicolumn{1}{c|}{\cellcolor[HTML]{E2EFDA}10441} &
		\multicolumn{1}{c|}{\cellcolor[HTML]{E2EFDA}20248} &
		\multicolumn{1}{c|}{\cellcolor[HTML]{E2EFDA}3814} &
		\cellcolor[HTML]{E2EFDA}37470 \\ \cline{2-10} 
	\end{tabular}
%		}
%	\end{table}	
%	{   \tiny Fuente de datos: NOTICOVID, SISCOVID, SINADEF.\\
%		Tasa de mortalidad ajustada 1 000 000 de habitantes*.\\
%		Tasa de incidencia ajustada 1 000 000 de habitantes*.\\ 
%	}
%\end{frame}

%\begin{frame}
%	\frametitle{Tabla por Casos y Tasa de Incidencia Perú, 2020 - 2022}
%	\vspace{-.5cm}
%	\begin{table}[]
%		\resizebox{\textwidth}{!}{%
%			\input{../tablas/tabla_incidencia_curso_vida.tex}
%		}
%	\end{table}	
%	{\tiny Fuente de datos: SISCOVID, NOTICOVID.}
%	
%\end{frame}

%\begin{frame}
%	\frametitle{Tabla por Defunciones y Tasa de Mortalidad Perú, 2020 - 2022}
%	\vspace{-.5cm}
%	
%	\begin{table}[]
%		\resizebox{\textwidth}{!}{%
%				\begin{tabular}{lccc|cccccc|}
		\cline{5-10}
		&
		\multicolumn{1}{l}{} &
		&
		&
		\multicolumn{6}{c|}{\cellcolor[HTML]{F2F2F2}} \\
		&
		\multicolumn{1}{l}{} &
		\multicolumn{1}{l}{} &
		\multicolumn{1}{l|}{} &
		\multicolumn{6}{c|}{\multirow{-2}{*}{\cellcolor[HTML]{F2F2F2}\textbf{Etapa de Vida}}} \\ \cline{5-10} 
		&
		\multicolumn{1}{l}{} &
		\multicolumn{1}{l}{} &
		\multicolumn{1}{l|}{} &
		\multicolumn{1}{c|}{\cellcolor[HTML]{F2F2F2}\textbf{Niño}} &
		\multicolumn{1}{l|}{\cellcolor[HTML]{F2F2F2}\textbf{Adolescente}} &
		\multicolumn{1}{l|}{\cellcolor[HTML]{F2F2F2}\textbf{Joven}} &
		\multicolumn{1}{l|}{\cellcolor[HTML]{F2F2F2}\textbf{Adulto}} &
		\multicolumn{1}{l|}{\cellcolor[HTML]{F2F2F2}\textbf{Adulto Mayor}} &
		\cellcolor[HTML]{F2F2F2}\textbf{Total} \\ \cline{2-10} 
		\multicolumn{1}{l|}{} &
		\multicolumn{1}{c|}{\cellcolor[HTML]{ECF4FF}} &
		\multicolumn{1}{c|}{\cellcolor[HTML]{ECF4FF}} &
		\cellcolor[HTML]{ECF4FF}\textbf{Tasa (\%)} &
		\multicolumn{1}{c|}{\cellcolor[HTML]{ECF4FF}0.4} &
		\multicolumn{1}{c|}{\cellcolor[HTML]{ECF4FF}0.049} &
		\multicolumn{1}{c|}{\cellcolor[HTML]{ECF4FF}0.12} &
		\multicolumn{1}{c|}{\cellcolor[HTML]{ECF4FF}0.57} &
		\multicolumn{1}{c|}{\cellcolor[HTML]{ECF4FF}7.9} &
		\cellcolor[HTML]{ECF4FF}1.3 \\ \cline{4-10} 
		\multicolumn{1}{l|}{} &
		\multicolumn{1}{c|}{\cellcolor[HTML]{ECF4FF}} &
		\multicolumn{1}{c|}{\multirow{-2}{*}{\cellcolor[HTML]{ECF4FF}\textbf{Letalidad}}} &
		\cellcolor[HTML]{ECF4FF} &
		\multicolumn{1}{c|}{\cellcolor[HTML]{ECF4FF}} &
		\multicolumn{1}{c|}{\cellcolor[HTML]{ECF4FF}} &
		\multicolumn{1}{c|}{\cellcolor[HTML]{ECF4FF}} &
		\multicolumn{1}{c|}{\cellcolor[HTML]{ECF4FF}} &
		\multicolumn{1}{c|}{\cellcolor[HTML]{ECF4FF}} &
		\cellcolor[HTML]{ECF4FF} \\ \cline{3-3}
		\multicolumn{1}{l|}{} &
		\multicolumn{1}{c|}{\cellcolor[HTML]{ECF4FF}} &
		\multicolumn{1}{c|}{\cellcolor[HTML]{ECF4FF}} &
		\multirow{-2}{*}{\cellcolor[HTML]{ECF4FF}\textbf{Defunciones}} &
		\multicolumn{1}{c|}{\multirow{-2}{*}{\cellcolor[HTML]{ECF4FF}07}} &
		\multicolumn{1}{c|}{\multirow{-2}{*}{\cellcolor[HTML]{ECF4FF}01}} &
		\multicolumn{1}{c|}{\multirow{-2}{*}{\cellcolor[HTML]{ECF4FF}29}} &
		\multicolumn{1}{c|}{\multirow{-2}{*}{\cellcolor[HTML]{ECF4FF}375}} &
		\multicolumn{1}{c|}{\multirow{-2}{*}{\cellcolor[HTML]{ECF4FF}973}} &
		\multirow{-2}{*}{\cellcolor[HTML]{ECF4FF}1385} \\ \cline{4-10} 
		\multicolumn{1}{l|}{} &
		\multicolumn{1}{c|}{\cellcolor[HTML]{ECF4FF}} &
		\multicolumn{1}{c|}{\multirow{-2}{*}{\cellcolor[HTML]{ECF4FF}\textbf{Mortalidad}}} &
		\cellcolor[HTML]{ECF4FF}\textbf{Tasa*} &
		\multicolumn{1}{c|}{\cellcolor[HTML]{ECF4FF}5.2} &
		\multicolumn{1}{c|}{\cellcolor[HTML]{ECF4FF}0.74} &
		\multicolumn{1}{c|}{\cellcolor[HTML]{ECF4FF}21} &
		\multicolumn{1}{c|}{\cellcolor[HTML]{ECF4FF}276} &
		\multicolumn{1}{c|}{\cellcolor[HTML]{ECF4FF}717} &
		\cellcolor[HTML]{ECF4FF}1020 \\ \cline{3-10} 
		\multicolumn{1}{l|}{} &
		\multicolumn{1}{c|}{\cellcolor[HTML]{ECF4FF}} &
		\multicolumn{1}{c|}{\cellcolor[HTML]{ECF4FF}} &
		\cellcolor[HTML]{ECF4FF}\textbf{Casos +} &
		\multicolumn{1}{c|}{\cellcolor[HTML]{ECF4FF}1749} &
		\multicolumn{1}{c|}{\cellcolor[HTML]{ECF4FF}2029} &
		\multicolumn{1}{c|}{\cellcolor[HTML]{ECF4FF}25091} &
		\multicolumn{1}{c|}{\cellcolor[HTML]{ECF4FF}66024} &
		\multicolumn{1}{c|}{\cellcolor[HTML]{ECF4FF}12255} &
		\cellcolor[HTML]{ECF4FF}107148 \\ \cline{4-10} 
		\multicolumn{1}{l|}{} &
		\multicolumn{1}{c|}{\multirow{-6}{*}{\cellcolor[HTML]{ECF4FF}\textbf{2020}}} &
		\multicolumn{1}{c|}{\multirow{-2}{*}{\cellcolor[HTML]{ECF4FF}\textbf{Incidencia}}} &
		\cellcolor[HTML]{ECF4FF}\textbf{Tasa*} &
		\multicolumn{1}{c|}{\cellcolor[HTML]{ECF4FF}1288} &
		\multicolumn{1}{c|}{\cellcolor[HTML]{ECF4FF}1495} &
		\multicolumn{1}{c|}{\cellcolor[HTML]{ECF4FF}18483} &
		\multicolumn{1}{c|}{\cellcolor[HTML]{ECF4FF}48637} &
		\multicolumn{1}{c|}{\cellcolor[HTML]{ECF4FF}9028} &
		\cellcolor[HTML]{ECF4FF}7.9 \\ \cline{2-10} 
		\multicolumn{1}{l|}{} &
		\multicolumn{1}{c|}{\cellcolor[HTML]{FFFFC7}} &
		\multicolumn{1}{c|}{\cellcolor[HTML]{FFFFC7}} &
		\cellcolor[HTML]{FFFFC7}\textbf{Tasa (\%)} &
		\multicolumn{1}{c|}{\cellcolor[HTML]{FFFFC7}0.94} &
		\multicolumn{1}{c|}{\cellcolor[HTML]{FFFFC7}0.087} &
		\multicolumn{1}{c|}{\cellcolor[HTML]{FFFFC7}0.13} &
		\multicolumn{1}{c|}{\cellcolor[HTML]{FFFFC7}1.9} &
		\multicolumn{1}{c|}{\cellcolor[HTML]{FFFFC7}19} &
		\cellcolor[HTML]{FFFFC7}3.8 \\ \cline{4-10} 
		\multicolumn{1}{l|}{} &
		\multicolumn{1}{c|}{\cellcolor[HTML]{FFFFC7}} &
		\multicolumn{1}{c|}{\multirow{-2}{*}{\cellcolor[HTML]{FFFFC7}\textbf{Letalidad}}} &
		\cellcolor[HTML]{FFFFC7} &
		\multicolumn{1}{c|}{\cellcolor[HTML]{FFFFC7}} &
		\multicolumn{1}{c|}{\cellcolor[HTML]{FFFFC7}} &
		\multicolumn{1}{c|}{\cellcolor[HTML]{FFFFC7}} &
		\multicolumn{1}{c|}{\cellcolor[HTML]{FFFFC7}} &
		\multicolumn{1}{c|}{\cellcolor[HTML]{FFFFC7}} &
		\cellcolor[HTML]{FFFFC7} \\ \cline{3-3}
		\multicolumn{1}{l|}{} &
		\multicolumn{1}{c|}{\cellcolor[HTML]{FFFFC7}} &
		\multicolumn{1}{c|}{\cellcolor[HTML]{FFFFC7}} &
		\multirow{-2}{*}{\cellcolor[HTML]{FFFFC7}\textbf{Defunciones}} &
		\multicolumn{1}{c|}{\multirow{-2}{*}{\cellcolor[HTML]{FFFFC7}11}} &
		\multicolumn{1}{c|}{\multirow{-2}{*}{\cellcolor[HTML]{FFFFC7}04}} &
		\multicolumn{1}{c|}{\multirow{-2}{*}{\cellcolor[HTML]{FFFFC7}25}} &
		\multicolumn{1}{c|}{\multirow{-2}{*}{\cellcolor[HTML]{FFFFC7}826}} &
		\multicolumn{1}{c|}{\multirow{-2}{*}{\cellcolor[HTML]{FFFFC7}2127}} &
		\multirow{-2}{*}{\cellcolor[HTML]{FFFFC7}2993} \\ \cline{4-10} 
		\multicolumn{1}{l|}{} &
		\multicolumn{1}{c|}{\cellcolor[HTML]{FFFFC7}} &
		\multicolumn{1}{c|}{\multirow{-2}{*}{\cellcolor[HTML]{FFFFC7}\textbf{Mortalidad}}} &
		\cellcolor[HTML]{FFFFC7}\textbf{Tasa*} &
		\multicolumn{1}{c|}{\cellcolor[HTML]{FFFFC7}8.1} &
		\multicolumn{1}{c|}{\cellcolor[HTML]{FFFFC7}2.9} &
		\multicolumn{1}{c|}{\cellcolor[HTML]{FFFFC7}18} &
		\multicolumn{1}{c|}{\cellcolor[HTML]{FFFFC7}608} &
		\multicolumn{1}{c|}{\cellcolor[HTML]{FFFFC7}1567} &
		\cellcolor[HTML]{FFFFC7}2205 \\ \cline{3-10} 
		\multicolumn{1}{l|}{} &
		\multicolumn{1}{c|}{\cellcolor[HTML]{FFFFC7}} &
		\multicolumn{1}{c|}{\cellcolor[HTML]{FFFFC7}} &
		\cellcolor[HTML]{FFFFC7}\textbf{Casos +} &
		\multicolumn{1}{c|}{\cellcolor[HTML]{FFFFC7}1173} &
		\multicolumn{1}{c|}{\cellcolor[HTML]{FFFFC7}4573} &
		\multicolumn{1}{c|}{\cellcolor[HTML]{FFFFC7}19526} &
		\multicolumn{1}{c|}{\cellcolor[HTML]{FFFFC7}43215} &
		\multicolumn{1}{c|}{\cellcolor[HTML]{FFFFC7}11129} &
		\cellcolor[HTML]{FFFFC7}79616 \\ \cline{4-10} 
		\multicolumn{1}{l|}{} &
		\multicolumn{1}{c|}{\multirow{-6}{*}{\cellcolor[HTML]{FFFFC7}\textbf{2021}}} &
		\multicolumn{1}{c|}{\multirow{-2}{*}{\cellcolor[HTML]{FFFFC7}\textbf{Incidencia}}} &
		\cellcolor[HTML]{FFFFC7}\textbf{Tasa*} &
		\multicolumn{1}{c|}{\cellcolor[HTML]{FFFFC7}864} &
		\multicolumn{1}{c|}{\cellcolor[HTML]{FFFFC7}3369} &
		\multicolumn{1}{c|}{\cellcolor[HTML]{FFFFC7}14304} &
		\multicolumn{1}{c|}{\cellcolor[HTML]{FFFFC7}31834} &
		\multicolumn{1}{c|}{\cellcolor[HTML]{FFFFC7}8198} &
		\cellcolor[HTML]{FFFFC7}58649 \\ \cline{2-10} 
		\multicolumn{1}{l|}{} &
		\multicolumn{1}{c|}{\cellcolor[HTML]{E2EFDA}} &
		\multicolumn{1}{c|}{\cellcolor[HTML]{E2EFDA}} &
		\cellcolor[HTML]{E2EFDA}\textbf{Tasa(\%)} &
		\multicolumn{1}{c|}{\cellcolor[HTML]{E2EFDA}0.35} &
		\multicolumn{1}{c|}{\cellcolor[HTML]{E2EFDA}0.099} &
		\multicolumn{1}{c|}{\cellcolor[HTML]{E2EFDA}0.028} &
		\multicolumn{1}{c|}{\cellcolor[HTML]{E2EFDA}0.13} &
		\multicolumn{1}{c|}{\cellcolor[HTML]{E2EFDA}3.7} &
		\cellcolor[HTML]{E2EFDA}0.48 \\ \cline{4-10} 
		\multicolumn{1}{l|}{} &
		\multicolumn{1}{c|}{\cellcolor[HTML]{E2EFDA}} &
		\multicolumn{1}{c|}{\multirow{-2}{*}{\cellcolor[HTML]{E2EFDA}\textbf{Letalidad}}} &
		\cellcolor[HTML]{E2EFDA} &
		\multicolumn{1}{c|}{\cellcolor[HTML]{E2EFDA}} &
		\multicolumn{1}{c|}{\cellcolor[HTML]{E2EFDA}} &
		\multicolumn{1}{c|}{\cellcolor[HTML]{E2EFDA}} &
		\multicolumn{1}{c|}{\cellcolor[HTML]{E2EFDA}} &
		\multicolumn{1}{c|}{\cellcolor[HTML]{E2EFDA}} &
		\cellcolor[HTML]{E2EFDA} \\ \cline{3-3}
		\multicolumn{1}{l|}{} &
		\multicolumn{1}{c|}{\cellcolor[HTML]{E2EFDA}} &
		\multicolumn{1}{c|}{\cellcolor[HTML]{E2EFDA}} &
		\multirow{-2}{*}{\cellcolor[HTML]{E2EFDA}\textbf{Defunciones}} &
		\multicolumn{1}{c|}{\multirow{-2}{*}{\cellcolor[HTML]{E2EFDA}07}} &
		\multicolumn{1}{c|}{\multirow{-2}{*}{\cellcolor[HTML]{E2EFDA}02}} &
		\multicolumn{1}{c|}{\multirow{-2}{*}{\cellcolor[HTML]{E2EFDA}04}} &
		\multicolumn{1}{c|}{\multirow{-2}{*}{\cellcolor[HTML]{E2EFDA}37}} &
		\multicolumn{1}{c|}{\multirow{-2}{*}{\cellcolor[HTML]{E2EFDA}193}} &
		\multirow{-2}{*}{\cellcolor[HTML]{E2EFDA}234} \\ \cline{4-10} 
		\multicolumn{1}{l|}{} &
		\multicolumn{1}{c|}{\cellcolor[HTML]{E2EFDA}} &
		\multicolumn{1}{c|}{\multirow{-2}{*}{\cellcolor[HTML]{E2EFDA}\textbf{Mortalidad}}} &
		\cellcolor[HTML]{E2EFDA}\textbf{Tasa *} &
		\multicolumn{1}{c|}{\cellcolor[HTML]{E2EFDA}5.2} &
		\multicolumn{1}{c|}{\cellcolor[HTML]{E2EFDA}1.5} &
		\multicolumn{1}{c|}{\cellcolor[HTML]{E2EFDA}2.9} &
		\multicolumn{1}{c|}{\cellcolor[HTML]{E2EFDA}27} &
		\multicolumn{1}{c|}{\cellcolor[HTML]{E2EFDA}142} &
		\cellcolor[HTML]{E2EFDA}172 \\ \cline{3-10} 
		\multicolumn{1}{l|}{} &
		\multicolumn{1}{c|}{\cellcolor[HTML]{E2EFDA}} &     
		\multicolumn{1}{c|}{\cellcolor[HTML]{E2EFDA}} &
		\cellcolor[HTML]{E2EFDA}\textbf{Casos +} &
		\multicolumn{1}{c|}{\cellcolor[HTML]{E2EFDA}2001} &
		\multicolumn{1}{c|}{\cellcolor[HTML]{E2EFDA}2025} &
		\multicolumn{1}{c|}{\cellcolor[HTML]{E2EFDA}14174} &
		\multicolumn{1}{c|}{\cellcolor[HTML]{E2EFDA}27487} &
		\multicolumn{1}{c|}{\cellcolor[HTML]{E2EFDA}5178} &
		\cellcolor[HTML]{E2EFDA}50865 \\ \cline{4-10} 
		\multicolumn{1}{l|}{} &
		\multicolumn{1}{c|}{\multirow{-6}{*}{\cellcolor[HTML]{E2EFDA}\textbf{2022}}} &
		\multicolumn{1}{c|}{\multirow{-2}{*}{\cellcolor[HTML]{E2EFDA}\textbf{Incidencia}}} &
		\cellcolor[HTML]{E2EFDA}\textbf{Tasa} &
		\multicolumn{1}{c|}{\cellcolor[HTML]{E2EFDA}1474} &
		\multicolumn{1}{c|}{\cellcolor[HTML]{E2EFDA}1492} &
		\multicolumn{1}{c|}{\cellcolor[HTML]{E2EFDA}10441} &
		\multicolumn{1}{c|}{\cellcolor[HTML]{E2EFDA}20248} &
		\multicolumn{1}{c|}{\cellcolor[HTML]{E2EFDA}3814} &
		\cellcolor[HTML]{E2EFDA}37470 \\ \cline{2-10} 
	\end{tabular}
%		}
%	\end{table}	
%	{\tiny Fuente de datos: SISCOVID, NOTICOVID.}
%	
%\end{frame}
%------------------------------------------------------------------------------------------------------------------------------------------------------------------------------------------------------------------------------------------
% SECCIÓN 2: Indicadores Epidemiológicos, Región Cusco
%------------------------------------------------------------------------------------------------------------------------------------------------------------------------------------------------------------------------------------------

\section{Cobertura de Vacunación, Región Cusco}

\begin{frame}[label=cobertura_vacuna]
	\frametitle{Cobertura de Vacunación por Provincias, Región Cusco}
	\vspace{-.7cm}
	\begin{center}
		\includegraphics[width=1.0\linewidth, trim={.2cm .2cm .2cm .2cm},clip]{../sala_nacional/Cobertura_Vacunacion_Provincias.jpg}
	\end{center}
	{\tiny Fuente de datos: SICOVAC - HIS MINSA, Dirección de Estadística GERESA Cusco.} 
\end{frame}

\begin{frame}[label=cobertura_vacuna]
	\frametitle{Porcentaje de Cobertura de Vacunación por Grupo de Edad, Región Cusco}
	\vspace{-.5cm}
	\begin{center}
		\includegraphics[width=0.9\linewidth, trim={.2cm .5cm .2cm .2cm},clip]{../figuras/vacunacion_grupo_edad_dosis.pdf}
	\end{center}
	{\tiny Fuente de datos: SICOVAC - HIS MINSA, Dirección de Estadística GERESA Cusco.} 
\end{frame}

\begin{frame}[label=cobertura_vacuna_provincias]
	\frametitle{Porcentaje de Cobertura de Vacunación de 80 años a más}
	\vspace{-.5cm}
	\begin{center}
		\includegraphics[width=0.85\linewidth, trim={.2cm .5cm .2cm .2cm},clip]{../figuras/vacunacion_provincial_edad_practica_9.pdf}
	\end{center}
	{\tiny Fuente de datos: SICOVAC - HIS MINSA, Dirección de Estadística GERESA Cusco.} \hyperlink{indice}{\beamergotobutton{Índice}}
	
	Ver detalles de estos indicadores para cada grupo de edad de las provincias haciendo clic en los siguientes enlaces:
	\hyperlink{vacunas_90}{\beamergotobutton{80 a Más}}
	\hyperlink{vacunas_80}{\beamergotobutton{70 a 79 años}}
	\hyperlink{vacunas_70}{\beamergotobutton{60 a 69 años}}
	\hyperlink{vacunas_60}{\beamergotobutton{50 a 59 años}} \hyperlink{vacunas_50}{\beamergotobutton{40 a 49 años}} \hyperlink{vacunas_40}{\beamergotobutton{30 a 39 años}} \hyperlink{vacunas_30}{\beamergotobutton{18 a 29 años}}
	\hyperlink{vacunas_20}{\beamergotobutton{12 a 17 años}} \hyperlink{vacunas_10}{\beamergotobutton{5 a 11 años}}
	
\end{frame}



	%-------------------------------------------------------------------------------------------------------------------------------------------------------------------------------------------------\textit{}-----------------------------------------
	% SECCIÓN 2: Indicadores de Gestión Hospitalaria
	%------------------------------------------------------------------------------------------------------------------------------------------------------------------------------------------------------------------------------------------
	\section{Indicadores de Gestión Hospitalaria}
	
	\begin{frame}[label=camas]
		\frametitle{Total y Ocupación de Camas UCI Hospitalarias, 2021-2022. Datos}
		\vspace{-.2cm}
		\begin{center}
			\includegraphics[width=0.9\linewidth, trim={0cm .5cm 0cm 0.2cm},clip]{../figuras/uci.png}		
		\begin{table}[]
			\vspace{-.5cm}
			\resizebox{8 cm}{!}{%
				\begin{tabular}{cccc}
					\hline
					\multicolumn{4}{c}{\textbf{UCIN,   SE 49 	}}                                                   \\ \hline
					TOTAL CAMAS UCIN & CAMAS   UCIN OCUPADAS & CAMAS   UCIN LIBRES & PORCENTAJE   OCUPACIÓN UCIN \\ \hline
					07               & 0                   & 07                  & 0\%                        \\ \hline
				\end{tabular}%
			}
		\end{table}
		\end{center}
		{\tiny Datos UCIN actualizados hasta sem 49} \\
		{\tiny Fuente de datos: Referencias, contrareferencias.} 
		
	\end{frame}
	
	\begin{frame}
		\frametitle{Total y Ocupación de Camas Hospitalarias en el Nivel III, 2021-20222. Datos Actualizados hasta la sem 43.}
		\vspace{-.5cm}
		\begin{center}
			\includegraphics[width=0.8\linewidth, trim={0cm .5cm 0cm 0.2cm},clip]{../figuras/nivel_3.png}
		\end{center}
		{\tiny Fuente de datos: Referencias, contrareferencias.}
	\end{frame}

	\begin{frame}
		\frametitle{Total y Ocupación de Camas Hospitalarias. Hospital Regional, 2021-20222. Datos Actualizados hasta la sem 43.}
		\vspace{-.2cm}
		\begin{center}
			\includegraphics[width=0.9\linewidth, trim={0cm .5cm 0cm 0.2cm},clip]{../figuras/h_regional_uci.png}
		\end{center}
		{\tiny Fuente de datos: Referencias, contrareferencias.}
	\end{frame}
	\begin{frame}
		\frametitle{Total y Ocupación de Camas Hospitalarias. Hospital Regional, 2021-20222.}
		\vspace{-.2cm}
		\begin{center}
	-		\includegraphics[width=0.8\linewidth, trim={0cm .5cm 0cm 0.2cm},clip]{../figuras/h_regional_nouci.png}
			
			\begin{table}[]
				\resizebox{8 cm}{!}{%
					\begin{tabular}{cccc}
						\hline
						\multicolumn{4}{c}{\textbf{UCIN   HOSPITAL REGIONAL, SE 49}}                                 \\ \hline
						TOTAL CAMAS UCIN & CAMAS   UCIN OCUPADAS & CAMAS   UCIN LIBRES & PORCENTAJE   OCUPACIÓN UCIN \\ \hline
						5               & 0                    & 5                  & 0\%                        \\ \hline
					\end{tabular}%
				}
			\end{table}
			
		\end{center}
		{\tiny Datos UCIN actualizados hasta sem 49} \\
		{\tiny Fuente de datos: Referencias, contrareferencias.}
 	\end{frame}
	
	\begin{frame}
		\frametitle{Total y Ocupación de Camas Hospitalarias. Hospital Antonio Lorena, 2021-20222. Datos Actualizados hasta la sem 43.}
		\vspace{-.2cm}
		\begin{center}
			\includegraphics[width=0.77\linewidth, trim={0cm .5cm 0cm 0.2cm},clip]{../figuras/h_lorena_uci.png}				
		\end{center}
		{\tiny Fuente de datos: Referencias, contrareferencias.}
	\end{frame}
	\begin{frame}
		\frametitle{Total y Ocupación de Camas Hospitalarias. Hospital Antonio Lorena, 2021-20222.}
		\vspace{-.2cm}
		\begin{center}
			\includegraphics[width=0.8\linewidth, trim={0cm .5cm 0cm 0.2cm},clip]{../figuras/h_lorena_nouci.png}	
			\begin{table}[]
				\resizebox{8 cm}{!}{%
					\begin{tabular}{cccc}
						\hline
						\multicolumn{4}{c}{\textbf{UCIN   HOSPITAL ANTONIO LORENA, SE 49}}                           \\ \hline
						TOTAL CAMAS UCIN & CAMAS   UCIN OCUPADAS & CAMAS   UCIN LIBRES & PORCENTAJE   OCUPACIÓN UCIN \\ \hline
						2                & 0                    & 2                  & 0\%                        \\ \hline
					\end{tabular}%
				}
			\end{table}
			
		\end{center}
		{\tiny Datos UCIN actualizados hasta sem 49} \\
		{\tiny Fuente de datos: Referencias, contrareferencias.}
	\end{frame}
	
	\begin{frame}
		\frametitle{Total y Ocupación de Camas Hospitalarias, Hospital Nacional Adolfo Guevara Velasco, 2021-20222.}
		\vspace{-.2cm}
		\begin{center}
			\includegraphics[width=0.77\linewidth, trim={0cm .5cm 0cm 0.2cm},clip]{../figuras/h_adolfo_uci.png}
		\end{center}
		{\tiny Fuente de datos: Referencias, contrareferencias}
	\end{frame}
	\begin{frame}
		\frametitle{Total y Ocupación de Camas Hospitalarias, Hospital Nacional Adolfo Guevara Velasco, 2021-20222.}
		\vspace{-.2cm}
		\begin{center}
			\includegraphics[width=0.8\linewidth, trim={0cm .5cm 0cm 0.2cm},clip]{../figuras/h_adolfo_nouci.png}
			
			\begin{table}[]
				\resizebox{8 cm}{!}{%
					\begin{tabular}{cccc}
						\hline
						\multicolumn{4}{c}{\textbf{UCIN   HOSPITAL ADOLFO GUEVARA, SE 49}}                           \\ \hline
						TOTAL CAMAS UCIN & CAMAS   UCIN OCUPADAS & CAMAS   UCIN LIBRES & PORCENTAJE   OCUPACIÓN UCIN \\ \hline
						0               & 0                     & 0                   & 0\%                        \\ \hline
					\end{tabular}%
				}
			\end{table}
			
		\end{center}
		{\tiny Datos UCIN actualizados hasta sem 49} \\
		{\tiny Fuente de datos: Referencias, contrareferencias}
	\end{frame}
	
	\begin{frame}
		\frametitle{Total y Ocupación de Camas Hospitalarias en el Nivel II, 2021-20222.}
		\vspace{-.5cm}
		\begin{center}
			\includegraphics[width=0.8\linewidth, trim={0cm .5cm 0cm 0.2cm},clip]{../figuras/nivel_2.png}
		\end{center}
		{\tiny Fuente de datos: Referencias, contrareferencias.} \hyperlink{indice}{\beamergotobutton{Índice}} 
	\end{frame}
	
	
%-------------------------------------------------------------------------------------------------------------------------------------------------------------------------------------------------\textit{}-----------------------------------------
% SECCIÓN 3: Indicadores por Provincias
%------------------------------------------------------------------------------------------------------------------------------------------------------------------------------------------------------------------------------------------
\section{Indicadores por Provincias}
	\begin{frame}[label=semaforo]
		\frametitle{Tasa de Incidencia por Provincias, 2021-2022}
		\vspace{-.5cm}
		
		% en el input de las tablas sólo debe comenzar y terminar con tabular, borrar el tabular de input de la tabla
		\begin{table}[]
			\resizebox{\textwidth}{!}{%
				\begin{tabular}{lrcclr}
	\rowcolor[HTML]{DCE6F1} 
	\multicolumn{1}{c}{\cellcolor[HTML]{DCE6F1}\textbf{PROVINCIA}} & \multicolumn{1}{c}{\cellcolor[HTML]{DCE6F1}\textbf{Población}} & \textbf{PM+}                                               & \textbf{PA+}         & \multicolumn{1}{c}{\cellcolor[HTML]{DCE6F1}\textbf{Total de casos}} & \multicolumn{1}{c}{\cellcolor[HTML]{DCE6F1}\textbf{Incidencia x 10,000 hab}} \\
	\cellcolor[HTML]{FF5050}CUSCO                                  & 463,656                                                        & 3444                                                       & 25,707               & 29,151                                                              & 628.72                                                                       \\
	\cellcolor[HTML]{F4B084}CANCHIS                                & 105,049                                                        & 125                                                        & 3,225                & 3,350                                                               & 318.90                                                                       \\
	\cellcolor[HTML]{FFFF99}LA   CONVENCION                        & 185,793                                                        & 220                                                        & 4,695                & 4,915                                                               & 264.54                                                                       \\
	\cellcolor[HTML]{FFFF99}URUBAMBA                               & 66,439                                                         & 48                                                         & 1,480                & 1,528                                                               & 229.99                                                                       \\
	\cellcolor[HTML]{FFFF99}ESPINAR                                & 71,304                                                         & 46                                                         & 1,359                & 1,405                                                               & 197.04                                                                       \\
	\cellcolor[HTML]{FFFF99}QUISPICANCHI                           & 92,566                                                         & 148                                                        & 1,514                & 1,662                                                               & 179.55                                                                       \\
	\cellcolor[HTML]{FFFF99}CANAS                                  & 40,420                                                         & 15                                                         & 657                  & 672                                                                 & 166.25                                                                       \\
	\cellcolor[HTML]{FFFF99}ANTA                                   & 57,731                                                         & 76                                                         & 877                  & 953                                                                 & 165.08                                                                       \\
	\cellcolor[HTML]{C6E0B4}CHUMBIVILCAS                           & 84,925                                                         & 161                                                        & 1,196                & 1,357                                                               & 159.79                                                                       \\
	\cellcolor[HTML]{C6E0B4}ACOMAYO                                & 28,477                                                         & 57                                                         & 370                  & 427                                                                 & 149.95                                                                       \\
	\cellcolor[HTML]{C6E0B4}CALCA                                  & 76,462                                                         & 38                                                         & 922                  & 960                                                                 & 125.55                                                                       \\
	\cellcolor[HTML]{C6E0B4}PAUCARTAMBO                            & 52,989                                                         & 74                                                         & 574                  & 648                                                                 & 122.29                                                                       \\
	\cellcolor[HTML]{C6E0B4}PARURO                                 & 31,264                                                         & 80                                                         & 279                  & 359                                                                 & 114.83                                                                       \\
	& \multicolumn{1}{l}{}                                           & \multicolumn{1}{l}{}                                       & \multicolumn{1}{l}{} &                                                                     & \multicolumn{1}{l}{}                                                         \\
	\rowcolor[HTML]{DDEBF7} 
	\textbf{Total   general}                                       & \textbf{1,357,075}                                             & \multicolumn{1}{r}{\cellcolor[HTML]{DDEBF7}\textbf{4,532}} & \textbf{42,855}      & \textbf{47,387}                                                     & \textbf{349.18}                                                             
\end{tabular}
			}
		\end{table}
		{\tiny PA: Prueba Antigénica, PM: Prueba Molecular}\\[0.5 cm]
		{\tiny Fuente de datos: SISCOVID, NOTICOVID.}
	\end{frame}
	
	\begin{frame}
		\frametitle{Tasa de Positividad por Provincias, 2021}
		\vspace{-.5cm}
		
		% en el input de las tablas sólo debe comenzar y terminar con tabular, borrar el tabular de input de la tabla
		\begin{table}[]
			\resizebox{\textwidth}{!}{%
				\begin{tabular}{lrrrr}
	\rowcolor[HTML]{ECF4FF} 
	\textbf{PROVINCIA}                                                 & \multicolumn{1}{l}{\cellcolor[HTML]{ECF4FF}\textbf{Total de pruebas}} & \multicolumn{1}{l}{\cellcolor[HTML]{ECF4FF}\textbf{Tasa de positividad general}} & \multicolumn{1}{l}{\cellcolor[HTML]{ECF4FF}\textbf{Tasa de positividad PM}} & \multicolumn{1}{l}{\cellcolor[HTML]{ECF4FF}\textbf{Tasa de positividad Pruebas AG}} \\
	\cellcolor[HTML]{FD6864}URUBAMBA                                   & 5,314                                                                 & 24.6                                                                             & 33.3                                                                        & 24.4                                                                                \\
	\cellcolor[HTML]{FD6864}CANCHIS                                    & 12,174                                                                & 22.8                                                                             & 16.2                                                                        & 23.2                                                                                \\
	\cellcolor[HTML]{FD6864}CANAS                                      & 2,454                                                                 & 22.6                                                                             & 13.2                                                                        & 23.0                                                                                \\
	\cellcolor[HTML]{FD6864}LA CONVENCION                              & 16,488                                                                & 22.4                                                                             & 21.9                                                                        & 22.5                                                                                \\
	\cellcolor[HTML]{FD6864}CALCA                                      & 3,439                                                                 & 22.2                                                                             & 26.6                                                                        & 22.0                                                                                \\
	\cellcolor[HTML]{FD6864}ANTA                                       & 3,641                                                                 & 21.3                                                                             & 25.6                                                                        & 20.9                                                                                \\
	\cellcolor[HTML]{FD6864}CUSCO                                      & 112,516                                                               & 20.1                                                                             & 20.7                                                                        & 20.1                                                                                \\
	\cellcolor[HTML]{FD6864}ACOMAYO                                    & 1,474                                                                 & 20.1                                                                             & 11.1                                                                        & 20.9                                                                                \\
	\cellcolor[HTML]{FD6864}PAUCARTAMBO                                & 2,612                                                                 & 19.3                                                                             & 14.7                                                                        & 19.9                                                                                \\
	\cellcolor[HTML]{FD6864}QUISPICANCHI                               & 7,262                                                                 & 18.1                                                                             & 15.0                                                                        & 18.5                                                                                \\
	\cellcolor[HTML]{FD6864}CHUMBIVILCAS                               & 5,712                                                                 & 17.9                                                                             & 12.1                                                                        & 18.6                                                                                \\
	\cellcolor[HTML]{FD6864}PARURO                                     & 1,803                                                                 & 15.9                                                                             & 20.5                                                                        & 15.1                                                                                \\
	\cellcolor[HTML]{FD6864}ESPINAR                                    & 8,449                                                                 & 13.7                                                                             & 14.7                                                                        & 13.7                                                                                \\
	&                                                                       &                                                                                  &                                                                             &                                                                                     \\
	\textbf{\begin{tabular}[c]{@{}l@{}}Total\\   general\end{tabular}} & \textbf{183,338}                                                      & \textbf{20.2}                                                                    & \textbf{19.9}                                                               & \textbf{20.3}                                                                      
\end{tabular}
			}
		\end{table}	
		{\tiny Fuente de datos: SISCOVID, NOTICOVID.}
		
	\end{frame}

	\begin{frame}
	\frametitle{Defunciones Cero por Provincias por Semana, 2021-2022}
	\vspace{-.5cm}
	
	% en el input de las tablas sólo debe comenzar y terminar con tabular, borrar el tabular de input de la tabla
	\begin{table}[]
		\resizebox{\textwidth}{!}{%
			\begin{tabular}{lccccccccc}
	\textbf{}              	  & \multicolumn{1}{l}{}                        & \multicolumn{1}{l}{}      & \multicolumn{1}{l}{}                         & \multicolumn{1}{l}{}                         & \multicolumn{1}{l}{}                         & \multicolumn{1}{l}{}                        & \multicolumn{1}{l}{}                         & \multicolumn{1}{l}{}                         & \multicolumn{1}{l}{}     \\
	\textbf{}                                                                               
	&\textbf{SE-03}
	&\textbf{SE-04}								&\textbf{SE-05}	
	&\textbf{SE-06}								&\textbf{SE-07}				&\textbf{SE-08}
	&\textbf{SE-09}								&\textbf{SE-10}
	&\textbf{SE-11}\\
	\textbf{}              	  	
	&\textbf{16ene-22ene}						&\textbf{23ene-29ene}						&\textbf{30ene-05feb}
	&\textbf{05feb-12feb}						&\textbf{13feb-19feb}
	&\textbf{20feb-26feb}						&\textbf{27feb-05mar}
	&\textbf{06mar-12mar}						&\textbf{13mar-19mar}\\
	\textbf{Acomayo}                        	
	&\cellcolor[HTML]{FCC46C}				    &\cellcolor[HTML]{FCC46C}
	&\cellcolor[HTML]{FCC46C}					&\cellcolor[HTML]{FCC46C}
	&\cellcolor[HTML]{FCC46C}					&1
	&\cellcolor[HTML]{FCC46C}					&\cellcolor[HTML]{FCC46C} 
	&\cellcolor[HTML]{FCC46C}\\
	\textbf{Anta}                                                          				
	&\cellcolor[HTML]{FCC46C}					&2 				
	&1											&\cellcolor[HTML]{FCC46C}					&2
	&\cellcolor[HTML]{FCC46C}					&\cellcolor[HTML]{FCC46C}					&1
	&\cellcolor[HTML]{FCC46C}\\
	\textbf{Calca}      				       								            &\cellcolor[HTML]{FCC46C}					&\cellcolor[HTML]{FCC46C}
	&1 											&1	
	&\cellcolor[HTML]{FCC46C}					&1											&\cellcolor[HTML]{FCC46C} 				&1											&1\\             			
	\textbf{Canas}                              		
	&\cellcolor[HTML]{FCC46C}					&1
	&1											&\cellcolor[HTML]{FCC46C}
	&1											&\cellcolor[HTML]{FCC46C}
	&\cellcolor[HTML]{FCC46C}					&\cellcolor[HTML]{FCC46C}
	&\cellcolor[HTML]{FCC46C} \\
	\textbf{Canchis}                             		
	&4											&3
	&2											&4
	&1											&1
	&\cellcolor[HTML]{FCC46C}					&3
	&1\\
	\textbf{Chumbivilcas}                      			
	&\cellcolor[HTML]{FCC46C} 					&1
	&\cellcolor[HTML]{FCC46C}					&3
	&4											&\cellcolor[HTML]{FCC46C}
	&\cellcolor[HTML]{FCC46C}					&\cellcolor[HTML]{FCC46C}
	&1\\
	\textbf{Cusco}                            										
	&11											&9 	
	&14 										&4
	&8											&3
	&1											&1
	&1\\
	\textbf{Espinar}       					             								
	 &\cellcolor[HTML]{FCC46C}
	&1											&1
	&\cellcolor[HTML]{FCC46C}					&1
	&2											&\cellcolor[HTML]{FCC46C}	
	&\cellcolor[HTML]{FCC46C} 					&\cellcolor[HTML]{FCC46C}\\
	\textbf{La Convención}                      					
	&3
	&4											&5
	&2											&3
	&1 											&1 
	&1											&\cellcolor[HTML]{FCC46C}\\
	\textbf{Paruro}                            					
	&1
	&\cellcolor[HTML]{FCC46C}					&\cellcolor[HTML]{FCC46C}
	&\cellcolor[HTML]{FCC46C} 					&1
	&1											&1
	&\cellcolor[HTML]{FCC46C}					&\cellcolor[HTML]{FCC46C}\\
	\textbf{Paucartambo}               		                       					
	&1											&1		
	&1											&\cellcolor[HTML]{FCC46C}
	&\cellcolor[HTML]{FCC46C}
	&\cellcolor[HTML]{FCC46C}					&\cellcolor[HTML]{FCC46C}
	&1											&\cellcolor[HTML]{FCC46C}\\
	\textbf{Quispicanchi}                                         	                 	
	&3											&5
	&4											&\cellcolor[HTML]{FCC46C}
	&1											&\cellcolor[HTML]{FCC46C}
	&1											&\cellcolor[HTML]{FCC46C}
	&\cellcolor[HTML]{FCC46C}\\
	\textbf{Urubamba}                                                          			
	&1
	&\cellcolor[HTML]{FCC46C}					&3
	&\cellcolor[HTML]{FCC46C}					&1
	&2											&\cellcolor[HTML]{FCC46C}
	&\cellcolor[HTML]{FCC46C}					&\cellcolor[HTML]{FCC46C}\\	
	&\multicolumn{1}{l}{}                       &\multicolumn{1}{l}{}            &\multicolumn{1}{l}{}                         
	&\multicolumn{1}{l}{}                       &\multicolumn{1}{l}{}            &\multicolumn{1}{l}{}                       &\multicolumn{1}{l}{}                       &\multicolumn{1}{l}{}            &\multicolumn{1}{l}{}    
\end{tabular}
		}
	\end{table}	
	{\tiny Fuente de datos: SINADEF - NOTICOVID. \\}
	\vspace{0.2cm}
	$\rightarrow$ Para la SE 50, la provincia del \textbf{\color{mycolor3}Cusco} registró \textbf{\color{mycolor3}cero} defunciónes por COVID-19.\\
	$\rightarrow$ La Provincia de La Convecnión y Quispicanchis provincias registraron \textbf{\color{mycolor3}una} defunción por COVID-19.\\
	%		$\rightarrow$ Las demás provincias de la región del Cusco registraron \textbf{\color{mycolor5}cero} defunciónes por COVID-19.
\end{frame}

	\begin{frame}
	\frametitle{Tabla Resumen de Indicadores COVID-19, 2020-2022}
	\vspace{-.5cm}
	
	% en el input de las tablas sólo debe comenzar y terminar con tabular, borrar el tabular de input de la tabla
	\begin{table}[]
		\resizebox{\textwidth}{!}{%
				\begin{tabular}{lccc|cccccc|}
		\cline{5-10}
		&
		\multicolumn{1}{l}{} &
		&
		&
		\multicolumn{6}{c|}{\cellcolor[HTML]{F2F2F2}} \\
		&
		\multicolumn{1}{l}{} &
		\multicolumn{1}{l}{} &
		\multicolumn{1}{l|}{} &
		\multicolumn{6}{c|}{\multirow{-2}{*}{\cellcolor[HTML]{F2F2F2}\textbf{Etapa de Vida}}} \\ \cline{5-10} 
		&
		\multicolumn{1}{l}{} &
		\multicolumn{1}{l}{} &
		\multicolumn{1}{l|}{} &
		\multicolumn{1}{c|}{\cellcolor[HTML]{F2F2F2}\textbf{Niño}} &
		\multicolumn{1}{l|}{\cellcolor[HTML]{F2F2F2}\textbf{Adolescente}} &
		\multicolumn{1}{l|}{\cellcolor[HTML]{F2F2F2}\textbf{Joven}} &
		\multicolumn{1}{l|}{\cellcolor[HTML]{F2F2F2}\textbf{Adulto}} &
		\multicolumn{1}{l|}{\cellcolor[HTML]{F2F2F2}\textbf{Adulto Mayor}} &
		\cellcolor[HTML]{F2F2F2}\textbf{Total} \\ \cline{2-10} 
		\multicolumn{1}{l|}{} &
		\multicolumn{1}{c|}{\cellcolor[HTML]{ECF4FF}} &
		\multicolumn{1}{c|}{\cellcolor[HTML]{ECF4FF}} &
		\cellcolor[HTML]{ECF4FF}\textbf{Tasa (\%)} &
		\multicolumn{1}{c|}{\cellcolor[HTML]{ECF4FF}0.4} &
		\multicolumn{1}{c|}{\cellcolor[HTML]{ECF4FF}0.049} &
		\multicolumn{1}{c|}{\cellcolor[HTML]{ECF4FF}0.12} &
		\multicolumn{1}{c|}{\cellcolor[HTML]{ECF4FF}0.57} &
		\multicolumn{1}{c|}{\cellcolor[HTML]{ECF4FF}7.9} &
		\cellcolor[HTML]{ECF4FF}1.3 \\ \cline{4-10} 
		\multicolumn{1}{l|}{} &
		\multicolumn{1}{c|}{\cellcolor[HTML]{ECF4FF}} &
		\multicolumn{1}{c|}{\multirow{-2}{*}{\cellcolor[HTML]{ECF4FF}\textbf{Letalidad}}} &
		\cellcolor[HTML]{ECF4FF} &
		\multicolumn{1}{c|}{\cellcolor[HTML]{ECF4FF}} &
		\multicolumn{1}{c|}{\cellcolor[HTML]{ECF4FF}} &
		\multicolumn{1}{c|}{\cellcolor[HTML]{ECF4FF}} &
		\multicolumn{1}{c|}{\cellcolor[HTML]{ECF4FF}} &
		\multicolumn{1}{c|}{\cellcolor[HTML]{ECF4FF}} &
		\cellcolor[HTML]{ECF4FF} \\ \cline{3-3}
		\multicolumn{1}{l|}{} &
		\multicolumn{1}{c|}{\cellcolor[HTML]{ECF4FF}} &
		\multicolumn{1}{c|}{\cellcolor[HTML]{ECF4FF}} &
		\multirow{-2}{*}{\cellcolor[HTML]{ECF4FF}\textbf{Defunciones}} &
		\multicolumn{1}{c|}{\multirow{-2}{*}{\cellcolor[HTML]{ECF4FF}07}} &
		\multicolumn{1}{c|}{\multirow{-2}{*}{\cellcolor[HTML]{ECF4FF}01}} &
		\multicolumn{1}{c|}{\multirow{-2}{*}{\cellcolor[HTML]{ECF4FF}29}} &
		\multicolumn{1}{c|}{\multirow{-2}{*}{\cellcolor[HTML]{ECF4FF}375}} &
		\multicolumn{1}{c|}{\multirow{-2}{*}{\cellcolor[HTML]{ECF4FF}973}} &
		\multirow{-2}{*}{\cellcolor[HTML]{ECF4FF}1385} \\ \cline{4-10} 
		\multicolumn{1}{l|}{} &
		\multicolumn{1}{c|}{\cellcolor[HTML]{ECF4FF}} &
		\multicolumn{1}{c|}{\multirow{-2}{*}{\cellcolor[HTML]{ECF4FF}\textbf{Mortalidad}}} &
		\cellcolor[HTML]{ECF4FF}\textbf{Tasa*} &
		\multicolumn{1}{c|}{\cellcolor[HTML]{ECF4FF}5.2} &
		\multicolumn{1}{c|}{\cellcolor[HTML]{ECF4FF}0.74} &
		\multicolumn{1}{c|}{\cellcolor[HTML]{ECF4FF}21} &
		\multicolumn{1}{c|}{\cellcolor[HTML]{ECF4FF}276} &
		\multicolumn{1}{c|}{\cellcolor[HTML]{ECF4FF}717} &
		\cellcolor[HTML]{ECF4FF}1020 \\ \cline{3-10} 
		\multicolumn{1}{l|}{} &
		\multicolumn{1}{c|}{\cellcolor[HTML]{ECF4FF}} &
		\multicolumn{1}{c|}{\cellcolor[HTML]{ECF4FF}} &
		\cellcolor[HTML]{ECF4FF}\textbf{Casos +} &
		\multicolumn{1}{c|}{\cellcolor[HTML]{ECF4FF}1749} &
		\multicolumn{1}{c|}{\cellcolor[HTML]{ECF4FF}2029} &
		\multicolumn{1}{c|}{\cellcolor[HTML]{ECF4FF}25091} &
		\multicolumn{1}{c|}{\cellcolor[HTML]{ECF4FF}66024} &
		\multicolumn{1}{c|}{\cellcolor[HTML]{ECF4FF}12255} &
		\cellcolor[HTML]{ECF4FF}107148 \\ \cline{4-10} 
		\multicolumn{1}{l|}{} &
		\multicolumn{1}{c|}{\multirow{-6}{*}{\cellcolor[HTML]{ECF4FF}\textbf{2020}}} &
		\multicolumn{1}{c|}{\multirow{-2}{*}{\cellcolor[HTML]{ECF4FF}\textbf{Incidencia}}} &
		\cellcolor[HTML]{ECF4FF}\textbf{Tasa*} &
		\multicolumn{1}{c|}{\cellcolor[HTML]{ECF4FF}1288} &
		\multicolumn{1}{c|}{\cellcolor[HTML]{ECF4FF}1495} &
		\multicolumn{1}{c|}{\cellcolor[HTML]{ECF4FF}18483} &
		\multicolumn{1}{c|}{\cellcolor[HTML]{ECF4FF}48637} &
		\multicolumn{1}{c|}{\cellcolor[HTML]{ECF4FF}9028} &
		\cellcolor[HTML]{ECF4FF}7.9 \\ \cline{2-10} 
		\multicolumn{1}{l|}{} &
		\multicolumn{1}{c|}{\cellcolor[HTML]{FFFFC7}} &
		\multicolumn{1}{c|}{\cellcolor[HTML]{FFFFC7}} &
		\cellcolor[HTML]{FFFFC7}\textbf{Tasa (\%)} &
		\multicolumn{1}{c|}{\cellcolor[HTML]{FFFFC7}0.94} &
		\multicolumn{1}{c|}{\cellcolor[HTML]{FFFFC7}0.087} &
		\multicolumn{1}{c|}{\cellcolor[HTML]{FFFFC7}0.13} &
		\multicolumn{1}{c|}{\cellcolor[HTML]{FFFFC7}1.9} &
		\multicolumn{1}{c|}{\cellcolor[HTML]{FFFFC7}19} &
		\cellcolor[HTML]{FFFFC7}3.8 \\ \cline{4-10} 
		\multicolumn{1}{l|}{} &
		\multicolumn{1}{c|}{\cellcolor[HTML]{FFFFC7}} &
		\multicolumn{1}{c|}{\multirow{-2}{*}{\cellcolor[HTML]{FFFFC7}\textbf{Letalidad}}} &
		\cellcolor[HTML]{FFFFC7} &
		\multicolumn{1}{c|}{\cellcolor[HTML]{FFFFC7}} &
		\multicolumn{1}{c|}{\cellcolor[HTML]{FFFFC7}} &
		\multicolumn{1}{c|}{\cellcolor[HTML]{FFFFC7}} &
		\multicolumn{1}{c|}{\cellcolor[HTML]{FFFFC7}} &
		\multicolumn{1}{c|}{\cellcolor[HTML]{FFFFC7}} &
		\cellcolor[HTML]{FFFFC7} \\ \cline{3-3}
		\multicolumn{1}{l|}{} &
		\multicolumn{1}{c|}{\cellcolor[HTML]{FFFFC7}} &
		\multicolumn{1}{c|}{\cellcolor[HTML]{FFFFC7}} &
		\multirow{-2}{*}{\cellcolor[HTML]{FFFFC7}\textbf{Defunciones}} &
		\multicolumn{1}{c|}{\multirow{-2}{*}{\cellcolor[HTML]{FFFFC7}11}} &
		\multicolumn{1}{c|}{\multirow{-2}{*}{\cellcolor[HTML]{FFFFC7}04}} &
		\multicolumn{1}{c|}{\multirow{-2}{*}{\cellcolor[HTML]{FFFFC7}25}} &
		\multicolumn{1}{c|}{\multirow{-2}{*}{\cellcolor[HTML]{FFFFC7}826}} &
		\multicolumn{1}{c|}{\multirow{-2}{*}{\cellcolor[HTML]{FFFFC7}2127}} &
		\multirow{-2}{*}{\cellcolor[HTML]{FFFFC7}2993} \\ \cline{4-10} 
		\multicolumn{1}{l|}{} &
		\multicolumn{1}{c|}{\cellcolor[HTML]{FFFFC7}} &
		\multicolumn{1}{c|}{\multirow{-2}{*}{\cellcolor[HTML]{FFFFC7}\textbf{Mortalidad}}} &
		\cellcolor[HTML]{FFFFC7}\textbf{Tasa*} &
		\multicolumn{1}{c|}{\cellcolor[HTML]{FFFFC7}8.1} &
		\multicolumn{1}{c|}{\cellcolor[HTML]{FFFFC7}2.9} &
		\multicolumn{1}{c|}{\cellcolor[HTML]{FFFFC7}18} &
		\multicolumn{1}{c|}{\cellcolor[HTML]{FFFFC7}608} &
		\multicolumn{1}{c|}{\cellcolor[HTML]{FFFFC7}1567} &
		\cellcolor[HTML]{FFFFC7}2205 \\ \cline{3-10} 
		\multicolumn{1}{l|}{} &
		\multicolumn{1}{c|}{\cellcolor[HTML]{FFFFC7}} &
		\multicolumn{1}{c|}{\cellcolor[HTML]{FFFFC7}} &
		\cellcolor[HTML]{FFFFC7}\textbf{Casos +} &
		\multicolumn{1}{c|}{\cellcolor[HTML]{FFFFC7}1173} &
		\multicolumn{1}{c|}{\cellcolor[HTML]{FFFFC7}4573} &
		\multicolumn{1}{c|}{\cellcolor[HTML]{FFFFC7}19526} &
		\multicolumn{1}{c|}{\cellcolor[HTML]{FFFFC7}43215} &
		\multicolumn{1}{c|}{\cellcolor[HTML]{FFFFC7}11129} &
		\cellcolor[HTML]{FFFFC7}79616 \\ \cline{4-10} 
		\multicolumn{1}{l|}{} &
		\multicolumn{1}{c|}{\multirow{-6}{*}{\cellcolor[HTML]{FFFFC7}\textbf{2021}}} &
		\multicolumn{1}{c|}{\multirow{-2}{*}{\cellcolor[HTML]{FFFFC7}\textbf{Incidencia}}} &
		\cellcolor[HTML]{FFFFC7}\textbf{Tasa*} &
		\multicolumn{1}{c|}{\cellcolor[HTML]{FFFFC7}864} &
		\multicolumn{1}{c|}{\cellcolor[HTML]{FFFFC7}3369} &
		\multicolumn{1}{c|}{\cellcolor[HTML]{FFFFC7}14304} &
		\multicolumn{1}{c|}{\cellcolor[HTML]{FFFFC7}31834} &
		\multicolumn{1}{c|}{\cellcolor[HTML]{FFFFC7}8198} &
		\cellcolor[HTML]{FFFFC7}58649 \\ \cline{2-10} 
		\multicolumn{1}{l|}{} &
		\multicolumn{1}{c|}{\cellcolor[HTML]{E2EFDA}} &
		\multicolumn{1}{c|}{\cellcolor[HTML]{E2EFDA}} &
		\cellcolor[HTML]{E2EFDA}\textbf{Tasa(\%)} &
		\multicolumn{1}{c|}{\cellcolor[HTML]{E2EFDA}0.35} &
		\multicolumn{1}{c|}{\cellcolor[HTML]{E2EFDA}0.099} &
		\multicolumn{1}{c|}{\cellcolor[HTML]{E2EFDA}0.028} &
		\multicolumn{1}{c|}{\cellcolor[HTML]{E2EFDA}0.13} &
		\multicolumn{1}{c|}{\cellcolor[HTML]{E2EFDA}3.7} &
		\cellcolor[HTML]{E2EFDA}0.48 \\ \cline{4-10} 
		\multicolumn{1}{l|}{} &
		\multicolumn{1}{c|}{\cellcolor[HTML]{E2EFDA}} &
		\multicolumn{1}{c|}{\multirow{-2}{*}{\cellcolor[HTML]{E2EFDA}\textbf{Letalidad}}} &
		\cellcolor[HTML]{E2EFDA} &
		\multicolumn{1}{c|}{\cellcolor[HTML]{E2EFDA}} &
		\multicolumn{1}{c|}{\cellcolor[HTML]{E2EFDA}} &
		\multicolumn{1}{c|}{\cellcolor[HTML]{E2EFDA}} &
		\multicolumn{1}{c|}{\cellcolor[HTML]{E2EFDA}} &
		\multicolumn{1}{c|}{\cellcolor[HTML]{E2EFDA}} &
		\cellcolor[HTML]{E2EFDA} \\ \cline{3-3}
		\multicolumn{1}{l|}{} &
		\multicolumn{1}{c|}{\cellcolor[HTML]{E2EFDA}} &
		\multicolumn{1}{c|}{\cellcolor[HTML]{E2EFDA}} &
		\multirow{-2}{*}{\cellcolor[HTML]{E2EFDA}\textbf{Defunciones}} &
		\multicolumn{1}{c|}{\multirow{-2}{*}{\cellcolor[HTML]{E2EFDA}07}} &
		\multicolumn{1}{c|}{\multirow{-2}{*}{\cellcolor[HTML]{E2EFDA}02}} &
		\multicolumn{1}{c|}{\multirow{-2}{*}{\cellcolor[HTML]{E2EFDA}04}} &
		\multicolumn{1}{c|}{\multirow{-2}{*}{\cellcolor[HTML]{E2EFDA}37}} &
		\multicolumn{1}{c|}{\multirow{-2}{*}{\cellcolor[HTML]{E2EFDA}193}} &
		\multirow{-2}{*}{\cellcolor[HTML]{E2EFDA}234} \\ \cline{4-10} 
		\multicolumn{1}{l|}{} &
		\multicolumn{1}{c|}{\cellcolor[HTML]{E2EFDA}} &
		\multicolumn{1}{c|}{\multirow{-2}{*}{\cellcolor[HTML]{E2EFDA}\textbf{Mortalidad}}} &
		\cellcolor[HTML]{E2EFDA}\textbf{Tasa *} &
		\multicolumn{1}{c|}{\cellcolor[HTML]{E2EFDA}5.2} &
		\multicolumn{1}{c|}{\cellcolor[HTML]{E2EFDA}1.5} &
		\multicolumn{1}{c|}{\cellcolor[HTML]{E2EFDA}2.9} &
		\multicolumn{1}{c|}{\cellcolor[HTML]{E2EFDA}27} &
		\multicolumn{1}{c|}{\cellcolor[HTML]{E2EFDA}142} &
		\cellcolor[HTML]{E2EFDA}172 \\ \cline{3-10} 
		\multicolumn{1}{l|}{} &
		\multicolumn{1}{c|}{\cellcolor[HTML]{E2EFDA}} &     
		\multicolumn{1}{c|}{\cellcolor[HTML]{E2EFDA}} &
		\cellcolor[HTML]{E2EFDA}\textbf{Casos +} &
		\multicolumn{1}{c|}{\cellcolor[HTML]{E2EFDA}2001} &
		\multicolumn{1}{c|}{\cellcolor[HTML]{E2EFDA}2025} &
		\multicolumn{1}{c|}{\cellcolor[HTML]{E2EFDA}14174} &
		\multicolumn{1}{c|}{\cellcolor[HTML]{E2EFDA}27487} &
		\multicolumn{1}{c|}{\cellcolor[HTML]{E2EFDA}5178} &
		\cellcolor[HTML]{E2EFDA}50865 \\ \cline{4-10} 
		\multicolumn{1}{l|}{} &
		\multicolumn{1}{c|}{\multirow{-6}{*}{\cellcolor[HTML]{E2EFDA}\textbf{2022}}} &
		\multicolumn{1}{c|}{\multirow{-2}{*}{\cellcolor[HTML]{E2EFDA}\textbf{Incidencia}}} &
		\cellcolor[HTML]{E2EFDA}\textbf{Tasa} &
		\multicolumn{1}{c|}{\cellcolor[HTML]{E2EFDA}1474} &
		\multicolumn{1}{c|}{\cellcolor[HTML]{E2EFDA}1492} &
		\multicolumn{1}{c|}{\cellcolor[HTML]{E2EFDA}10441} &
		\multicolumn{1}{c|}{\cellcolor[HTML]{E2EFDA}20248} &
		\multicolumn{1}{c|}{\cellcolor[HTML]{E2EFDA}3814} &
		\cellcolor[HTML]{E2EFDA}37470 \\ \cline{2-10} 
	\end{tabular}
		}
	\end{table}	
	{\tiny Fuente de datos: NOTICOVID, SINADEF,SISCOVID. \\}
	\vspace{0.2cm}
	$\rightarrow$ Tasa de mortalidad ajustada 1 000 000 habitantes* \\
	$\rightarrow$ Tasa de incidencia ajustada 1 000 000 habitantes* \\
	%		$\rightarrow$ Las demás provincias de la región del Cusco registraron \textbf{\color{mycolor5}cero} defunciónes por COVID-19.
\end{frame}
	

	
	\begin{frame}[label=indicadores_provinciales]
		\frametitle{Tasa de Letalidad y Mortalidad, 2022}
		\vspace{-.5cm}
		
		% en el input de las tablas sólo debe comenzar y terminar con tabular, borrar el tabular de input de la tabla
		\begin{table}[]
			\resizebox{\textwidth}{!}{%
				\begin{tabular}{lrrrrr}
	\rowcolor[HTML]{ECF4FF} 
	\textbf{Provincias}                   & \multicolumn{1}{l}{\cellcolor[HTML]{ECF4FF}\textbf{Población}} & \multicolumn{1}{l}{\cellcolor[HTML]{ECF4FF}\textbf{Pruebas Totales}} & \multicolumn{1}{l}{\cellcolor[HTML]{ECF4FF}\textbf{Defunciones}} & \multicolumn{1}{l}{\cellcolor[HTML]{ECF4FF}\textbf{Tasa de letalidad}} & \multicolumn{1}{l}{\cellcolor[HTML]{ECF4FF}\textbf{\begin{tabular}[c]{@{}l@{}}Tasa de mortalidad x \\   100.000 hab\end{tabular}}} \\
	\cellcolor[HTML]{FD6864}CANCHIS       & 105,049                                                        & 2,599                                                                & 19                                                             & 0.7\%                                                                  & 18.1                                                                                                                               \\
	\cellcolor[HTML]{FD6864}QUISPICANCHI  & 92,566                                                         & 1,192                                                                & 14                                                             & 1.2\%                                                                  & 15.1                                                                                                                               \\
	\cellcolor[HTML]{FFCE93}LA CONVENCIÓN & 185,793                                                        & 3,565                                                                & 22                                                             & 0.6\%                                                                  & 11.8                                                                                                                               \\
	\cellcolor[HTML]{FFCE93}CUSCO         & 463,656                                                        & 21,550                                                               & 50                                                             & 0.2\%                                                                  & 10.8                                                                                                                               \\
	\cellcolor[HTML]{FFFC9E}URUBAMBA      & 66,439                                                         & 1,230                                                                & 6                                                              & 0.5\%                                                                  & 9.0                                                                                                                                \\
	\cellcolor[HTML]{FFFC9E}PAUCARTAMBO   & 52,989                                                         & 465                                                                  & 4                                                              & 0.9\%                                                                  & 7.5                                                                                                                                \\
	\cellcolor[HTML]{FFFC9E}CHUMBIVILCAS  & 84,925                                                         & 905                                                                  & 6                                                              & 0.7\%                                                                  & 7.1                                                                                                                                \\
	\cellcolor[HTML]{FFFC9E}ANTA          & 57,731                                                         & 732                                                                  & 4                                                              & 0.5\%                                                                  & 6.9                                                                                                                                \\
	\cellcolor[HTML]{FFFC9E}ESPINAR       & 71,304                                                         & 937                                                                  & 4                                                              & 0.4\%                                                                  & 5.6                                                                                                                                \\
	\cellcolor[HTML]{FFFC9E}CALCA         & 76,462                                                         & 713                                                                  & 4                                                              & 0.6\%                                                                  & 5.2                                                                                                                                \\
	\cellcolor[HTML]{FFFC9E}CANAS         & 40,420                                                         & 488                                                                  & 2                                                              & 0.4\%                                                                  & 4.9                                                                                                                                \\
	\cellcolor[HTML]{9AFF99}ACOMAYO       & 28,477                                                         & 273                                                                  & 1                                                              & 0.4\%                                                                  & 3.5                                                                                                                                \\
	\cellcolor[HTML]{9AFF99}PARURO        & 31,264                                                         & 224                                                                  & 1                                                              & 0.4\%                                                                  & 3.2                                                                                                                                \\
	&                                                                &                                                                      &                                                                &                                                                        &                                                                                                                                    \\
	\rowcolor[HTML]{ECF4FF} 
	\textbf{Totales generales}            & \textbf{1,357,075}                                             & \textbf{34,873}                                                      & \textbf{137}                                                   & \textbf{0,39\%}                                                        & \textbf{10.1}                                                                                                                     
\end{tabular}
			}
		\end{table}
		{\tiny Fuente de datos: SINADEF - NOTICOVID. \hyperlink{indice}{\beamergotobutton{Índice}} \\} 
		
		Ver detalles de la tendencia (2020 y 2021-2022) de estos indicadores para cada provincia haciendo clic en los siguientes enlaces:\\ \hyperlink{Acomayo}{\beamergotobutton{Acomayo}} \hyperlink{Anta}{\beamergotobutton{Anta}} \hyperlink{Calca}{\beamergotobutton{Calca}} \hyperlink{Canas}{\beamergotobutton{Canas}} \hyperlink{Chumbivilcas}{\beamergotobutton{Chimbivilcas}}
		\hyperlink{Canchis}{\beamergotobutton{Canchis}} \hyperlink{Cusco}{\beamergotobutton{Cusco}}
		\hyperlink{Espinar}{\beamergotobutton{Espinar}}
		\hyperlink{laconvencion}{\beamergotobutton{La Convencion}}
		\hyperlink{Paruro}{\beamergotobutton{Paruro}} \hyperlink{Paucartambo}{\beamergotobutton{Paucartambo}}	
		\hyperlink{Quispicanchi}{\beamergotobutton{Quispicanchi}}
		\hyperlink{Urubamba}{\beamergotobutton{Urubamba}}
	\end{frame}

%-------------------------------------------------------------------------------------------------------------------------------------------------------------------------------------------------\textit{}-----------------------------------------
% SECCIÓN 4: Resúmen y Recomendaciones
%------------------------------------------------------------------------------------------------------------------------------------------------------------------------------------------------------------------------------------------
\section{Resumen}
\begin{frame}[label=Resumen]
	\frametitle{Análisis Situacional por COVID-19: Resumen}
	\vspace{-.5cm}
	\begin{itemize}
		\item La Región de Cusco se ubica en el \textbf{\color{mycolor4}octavo lugar} de \textbf{\color{mycolor3}mortalidad} acumulada a nivel nacional, con un índice de propagación de $0.88 $.   
		\item A la presente SE 49, se mantiene un \textbf{\color{mycolor4} incremento} en el número de \textbf{\color{mycolor3}casos}, así como de hospitalizaciones, las defunciones se encuentran aún bajas.
		\item La \textbf{\color{mycolor3}tasa de positividad para pruebas moleculares} \textbf{\color{mycolor4}ascendió} a $24.9\%$ y la \textbf{\color{mycolor3}tasa de positividad de pruebas antigénicas} ascendió a 36.3$\%$ (un 15$\%$ más). \end{itemize}
\end{frame}

\begin{frame}
	\frametitle{Análisis Situacional por COVID-19: Resumen}
	\vspace{-.5cm}
	\begin{itemize}
	\item La ocupación de camas \textbf{\color{mycolor3} no UCI} \textbf{\color{mycolor3}ascendió al $74\%$ a nivel regional, un $30\%$ más a la semana previa}.
	\item La ocupación de camas \textbf{\color{mycolor3} UCI}, \textbf{\color{mycolor4}ascendió al $31\%$, un $6\%$ más a la semana previa}. 
	\item Para la SE 49, predomina la variante ómicron en el $100\%$, del total de muestras secuenciadas por semana, manteniéndose la presencia de sublinajes como XBB y BQ1 y BQ1.1.
	\item La \textbf{\color{mycolor3}cobertura de vacunación regional} para COVID-19, muestra una \textbf{\color{mycolor4}cobertura por encima del $80\%$ para los grupos etarios mayores a 50 años}, con una brecha más amplia en los grupos etarios menores a 39 años.	
	\end{itemize}
\end{frame}

\begin{frame}
	\frametitle{Análisis Situacional por COVID-19: Resumen}
	\vspace{-.5cm}
	\begin{itemize}
		\item Semáforo epidemiológico regional
		\begin{itemize}
			\item Tasa de crecimiento semanal de \textbf{\color{mycolor4}casos} sintomáticos se encuentra en \textbf{\color{mycolor3}ámbar}.
			\item Tasa de crecimiento semanal de \textbf{\color{mycolor4}defunciones} en \textbf{\color{mycolor3}verde}.
			\item Tasa de \textbf{\color{mycolor4}positividad} de  \textbf{\color{mycolor4}pruebas moleculares} en \textbf{\color{mycolor3}ámbar}, mientras que la de \textbf{\color{mycolor4}pruebas antigénicas} en \textbf{\color{mycolor3}rojo}
			\item Disponibilidad de \textbf{\color{mycolor4}camas hospitalarias}: no UCI: en \textbf{\color{mycolor3}ámbar}.
			\item Disponibilidad de camas hospitalarias: \textbf{\color{mycolor4}UCI} en \textbf{\color{mycolor3}rojo}.
		\end{itemize} 
		\item Semáforo epidemiológico a nivel provincial
		\begin{itemize}
			\item Para la SE49, la provincia de Cusco, la Convención, Quispicanchis y Canchis mantienen el incremento de casos. Las defunciones aun se encuentran bajas.
		\end{itemize}
	\end{itemize}
\end{frame}
%
\begin{frame}[label=recomendaciones]
	\frametitle{Análisis Situacional por COVID-19: Recomendaciones}
	\vspace{-.5cm}
	\begin{itemize}
			\item Reforzar medidas de control por comandos C19 provinciales y distritales, con medidas de fiscalización
			\item Mejorar las brechas de vacunación, con énfasis en la 3ra y 4ta dosis.
			\item Énfasis en distanciamiento social, mascarilla y ambientes ventilados
			\item Campañas de comunicación (mensajes claros, consistentes y constantes)
			\item Aislamiento temprano de casos y contactos
			\item Conservar la burbuja familiar. 
			\item Facilitar la búsqueda activa de casos y contactos
			\item Fortalecer el seguimiento clínico de casos			
	\end{itemize} 
\end{frame}

\begin{frame}[label=links]
	\frametitle{Links Útiles}
	\vspace{-.5cm}
	\begin{itemize}
		\item Encuentre {\color{mycolor4} información de la pandemia actualizada diaria a nivel regional, provincial, y distrital} en nuestro {\color{mycolor4}\textbf{Dashboard GERESA}} haciendo clic \href{https://geresacusc.shinyapps.io/GERESA_dashboard/}{\color{mycolor2}aquí}.
		\item Encuentre {\color{mycolor4} información de la {\color{mycolor4}\textbf{COVID-19}} actualizada} en nuestro DASHBOARD haciendo clic \href{https://sites.google.com/view/geresacusco/inicio}{\color{mycolor2}aquí}.
		\item Encuentre información actualizada de los {\color{mycolor4}Mapas de Calor COVID-19} en el haciendo clic \href{http://www.diresacusco.gob.pe/diresa/}{\color{mycolor2}aquí}.
		\item Encuentre información diaria del {\color{mycolor4} Resumen de la Sala Situacional COVID-19} de la Región haciendo clic \href{https://app.powerbi.com/view?r=eyJrIjoiZDdiMzA4YWMtZTZmNC00ZWE2LWFmMmYtODkwZmM1ODhiYTljIiwidCI6IjM2NGE0NmEwLTk0YzctNGZkNi1iYTNjLTlmMmQzMjA5YzFlZiJ9}{\color{mycolor2}aquí}.
		\item Encuentre información resumen cuatro semanas de la situación epidemiológica de COVID-19 en los {\color{mycolor4}Boletines COVID-19} haciendo clic \href{https://sites.google.com/view/geresacusco/boletines-epidemiologicos-covid-19}{\color{mycolor2}aquí}. \hyperlink{indice}{\beamergotobutton{Índice}}
	\end{itemize}
\end{frame}


%------------------------------------------------------------------------------------------------------------------------------------------------------------------------------------------------------------------------------------------
% APÉNDICE
%------------------------------------------------------------------------------------------------------------------------------------------------------------------------------------------------------------------------------------------

%\backupbegin

%\setbeamercovered{invisible}
%\begin{frame}[plain,noframenumbering]
%	\titlepage
%\end{frame}

\appendix
\section{Apéndice}

\subsection{Vacunación por Provincias y Grupo de Edad}

\begin{frame}[label=vacunas_90]
	\frametitle{Porcentaje de Cobertura de Vacunación de 80 a Más}
	\vspace{-.5cm}
	\begin{center}
		\includegraphics[width=1.0\linewidth, trim={.2cm .5cm .2cm .2cm},clip]{../figuras/vacunacion_provincial_edad_practica_9.pdf}
	\end{center}
	{\tiny Fuente de datos: SICOVAC - HIS MINSA, Dirección de Estadística GERESA Cusco. \\}
	\hyperlink{cobertura_vacuna_provincias}{\beamergotobutton{regresar}}
\end{frame}

\begin{frame}[label=vacunas_80]
	\frametitle{Porcentaje de Cobertura de Vacunación de 70 a 79 años}
	\vspace{-.5cm}
	\begin{center}
		\includegraphics[width=1.0\linewidth, trim={.2cm .5cm .2cm .2cm},clip]{../figuras/vacunacion_provincial_edad_practica_8.pdf}
	\end{center}
	{\tiny Fuente de datos: SICOVAC - HIS MINSA, Dirección de Estadística GERESA Cusco. \\}
	\hyperlink{cobertura_vacuna_provincias}{\beamergotobutton{regresar}}
\end{frame}

\begin{frame}[label=vacunas_70]
	\frametitle{Porcentaje de Cobertura de Vacunación de 60 a 69 años}
	\vspace{-.5cm}
	\begin{center}
		\includegraphics[width=1.0\linewidth, trim={.2cm .5cm .2cm .2cm},clip]{../figuras/vacunacion_provincial_edad_practica_7.pdf}
	\end{center}
	{\tiny Fuente de datos: SICOVAC - HIS MINSA, Dirección de Estadística GERESA Cusco. \\}
\hyperlink{cobertura_vacuna_provincias}{\beamergotobutton{regresar}}
\end{frame}

\begin{frame}[label=vacunas_60]
	\frametitle{Porcentaje de Cobertura de Vacunación de 50 a 59 años, Provincias de Cusco}
	\vspace{-.5cm}
	\begin{center}
		\includegraphics[width=1.0\linewidth, trim={.2cm .5cm .2cm .2cm},clip]{../figuras/vacunacion_provincial_edad_practica_6.pdf}
	\end{center}
	{\tiny Fuente de datos: SICOVAC - HIS MINSA, Dirección de Estadística GERESA Cusco. \\}
	\hyperlink{cobertura_vacuna_provincias}{\beamergotobutton{regresar}}
\end{frame}

\begin{frame}[label=vacunas_50]
	\frametitle{Porcentaje de Cobertura de Vacunación de 40 a 49 años, Provincias de Cusco}
	\vspace{-.5cm}
	\begin{center}
		\includegraphics[width=1.0\linewidth, trim={.2cm .5cm .2cm .2cm},clip]{../figuras/vacunacion_provincial_edad_practica_5.pdf}
	\end{center}
	{\tiny Fuente de datos: SICOVAC - HIS MINSA, Dirección de Estadística GERESA Cusco. \\}
\hyperlink{cobertura_vacuna_provincias}{\beamergotobutton{regresar}}
\end{frame}

\begin{frame}[label=vacunas_40]
	\frametitle{Porcentaje de Cobertura de Vacunación de 30 a 39 años, Provincias de Cusco}
	\vspace{-.5cm}
	\begin{center}
		\includegraphics[width=1.0\linewidth, trim={.2cm .5cm .2cm .2cm},clip]{../figuras/vacunacion_provincial_edad_practica_4.pdf}
	\end{center}
	{\tiny Fuente de datos: SICOVAC - HIS MINSA, Dirección de Estadística GERESA Cusco. \\}
\hyperlink{cobertura_vacuna_provincias}{\beamergotobutton{regresar}}
\end{frame}

\begin{frame}[label=vacunas_30]
	\frametitle{Porcentaje de Cobertura de Vacunación de 18 a 29 años, Provincias de Cusco}
	\vspace{-.5cm}
	\begin{center}
		\includegraphics[width=1.0\linewidth, trim={.2cm .5cm .2cm .2cm},clip]{../figuras/vacunacion_provincial_edad_practica_3.pdf}
	\end{center}
	{\tiny Fuente de datos: SICOVAC - HIS MINSA, Dirección de Estadística GERESA Cusco. \\}
\hyperlink{cobertura_vacuna_provincias}{\beamergotobutton{regresar}}
\end{frame}

\begin{frame}[label=vacunas_20]
	\frametitle{Porcentaje de Cobertura de Vacunación de 12 a 17 años, Provincias de Cusco}
	\vspace{-.5cm}
	\begin{center}
		\includegraphics[width=1.0\linewidth, trim={.2cm .5cm .2cm .2cm},clip]{../figuras/vacunacion_provincial_edad_practica_2.pdf}
	\end{center}
	{\tiny Fuente de datos: SICOVAC - HIS MINSA, Dirección de Estadística GERESA Cusco. \\}
\hyperlink{cobertura_vacuna_provincias}{\beamergotobutton{regresar}}

\end{frame}

\begin{frame}[label=vacunas_10]
	\frametitle{Porcentaje de Cobertura de Vacunación de 05 a 11 años, Provincias de Cusco}
	\vspace{-.5cm}
	\begin{center}
		\includegraphics[width=1.0\linewidth, trim={.2cm .5cm .2cm .2cm},clip]{../figuras/vacunacion_provincial_edad_practica_1.pdf}
	\end{center}
	{\tiny Fuente de datos: SICOVAC - HIS MINSA, Dirección de Estadística GERESA Cusco. \\}
\hyperlink{cobertura_vacuna_provincias}{\beamergotobutton{regresar}}
\end{frame}

\subsection{Acomayo}
\begin{frame}[label=Acomayo]
	\frametitle{Curva Epidemica de Sintomaticos por Tipo de Prueba, Provincia Acomayo}
	\vspace{-.5cm}
	\begin{center}
		\includegraphics[width=0.8\linewidth, trim={0cm .5cm 0cm 0.2cm},clip]{../figuras/sinto_prueba20_21_1.png}
	\end{center}
	{\tiny Fuente de datos: SISCOVID, NOTICOVID, SINADEF.}
	\hyperlink{TipoPrueba}{\beamergotobutton{regresar}}
\end{frame}

\begin{frame}[label=Acomayo]
	\frametitle{Incidencia y Mortalidad, Provincia Acomayo}
	\vspace{-.5cm}
	\begin{center}
		\includegraphics[width=0.8\linewidth, trim={0cm .5cm 0cm 0.2cm},clip]{../figuras/incidencia_mortalidad_20_21_1.png}
	\end{center}
	{\tiny Fuente de datos: SISCOVID, NOTICOVID, SINADEF.}
\end{frame}

\begin{frame}
	\frametitle{Tasa de Positividad, Provincia Acomayo}
	\vspace{-.5cm}
	\begin{center}
		\includegraphics[width=0.8\linewidth, trim={0cm .5cm 0cm 0.2cm},clip]{../figuras/positividad_20_21_1.png}
	\end{center}
	{\tiny Fuente de datos: SISCOVID, NOTICOVID.}
\end{frame}

\begin{frame}
	\frametitle{Exceso de Defunciones por Todas las Causas, Provincia Acomayo}
	\vspace{-.5cm}
	\begin{center}	
		\includegraphics[width=0.8\linewidth, trim={0cm .5cm 0cm 0.2cm},clip]{../figuras/exceso_1.pdf}
	\end{center}
	{\tiny Fuente de datos: SINADEF - NOTICOVID.}
	
	\hyperlink{indicadores_provinciales}{\beamergotobutton{regresar}}
\end{frame}

\begin{frame}
	\frametitle{Cobertura de Vacunación, Provincia Acomayo}
	\vspace{-.5cm}
	\begin{center}	
		\includegraphics[width=0.8\linewidth, trim={0cm .5cm 0cm 0.2cm},clip]{../figuras/vacunacion__provincias_1.pdf}
	\end{center}
	{\tiny Fuente de datos: SICOVAC - HIS MINSA, Dirección de Estadística GERESA Cusco.}
	
	\hyperlink{indicadores_provinciales}{\beamergotobutton{regresar}}
\end{frame}

\subsection{Anta}
\begin{frame}[label=Anta]
	\frametitle{Curva Epidemica de Sintomaticos por Tipo de Prueba, Provincia Anta}
	\vspace{-.5cm}
	\begin{center}
		\includegraphics[width=0.8\linewidth, trim={0cm .5cm 0cm 0.2cm},clip]{../figuras/sinto_prueba20_21_2.png}
	\end{center}
	{\tiny Fuente de datos: SISCOVID, NOTICOVID, SINADEF.}
	\hyperlink{TipoPrueba}{\beamergotobutton{regresar}}
\end{frame}

\begin{frame}[label=Anta]
	\frametitle{Incidencia y Mortalidad, Provincia Anta}
	\vspace{-.5cm}
	\begin{center}
		\includegraphics[width=0.8\linewidth, trim={0cm .5cm 0cm 0.2cm},clip]{../figuras/incidencia_mortalidad_20_21_2.png}
	\end{center}
	{\tiny Fuente de datos: SISCOVID, NOTICOVID, SINADEF.}
\end{frame}

\begin{frame}
	\frametitle{Tasa de Positividad, Provincia Anta}
	\vspace{-.5cm}
	\begin{center}
		\includegraphics[width=0.8\linewidth, trim={0cm .5cm 0cm 0.2cm},clip]{../figuras/positividad_20_21_2.png}
	\end{center}
	{\tiny Fuente de datos: SISCOVID, NOTICOVID.}
\end{frame}

\begin{frame}
	\frametitle{Exceso de Defunciones por Todas las Causas, Provincia Anta}
	\vspace{-.5cm}
	\begin{center}
		\includegraphics[width=0.8\linewidth, trim={0cm .5cm 0cm 0.2cm},clip]{../figuras/exceso_2.pdf}
	\end{center}
	{\tiny Fuente de datos: SINADEF - NOTICOVID.}
	
	\hyperlink{indicadores_provinciales}{\beamergotobutton{regresar}}
\end{frame}

\begin{frame}
	\frametitle{Cobertura de Vacunación, Provincia Anta}
	\vspace{-.5cm}
	\begin{center}	
		\includegraphics[width=0.8\linewidth, trim={0cm .5cm 0cm 0.2cm},clip]{../figuras/vacunacion__provincias_2.pdf}
	\end{center}
	{\tiny Fuente de datos: SICOVAC - HIS MINSA, Dirección de Estadística GERESA Cusco.}
	
	\hyperlink{indicadores_provinciales}{\beamergotobutton{regresar}}
\end{frame}

\subsection{Calca}
\begin{frame}[label=Calca]
	\frametitle{Curva Epidemica de Sintomaticos por Tipo de Prueba, Provincia Calca}
	\vspace{-.5cm}
	\begin{center}
		\includegraphics[width=0.8\linewidth, trim={0cm .5cm 0cm 0.2cm},clip]{../figuras/sinto_prueba20_21_3.png}
	\end{center}
	{\tiny Fuente de datos: SISCOVID, NOTICOVID, SINADEF.}
	\hyperlink{TipoPrueba}{\beamergotobutton{regresar}}
\end{frame}

\begin{frame}[label=Calca]
	\frametitle{Incidencia y Mortalidad, Provincia Calca}
	\vspace{-.5cm}
	\begin{center}
		\includegraphics[width=0.8\linewidth, trim={0cm .5cm 0cm 0.2cm},clip]{../figuras/incidencia_mortalidad_20_21_3.png}
	\end{center}
	{\tiny Fuente de datos: SISCOVID, NOTICOVID, SINADEF.}
\end{frame}

\begin{frame}
	\frametitle{Tasa de Positividad, Provincia Calca}
	\vspace{-.5cm}
	\begin{center}
		\includegraphics[width=0.8\linewidth, trim={0cm .5cm 0cm 0.2cm},clip]{../figuras/positividad_20_21_3.png}
	\end{center}
	{\tiny Fuente de datos: SISCOVID, NOTICOVID.}
\end{frame}

\begin{frame}
	\frametitle{Exceso de Defunciones por Todas las Causas, provincia Calca}
	\vspace{-.5cm}
	\begin{center}
		\includegraphics[width=0.8\linewidth, trim={0cm .5cm 0cm 0.2cm},clip]{../figuras/exceso_3.pdf}
	\end{center}
	{\tiny Fuente de datos: SINADEF - NOTICOVID.}
	
	\hyperlink{indicadores_provinciales}{\beamergotobutton{regresar}}
\end{frame}

\begin{frame}
	\frametitle{Cobertura de Vacunación, Provincia Calca}
	\vspace{-.5cm}
	\begin{center}	
		\includegraphics[width=0.8\linewidth, trim={0cm .5cm 0cm 0.2cm},clip]{../figuras/vacunacion__provincias_3.pdf}
	\end{center}
	{\tiny Fuente de datos: SICOVAC - HIS MINSA, Dirección de Estadística GERESA Cusco.}
	
	\hyperlink{indicadores_provinciales}{\beamergotobutton{regresar}}
\end{frame}

\subsection{Canas}
\begin{frame}[label=Canas]
	\frametitle{Curva Epidemica de Sintomaticos por Tipo de Prueba, Provincia Canas}
	\vspace{-.5cm}
	\begin{center}
		\includegraphics[width=0.8\linewidth, trim={0cm .5cm 0cm 0.2cm},clip]{../figuras/sinto_prueba20_21_4.png}
	\end{center}
	{\tiny Fuente de datos: SISCOVID, NOTICOVID, SINADEF.}
	\hyperlink{TipoPrueba}{\beamergotobutton{regresar}}
\end{frame}

\begin{frame}[label=Canas]
	\frametitle{Incidencia y Mortalidad, Provincia Canas}
	\vspace{-.5cm}
	\begin{center}
		\includegraphics[width=0.8\linewidth, trim={0cm .5cm 0cm 0.2cm},clip]{../figuras/incidencia_mortalidad_20_21_4.png}
	\end{center}
	{\tiny Fuente de datos: SISCOVID, NOTICOVID, SINADEF}
\end{frame}

\begin{frame}
	\frametitle{Tasa de positividad, Provincia Canas}
	\vspace{-.5cm}
	\begin{center}
		\includegraphics[width=0.8\linewidth, trim={0cm .5cm 0cm 0.2cm},clip]{../figuras/positividad_20_21_4.png}
	\end{center}
	{\tiny Fuente de datos: SISCOVID, NOTICOVID.}
\end{frame}

\begin{frame}
	\frametitle{Exceso de Defunciones por Todas las Causas, provincia Canas}
	\vspace{-.5cm}
	\begin{center}
		\includegraphics[width=0.8\linewidth, trim={0cm .5cm 0cm 0.2cm},clip]{../figuras/exceso_4.pdf}
	\end{center}
	{\tiny Fuente de datos: SINADEF - NOTICOVID.}
	
	\hyperlink{indicadores_provinciales}{\beamergotobutton{regresar}}
\end{frame}

\begin{frame}
	\frametitle{Cobertura de Vacunación, Provincia Canas}
	\vspace{-.5cm}
	\begin{center}	
		\includegraphics[width=0.8\linewidth, trim={0cm .5cm 0cm 0.2cm},clip]{../figuras/vacunacion__provincias_4.pdf}
	\end{center}
	{\tiny Fuente de datos: SICOVAC - HIS MINSA, Dirección de Estadística GERESA Cusco.}
	
	\hyperlink{indicadores_provinciales}{\beamergotobutton{regresar}}
\end{frame}

\subsection{Canchis}
\begin{frame}[label=Canchis]
	\frametitle{Curva Epidemica de Sintomaticos por Tipo de Prueba, Provincia Canchis}
	\vspace{-.5cm}
	\begin{center}
		\includegraphics[width=0.8\linewidth, trim={0cm .5cm 0cm 0.2cm},clip]{../figuras/sinto_prueba20_21_5.png}
	\end{center}
	{\tiny Fuente de datos: SISCOVID, NOTICOVID, SINADEF.}
	\hyperlink{TipoPrueba}{\beamergotobutton{regresar}}
\end{frame}

\begin{frame}[label=Canchis]
	\frametitle{Incidencia y Mortalidad, Provincia Canchis}
	\vspace{-.5cm}
	\begin{center}
		\includegraphics[width=0.8\linewidth, trim={0cm .5cm 0cm 0.2cm},clip]{../figuras/incidencia_mortalidad_20_21_5.png}
	\end{center}
	{\tiny Fuente de datos: SISCOVID, NOTICOVID, SINADEF}
\end{frame}

\begin{frame}
	\frametitle{Tasa de Positividad, Provincia Canchis}
	\vspace{-.5cm}
	\begin{center}
		\includegraphics[width=0.8\linewidth, trim={0cm .5cm 0cm 0.2cm},clip]{../figuras/positividad_20_21_5.png}
	\end{center}
	{\tiny Fuente de datos: SISCOVID, NOTICOVID.}
\end{frame}

\begin{frame}
	\frametitle{Exceso de Defunciones por Todas las Causas, Provincia Canchis}
	\vspace{-.5cm}
	\begin{center}
		\includegraphics[width=0.8\linewidth, trim={0cm .5cm 0cm 0.2cm},clip]{../figuras/exceso_5.pdf}
	\end{center}
	{\tiny Fuente de datos: SINADEF - NOTICOVID.}
	
	\hyperlink{indicadores_provinciales}{\beamergotobutton{regresar}}
\end{frame}

\begin{frame}
	\frametitle{Cobertura de Vacunación, Provincia Canchis}
	\vspace{-.5cm}
	\begin{center}	
		\includegraphics[width=0.8\linewidth, trim={0cm .5cm 0cm 0.2cm},clip]{../figuras/vacunacion__provincias_5.pdf}
	\end{center}
	{\tiny Fuente de datos: SICOVAC - HIS MINSA, Dirección de Estadística GERESA Cusco.}
	
	\hyperlink{indicadores_provinciales}{\beamergotobutton{regresar}}
\end{frame}

\subsection{Chumbivilcas}
\begin{frame}[label=Chumbivilcas]
	\frametitle{Curva Epidemica de Sintomaticos por Tipo de Prueba, Provincia Chumbivilcas}
	\vspace{-.5cm}
	\begin{center}
		\includegraphics[width=0.8\linewidth, trim={0cm .5cm 0cm 0.2cm},clip]{../figuras/sinto_prueba20_21_6.png}
	\end{center}
	{\tiny Fuente de datos: SISCOVID, NOTICOVID, SINADEF.}
	\hyperlink{TipoPrueba}{\beamergotobutton{regresar}}
\end{frame}

\begin{frame}[label=Chumbivilcas]
	\frametitle{Incidencia y Mortalidad, Provincia Chumbivilcas}
	\vspace{-.5cm}
	\begin{center}
		\includegraphics[width=0.8\linewidth, trim={0cm .5cm 0cm 0.2cm},clip]{../figuras/incidencia_mortalidad_20_21_6.png}
	\end{center}
	{\tiny Fuente de datos: SISCOVID, NOTICOVID, SINADEF}
\end{frame}

\begin{frame}
	\frametitle{Tasa de Positividad, Provincia Chumbivilcas}
	\vspace{-.5cm}
	\begin{center}
		\includegraphics[width=0.8\linewidth, trim={0cm .5cm 0cm 0.2cm},clip]{../figuras/positividad_20_21_6.png}
	\end{center}
	{\tiny Fuente de datos: SISCOVID, NOTICOVID.}
\end{frame}

\begin{frame}
	\frametitle{Exceso de Defunciones por Todas las Causas, Provincia Chumbivilcas}
	\vspace{-.5cm}
	\begin{center}
		\includegraphics[width=0.8\linewidth, trim={0cm .5cm 0cm 0.2cm},clip]{../figuras/exceso_6.pdf}
	\end{center}
	{\tiny Fuente de datos: SINADEF - NOTICOVID.}
	
	\hyperlink{indicadores_provinciales}{\beamergotobutton{regresar}}
\end{frame}

\begin{frame}
	\frametitle{Cobertura de Vacunación, Provincia Chumbivilcas}
	\vspace{-.5cm}
	\begin{center}	
		\includegraphics[width=0.8\linewidth, trim={0cm .5cm 0cm 0.2cm},clip]{../figuras/vacunacion__provincias_6.pdf}
	\end{center}
	{\tiny Fuente de datos: SICOVAC - HIS MINSA, Dirección de Estadística GERESA Cusco.}
	
	\hyperlink{indicadores_provinciales}{\beamergotobutton{regresar}}
\end{frame}

\subsection{Cusco}
\begin{frame}[label=Cusco]
	\frametitle{Curva Epidemica de Sintomaticos por Tipo de Prueba, Provincia Cusco}
	\vspace{-.5cm}
	\begin{center}
		\includegraphics[width=0.8\linewidth, trim={0cm .5cm 0cm 0.2cm},clip]{../figuras/sinto_prueba20_21_7.png}
	\end{center}
	{\tiny Fuente de datos: SISCOVID, NOTICOVID, SINADEF.}
	\hyperlink{TipoPrueba}{\beamergotobutton{regresar}}
\end{frame}

\begin{frame}[label=Cusco]
	\frametitle{Incidencia y Mortalidad, Provincia Cusco}
	\vspace{-.5cm}
	\begin{center}
		\includegraphics[width=0.8\linewidth, trim={0cm .5cm 0cm 0.2cm},clip]{../figuras/incidencia_mortalidad_20_21_7.png}
	\end{center}
	{\tiny Fuente de datos: SISCOVID, NOTICOVID, SINADEF.}
\end{frame}

\begin{frame}
	\frametitle{Tasa de Positividad, Provincia Cusco}
	\vspace{-.5cm}
	\begin{center}
		\includegraphics[width=0.8\linewidth, trim={0cm .5cm 0cm 0.2cm},clip]{../figuras/positividad_20_21_7.png}
	\end{center}
	{\tiny Fuente de datos: SISCOVID, NOTICOVID.}
\end{frame}

\begin{frame}
	\frametitle{Exceso de Defunciones por Todas las Causas, Provincia Cusco}
	\vspace{-.5cm}
	\begin{center}
		\includegraphics[width=0.8\linewidth, trim={0cm .5cm 0cm 0.2cm},clip]{../figuras/exceso_7.pdf}
	\end{center}
	{\tiny Fuente de datos: SINADEF - NOTICOVID.}
	
	\hyperlink{indicadores_provinciales}{\beamergotobutton{regresar}}
\end{frame}

\begin{frame}
	\frametitle{Cobertura de Vacunación, Provincia Cusco}
	\vspace{-.5cm}
	\begin{center}	
		\includegraphics[width=0.8\linewidth, trim={0cm .5cm 0cm 0.2cm},clip]{../figuras/vacunacion__provincias_7.pdf}
	\end{center}
	{\tiny Fuente de datos: SICOVAC - HIS MINSA, Dirección de Estadística GERESA Cusco.}
	
	\hyperlink{indicadores_provinciales}{\beamergotobutton{regresar}}
\end{frame}

\subsection{Espinar}
\begin{frame}[label=Espinar]
	\frametitle{Curva Epidemica de Sintomaticos por Tipo de Prueba, Provincia Espinar}
	\vspace{-.5cm}
	\begin{center}
		\includegraphics[width=0.8\linewidth, trim={0cm .5cm 0cm 0.2cm},clip]{../figuras/sinto_prueba20_21_8.png}
	\end{center}
	{\tiny Fuente de datos: SISCOVID, NOTICOVID, SINADEF.}
	\hyperlink{TipoPrueba}{\beamergotobutton{regresar}}
\end{frame}

\begin{frame}[label=Espinar]
	\frametitle{Incidencia y Mortalidad, Provincia Espinar}
	\vspace{-.5cm}
	\begin{center}
		\includegraphics[width=0.8\linewidth, trim={0cm .5cm 0cm 0.2cm},clip]{../figuras/incidencia_mortalidad_20_21_8.png}
	\end{center}
	{\tiny Fuente de datos: SISCOVID, NOTICOVID, SINADEF.}
\end{frame}

\begin{frame}
	\frametitle{Tasa de Positividad, Provincia Espinar}
	\vspace{-.5cm}
	\begin{center}
		\includegraphics[width=0.8\linewidth, trim={0cm .5cm 0cm 0.2cm},clip]{../figuras/positividad_20_21_8.png}
	\end{center}
	{\tiny Fuente de datos: SISCOVID, NOTICOVID.}
\end{frame}

\begin{frame}
	\frametitle{Exceso de Defunciones por Todas las Causas, Provincia Espinar}
	\vspace{-.5cm}
	\begin{center}
		\includegraphics[width=0.8\linewidth, trim={0cm .5cm 0cm 0.2cm},clip]{../figuras/exceso_8.pdf}
	\end{center}
	{\tiny Fuente de datos: SINADEF - NOTICOVID.}
	
	\hyperlink{indicadores_provinciales}{\beamergotobutton{regresar}}
\end{frame}

\begin{frame}
	\frametitle{Cobertura de Vacunación, Provincia Espinar}
	\vspace{-.5cm}
	\begin{center}	
		\includegraphics[width=0.8\linewidth, trim={0cm .5cm 0cm 0.2cm},clip]{../figuras/vacunacion__provincias_8.pdf}
	\end{center}
	{\tiny Fuente de datos: SICOVAC - HIS MINSA, Dirección de Estadística GERESA Cusco.}
	
	\hyperlink{indicadores_provinciales}{\beamergotobutton{regresar}}
\end{frame}

\subsection{La Convención}
\begin{frame}[label=laconvecion]
	\frametitle{Curva Epidémica de Sintomáticos por Tipo de Prueba, Provincia La Convención}
	\vspace{-.5cm}
	\begin{center}
		\includegraphics[width=0.8\linewidth, trim={0cm .5cm 0cm 0.2cm},clip]{../figuras/sinto_prueba20_21_9.png}
	\end{center}
	{\tiny Fuente de datos: SISCOVID, NOTICOVID, SINADEF.}
	\hyperlink{TipoPrueba}{\beamergotobutton{regresar}}
\end{frame}
\begin{frame}[label=laconvencion]
	\frametitle{Incidencia y Mortalidad, Provincia La Convención}
	\vspace{-.5cm}
	\begin{center}
		\includegraphics[width=0.8\linewidth, trim={0cm .5cm 0cm 0.2cm},clip]{../figuras/incidencia_mortalidad_20_21_9.png}
	\end{center}
	{\tiny Fuente de datos: SISCOVID, NOTICOVID, SINADEF.}
\end{frame}

\begin{frame}
	\frametitle{Tasa de Positividad,Provincia La Convención}
	\vspace{-.5cm}
	\begin{center}
		\includegraphics[width=0.8\linewidth, trim={0cm .5cm 0cm 0.2cm},clip]{../figuras/positividad_20_21_9.png}
	\end{center}
	{\tiny Fuente de datos: SISCOVID, NOTICOVID.}
\end{frame}

\begin{frame}
	\frametitle{Exceso de Defunciones por Todas las Causas, Provincia La Convención}
	\vspace{-.5cm}
	\begin{center}
		\includegraphics[width=0.8\linewidth, trim={0cm .5cm 0cm 0.2cm},clip]{../figuras/exceso_9.pdf}
	\end{center}
	{\tiny Fuente de datos: SINADEF - NOTICOVID.}
	
	\hyperlink{indicadores_provinciales}{\beamergotobutton{regresar}}
\end{frame}

\begin{frame}
	\frametitle{Cobertura de Vacunación, Provincia La Convención}
	\vspace{-.5cm}
	\begin{center}	
		\includegraphics[width=0.8\linewidth, trim={0cm .5cm 0cm 0.2cm},clip]{../figuras/vacunacion__provincias_9.pdf}
	\end{center}
	{\tiny Fuente de datos: SICOVAC - HIS MINSA, Dirección de Estadística GERESA Cusco.}
	
	\hyperlink{indicadores_provinciales}{\beamergotobutton{regresar}}
\end{frame}

\subsection{Paruro}
\begin{frame}[label=Paruro]
	\frametitle{Curva Epidémica de Sintomáticos por Tipo de Prueba, Provincia Paruro}
	\vspace{-.5cm}
	\begin{center}
		\includegraphics[width=0.8\linewidth, trim={0cm .5cm 0cm 0.2cm},clip]{../figuras/sinto_prueba20_21_10.png}
	\end{center}
	{\tiny Fuente de datos: SISCOVID, NOTICOVID, SINADEF.}
	\hyperlink{TipoPrueba}{\beamergotobutton{regresar}}
\end{frame}

\begin{frame}[label=Paruro]
	\frametitle{Incidencia y Mortalidad, Provincia Paruro}
	\vspace{-.5cm}
	\begin{center}
		\includegraphics[width=0.8\linewidth, trim={0cm .5cm 0cm 0.2cm},clip]{../figuras/incidencia_mortalidad_20_21_10.png}
	\end{center}
	{\tiny Fuente de datos: SISCOVID, NOTICOVID, SINADEF.}
\end{frame}

\begin{frame}
	\frametitle{Tasa de Positividad, Provincia Paruro}
	\vspace{-.5cm}
	\begin{center}
		\includegraphics[width=0.8\linewidth, trim={0cm .5cm 0cm 0.2cm},clip]{../figuras/positividad_20_21_10.png}
	\end{center}
	{\tiny Fuente de datos: SISCOVID, NOTICOVID.}
\end{frame}

\begin{frame}
	\frametitle{Exceso de Defunciones por Todas las Causas, Provincia Paruro}
	\vspace{-.5cm}
	\begin{center}
		\includegraphics[width=0.8\linewidth, trim={0cm .5cm 0cm 0.2cm},clip]{../figuras/exceso_10.pdf}
	\end{center}
	{\tiny Fuente de datos: SINADEF - NOTICOVID.}
	
	\hyperlink{indicadores_provinciales}{\beamergotobutton{regresar}}
\end{frame}

\begin{frame}
	\frametitle{Cobertura de Vacunación, Provincia Paruro}
	\vspace{-.5cm}
	\begin{center}	
		\includegraphics[width=0.8\linewidth, trim={0cm .5cm 0cm 0.2cm},clip]{../figuras/vacunacion__provincias_10.pdf}
	\end{center}
	{\tiny Fuente de datos: SICOVAC - HIS MINSA, Dirección de Estadística GERESA Cusco.}
	
	\hyperlink{indicadores_provinciales}{\beamergotobutton{regresar}}
\end{frame}

\subsection{Paucartambo}
\begin{frame}[label=Paucartambo]
	\frametitle{Curva Epidémica de Sintomáticos por Tipo de Prueba, Provincia Paucartambo}
	\vspace{-.5cm}
	\begin{center}
		\includegraphics[width=0.8\linewidth, trim={0cm .5cm 0cm 0.2cm},clip]{../figuras/sinto_prueba20_21_11.png}
	\end{center}
	{\tiny Fuente de datos: SISCOVID, NOTICOVID, SINADEF.}
	\hyperlink{TipoPrueba}{\beamergotobutton{regresar}}
\end{frame}

\begin{frame}[label=Paucartambo]
	\frametitle{Incidencia y Mortalidad, Provincia Paucartambo}
	\vspace{-.5cm}
	\begin{center}
		\includegraphics[width=0.8\linewidth, trim={0cm .5cm 0cm 0.2cm},clip]{../figuras/incidencia_mortalidad_20_21_11.png}
	\end{center}
	{\tiny Fuente de datos: SISCOVID, NOTICOVID, SINADEF.}
\end{frame}

\begin{frame}
	\frametitle{Tasa de Positividad, Provincia Paucartambo}
	\vspace{-.5cm}
	\begin{center}
		\includegraphics[width=0.8\linewidth, trim={0cm .5cm 0cm 0.2cm},clip]{../figuras/positividad_20_21_11.png}
	\end{center}
	{\tiny Fuente de datos: SISCOVID, NOTICOVID.}
\end{frame}

\begin{frame}
	\frametitle{Exceso de Defunciones por Todas las causas, provincia Paucartambo}
	\vspace{-.5cm}
	\begin{center}
		\includegraphics[width=0.8\linewidth, trim={0cm .5cm 0cm 0.2cm},clip]{../figuras/exceso_11.pdf}
	\end{center}
	{\tiny Fuente de datos: SINADEF - NOTICOVID.}
	
	\hyperlink{indicadores_provinciales}{\beamergotobutton{regresar}}
\end{frame}

\begin{frame}
	\frametitle{Cobertura de Vacunación, Provincia Paucartambo}
	\vspace{-.5cm}
	\begin{center}	
		\includegraphics[width=0.8\linewidth, trim={0cm .5cm 0cm 0.2cm},clip]{../figuras/vacunacion__provincias_11.pdf}
	\end{center}
	{\tiny Fuente de datos: SICOVAC - HIS MINSA, Dirección de Estadística GERESA Cusco.}
	
	\hyperlink{indicadores_provinciales}{\beamergotobutton{regresar}}
\end{frame}


\subsection{Quispicanchi}
\begin{frame}[label=Quispicanchi]
	\frametitle{Curva Epidémica de Sintomáticos por Tipo de Prueba, Provincia Quispicanchi}
	\vspace{-.5cm}
	\begin{center}
		\includegraphics[width=0.8\linewidth, trim={0cm .5cm 0cm 0.2cm},clip]{../figuras/sinto_prueba20_21_12.png}
	\end{center}
	{\tiny Fuente de datos: SISCOVID, NOTICOVID, SINADEF.}
	\hyperlink{TipoPrueba}{\beamergotobutton{regresar}}
\end{frame}

\begin{frame}[label=Quispicanchi]
	\frametitle{Incidencia y Mortalidad, Provincia Quispicanchi}
	\vspace{-.5cm}
	\begin{center}
		\includegraphics[width=0.8\linewidth, trim={0cm .5cm 0cm 0.2cm},clip]{../figuras/incidencia_mortalidad_20_21_12.png}
	\end{center}
	{\tiny Fuente de datos: SISCOVID, NOTICOVID, SINADEF.}
\end{frame}

\begin{frame}
	\frametitle{Tasa de Positividad, Provincia Quispicanchi}
	\vspace{-.5cm}
	\begin{center}
		\includegraphics[width=0.8\linewidth, trim={0cm .5cm 0cm 0.2cm},clip]{../figuras/positividad_20_21_12.png}
	\end{center}
	{\tiny Fuente de datos: SISCOVID, NOTICOVID.}
\end{frame}

\begin{frame}
	\frametitle{Exceso de Defunciones por Todas las Causas, Provincia Quispicanchi}
	\vspace{-.5cm}
	\begin{center}
		\includegraphics[width=0.8\linewidth, trim={0cm .5cm 0cm 0.2cm},clip]{../figuras/exceso_12.pdf}
	\end{center}
	{\tiny Fuente de datos: SINADEF - NOTICOVID.}
	
	\hyperlink{indicadores_provinciales}{\beamergotobutton{regresar}}
\end{frame}

\begin{frame}
	\frametitle{Cobertura de Vacunación, Provincia Quispicanchis}
	\vspace{-.5cm}
	\begin{center}	
		\includegraphics[width=0.8\linewidth, trim={0cm .5cm 0cm 0.2cm},clip]{../figuras/vacunacion__provincias_12.pdf}
	\end{center}
	{\tiny Fuente de datos: SICOVAC - HIS MINSA, Dirección de Estadística GERESA Cusco.}
	
	\hyperlink{indicadores_provinciales}{\beamergotobutton{regresar}}
\end{frame}

\subsection{Urubamba}
\begin{frame}[label=Urubamba]
	\frametitle{Curva Epidémica de Sintomáticos por Tipo de Prueba, Provincia Urubamba}
	\vspace{-.5cm}
	\begin{center}
		\includegraphics[width=0.8\linewidth, trim={0cm .5cm 0cm 0.2cm},clip]{../figuras/sinto_prueba20_21_13.png}
	\end{center}
	{\tiny Fuente de datos: SISCOVID, NOTICOVID, SINADEF.}
	\hyperlink{TipoPrueba}{\beamergotobutton{regresar}}
\end{frame}

\begin{frame}[label=Urubamba]
	\frametitle{\large Incidencia y Mortalidad, Provincia Urubamba}
	\vspace{-.5cm}
	\begin{center}
		\includegraphics[width=0.8\linewidth, trim={0cm .5cm 0cm 0.2cm},clip]{../figuras/incidencia_mortalidad_20_21_13.png}
	\end{center}
	{\tiny Fuente de datos: SISCOVID, NOTICOVID, SINADEF.}
\end{frame}

\begin{frame}
	\frametitle{Tasa de Positividad, Provincia Urubamba}
	\vspace{-.5cm}
	\begin{center}
		\includegraphics[width=0.8\linewidth, trim={0cm .5cm 0cm 0.2cm},clip]{../figuras/positividad_20_21_13.png}
	\end{center}
	{\tiny Fuente de datos: SISCOVID, NOTICOVID.}
\end{frame}

\begin{frame}
	\frametitle{Exceso de Defunciones por Todas las Causas, Provincia Urubamba}
	\vspace{-.5cm}
	\begin{center}
		\includegraphics[width=0.8\linewidth, trim={0cm .5cm 0cm 0.2cm},clip]{../figuras/exceso_13.pdf}
	\end{center}
	{\tiny Fuente de datos: SINADEF - NOTICOVID.} \hyperlink{indice}{\beamergotobutton{Índice}} 
	
	\hyperlink{indicadores_provinciales}{\beamergotobutton{regresar}}
\end{frame}

\begin{frame}
	\frametitle{Cobertura de Vacunación, Provincia Urubamba}
	\vspace{-.5cm}
	\begin{center}	
		\includegraphics[width=0.8\linewidth, trim={0cm .5cm 0cm 0.2cm},clip]{../figuras/vacunacion__provincias_13.pdf}
	\end{center}
	{\tiny Fuente de datos: SICOVAC - HIS MINSA, Dirección de Estadística GERESA Cusco.}
	
	\hyperlink{indicadores_provinciales}{\beamergotobutton{regresar}}
\end{frame}

\subsection{Mapas Variantes}
	\begin{frame}[label=mapa_provincia_cusco]
	\frametitle{Cantidad de Casos Variantes en la Provincia Cusco, 2021-2022}
	\begin{center}
		\includegraphics[width=0.65\linewidth]{../figuras/variantes_distrital_cusco.pdf}
	\end{center}
	{\tiny Fuente de datos: NETLAB Cusco, UNSAAC, UPCH.}
	
	\hyperlink{mapa_variantes}{\beamergotobutton{regresar}}
	\end{frame}

	\begin{frame}[label=mapa_distrital]
	\frametitle{Cantidad de Casos Variantes en los Distritos de la Región Cusco, 2021-2022}
	\begin{center}
		\includegraphics[width=0.55\linewidth]{../figuras/variantes_distrital.pdf}
	\end{center}
	{\tiny Fuente de datos: NETLAB Cusco, UNSAAC, UPCH.}
	
	\hyperlink{mapa_variantes}{\beamergotobutton{regresar}}
	\end{frame}

	\begin{frame}[label=mapa_lambda]
		\frametitle{Cantidad de Casos Variantes \textbf{Lambda} por Provincias, 2021-2022}
		\begin{center}
			\includegraphics[width=0.55\linewidth]{../figuras/variantes_provincial_lambda.pdf}
		\end{center}
		{\tiny Fuente de datos: NETLAB Cusco, UNSAAC, UPCH.}
		
		\hyperlink{mapa_variantes}{\beamergotobutton{regresar}}
	\end{frame}

	\begin{frame}[label=mapa_gamma]
		\frametitle{Cantidad de Casos  Variante \textbf{Gamma} por Provincias, 2021-2022}
		\begin{center}
			\includegraphics[width=0.55\linewidth]{../figuras/variantes_provincial_gamma.pdf}
		\end{center}
		{\tiny Fuente de datos: NETLAB Cusco, UNSAAC, UPCH.}
		
		\hyperlink{mapa_variantes}{\beamergotobutton{regresar}}
	\end{frame}
	
	\begin{frame}[label=mapa_delta]
		\frametitle{Cantidad de Casos  Variante \textbf{Delta} por Provincias, 2021-2022}
		\begin{center}
			\includegraphics[width=0.55\linewidth]{../figuras/variantes_provincial_delta.pdf}
		\end{center}
		{\tiny Fuente de datos: NETLAB Cusco, UNSAAC, UPCH.}
		
		\hyperlink{mapa_variantes}{\beamergotobutton{regresar}}
	\end{frame}

	\begin{frame}[label=mapa_omicron]
	\frametitle{Cantidad de Casos  Variante \textbf{Omicron} por Provincias, 2021-2022}
	\begin{center}
		\includegraphics[width=0.55\linewidth]{../figuras/variantes_provincial_omicron.pdf}
	\end{center}
	{\tiny Fuente de datos: NETLAB Cusco, UNSAAC, UPCH.}
	
	\hyperlink{mapa_variantes}{\beamergotobutton{regresar}}
\end{frame}
%\backupend

\end{document}
